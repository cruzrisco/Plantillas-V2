

This section analyzes related work that: i) presents guidelines on how to report changes in SE replications; ii) presents such changes in an easy-to-interpret way, e.g., through tables; and iii) defines or uses templates or patterns.

Carver \cite{carver2010towards} presents an initial proposal for guidelines on the content of publications reporting replications. In this article, the content that should be included about the original study and about replication is indicated. It is recommended to include a section on changes to the original experiment but without specifying how to report such changes.

Apart from Carver's methodological proposal, several studies have been found that document changes in replication using \emph{ad-hoc} tables. Among them we highlight the following:

Solari \cite{solari2013identifying} refers to identifying experimental incidents that occur during replication. Examples of incidents are unexpected changes in any of the variables defined for the experiment. In the study, five replications of the same experiment carried out by different experimenters in different locations are documented. Incidents are classified by replication and incident category. With the information stored, a summary table is presented that facilitates the analysis and comparison of the replications.

The work of Lung et al. \cite{lung2008difficulty}, is the most related to our proposal. A replication of an experiment with human subjects is documented by presenting a summary table of differences and changes made to adapt the procedure to local circumstances. For each change introduced, the situation in the original experiment, the situation in replication and the reasons for making the change are clearly explained. 

In \cite{1330459}, Almqvist presents a Systematic Literature Review about Replications. It includes a template to collect the configuration of the replications and in one of its rows it jointly describes all the changes made in the replication.

In \cite{apa2014effectiveness}, Apa et al. present a replication of a failure detection experiment. The changes are described, in a section on changes in the original experiment, in textual form and in a table where for each change the situation in the original experiment and in replication is compared.

In \cite{fucci2016external}, Fucci et al. document an external replication of an experiment on evidence-based development. It contains a section of changes to the original configuration where it presents a table with the adjustments made to the base experiment.

In \cite{juristo2012comparing}, Juristo et al. document eight replications of an experiment to evaluate the effectiveness of three verification and validation techniques. A table is presented where for each change, its situation in each replication and the design decision adopted is analyzed.

In \cite{quesada2016empirical}, Quesada et al. describe a controlled experiment on function point analysis and its two replications. The differences in the experimental configuration between the experiment and replications are explained using a table and in textual form.

On the other hand, there are the studies on template definition in other areas within SE. The idea of using templates and patterns to define changes in the replication of experiments is based on the templates proposed by Durán et al. \cite{duran1999requirements} for the elicitation of requirements.  The templates and L-patterns have been successfully applied in several areas; Del Rio et al. \cite{del2012defining} have used them for the definition of business process performance indicators and Segura et al.  \cite{segura2017template} have used them for the definition of metamorphic relationships.

In the description of experiments, the Goal-Question-Metric (GQM) template by Basili \cite{Basili1994}, recommended by Wohlin \cite{wohlin:experimentation}, should be highlighted for the definition of experiment objectives. Finally, in the Design Science Research (DSR) methodology, Wieringa \cite{38631e0608b54d4299d5707f3a78debf} defines a template for the specification of a problem that will lead to future research.
% ------------------------------------------------------
% File    : Jbd-Trabajos.tex
% Content : Comparación de trabajos relacionados
% Date    : 10/Marzo/2018
% Version : 1.0
% Authors : 
% ------------------------------------------------------
% ------------------------------------------------------
% Last update: ...// 
% - Added ..
% ------------------------------------------------------


%\begin{table}[htpb]
\begin{table}
\label{tab:trabajos}
\caption{Comparison of related work}
%\label{tab:1}       % Give a unique label
%\centering
%\small
% p{5 cm} |p{5 cm} |
%\begin{tabular}{| p{1cm} | p{0.6cm} | p{1 cm}| p{1 cm} |  p{1 cm} |}  &\ding{51
%\begin{tabular}{| l | c | c | p{2cm} | c | c |}
\begin{tabular}{| l |c | c | c | c| c | }

\hline

\textbf{Related works} & \textbf{Topic}  & \textbf{ESE} & \textbf{Use template} &  \textbf{Use pattern}   \\ \hline
Carver  \cite{carver2010towards}   & Guidelines & \ding{51}  &\ding{55} &  \ding{55} \\ \hline
Solari \cite{solari2013identifying} & Reports a replication  & \ding{51} &\ding{55} &\ding{55} \\ \hline
Lung et al. \cite{lung2008difficulty} & Reports a replication  & \ding{51} &\ding{55} &\ding{55}   \\ \hline
Almqvist \cite{1330459} & SLR replications & \ding{51} &\ding{55} &\ding{55}  \\ \hline

Apa et al. \cite{apa2014effectiveness} & Reports a replication & \ding{51} &\ding{55} &\ding{55}  \\ \hline
Fucci et al.  \cite{fucci2016external} & Reports a replication & \ding{51} &\ding{55} &\ding{55}  \\ \hline
Juristo et al.  \cite{juristo2012comparing} & Reports a replication & \ding{51} &\ding{55} &\ding{55}  \\ \hline
Quesada et al.  \cite{quesada2016empirical} & Reports a replication & \ding{51} &\ding{55} &\ding{55}  \\ \hline
 
Durán et al. \cite{duran1999requirements} & Elicitation of requirements  & \ding{55} &\ding{51} &\ding{51}  \\ \hline
Del Río et al. \cite{del2012defining} & Performance indicators  & \ding{55} &\ding{51} &\ding{51}   \\ \hline
Segura et al. \cite{segura2017template} & Metamorphic relationships & \ding{55} &\ding{55} &\ding{51}  \\ \hline

Basili \cite{Basili1994} & GQM & \ding{51} &\ding{55} &\ding{51} \\ \hline
Wieringa \cite{38631e0608b54d4299d5707f3a78debf} & DSR & \ding{55} &\ding{55} &\ding{51}  \\ \hline
This work & Templates for changes & \ding{51}  &\ding{51} &\ding{51}  \\ \hline
%\noalign{\smallskip}\hline

\end{tabular}

\end{table}


%76 words



The table \ref{tab:trabajos} analyses, for each of the above-mentioned works: i) the subject matter, ii) their belonging to the Empirical SE area, iii) the use of templates to facilitate reuse and visual presentation, and iv) the use of patterns to facilitate writing.



