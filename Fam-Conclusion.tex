
\textcolor[rgb]{1,0,0}{Pendiente de terminar.} In this work we validate a template including L-patterns to facilitate the specification of replication changes. By systematizing the conception of changes and how to document them, it will be easier for other researchers to understand them.
%The structure of the information in the form of a template and the proposal of standard phrases facilitates the writing of changes in the the replications.

%\textcolor[rgb]{1,0,0}{Quitar Comentario Bea dice ver 2 cuestiones independientes y ¿en que consiste el principio? } In addition, an in-depth knowledge of the changes makes it easier to see whether the fundamental principle of independence based on aggregated data analysis according to Kitchenham \cite{kitchenham2008role}, which promotes the introduction of changes in replication to avoid propagating problems from the original experiment, is met.

The validation of the template in the three series of experiments has allowed us to realise that relevant information was missing as well as to improve the wording or phrases that are proposed as parameters in some of the defined L-patterns. 

Analysing the difficulties encountered in defining each change, emphasis has been placed on: i) discerning whether the changes lead to modifications in the template and/or metamodel and ii) highlighting the missing information in describing the changes so that the changes are not \textbf{under-specified}.

In summary, the main modifications required are:
i) add the fields \emph{date} and \emph{site} where the replication is carried out; ii) changes in \emph{context variables} are included in the  \emph{operationalization dimension}; and iii) the L-pattern of \emph{threat to validity} is modified in order to be able to specify that the validity is \emph{decreased} or \emph{no threat to validity is addressed}.

%\textbf{Due to limitations in the template}. Implies a change in the template (e.g. modify the L-pattern in case the threat is not identified). Some of these situations result in changes in the metamodel.

In the short term, future work will be aimed at developing a web form or a wizard so that experimenters can define the changes in a friendly way and with this information is generated, for example, the \LaTeX\ code to insert in the articles. The web form will allow the use of the template by parts of other researchers who do not know them and test its usefulness.
Further, our intention is to extend the experimental information repository EXEMPLAR \cite{ParejoExemplar2014} with aspects of replications based on metamodel and template proposed and reviewed in this paper.