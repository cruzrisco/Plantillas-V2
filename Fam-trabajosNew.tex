
This section reviews some templates and patterns defined or used in the context of SE.

The idea of using templates and patterns to define changes in the replication of experiments is based on the templates proposed by Durán et al. \cite{duran1999requirements} for the elicitation of requirements. The templates and L-patterns have been successfully applied in several areas within SE. Del Rio et al. \cite{del2012defining} have used them for the definition of business process performance indicators.

In the description of experiments, the Goal-Question-Metric (GQM) template by Basili \cite{Basili1994}, recommended by Wohlin \cite{wohlin:experimentation}, should be highlighted for the definition of experiment objectives.

Segura et al.  \cite{segura2017template} propose a template inspired by the GQM template for the definition of metamorphic relationships. The format is textual rather than tabular to facilitate integration into research work.

Finally, in the Design Science Research (DSR) methodology, Wieringa \cite{38631e0608b54d4299d5707f3a78debf} defines a template for the specification of a problem that will lead to future research:\newline
%% ------------------------------------------------------
% File    : Jbd-Trabajos.tex
% Content : Comparación de trabajos relacionados
% Date    : 10/Marzo/2018
% Version : 1.0
% Authors : 
% ------------------------------------------------------
% ------------------------------------------------------
% Last update: ...// 
% - Added ..
% ------------------------------------------------------


%\begin{table}[htpb]
\begin{table}
\label{tab:trabajos}
\caption{Comparison of related work}
%\label{tab:1}       % Give a unique label
%\centering
%\small
% p{5 cm} |p{5 cm} |
%\begin{tabular}{| p{1cm} | p{0.6cm} | p{1 cm}| p{1 cm} |  p{1 cm} |}  &\ding{51
%\begin{tabular}{| l | c | c | p{2cm} | c | c |}
\begin{tabular}{| l |c | c | c | c| c | }

\hline

\textbf{Related works} & \textbf{Topic}  & \textbf{ESE} & \textbf{Use template} &  \textbf{Use pattern}   \\ \hline
Carver  \cite{carver2010towards}   & Guidelines & \ding{51}  &\ding{55} &  \ding{55} \\ \hline
Solari \cite{solari2013identifying} & Reports a replication  & \ding{51} &\ding{55} &\ding{55} \\ \hline
Lung et al. \cite{lung2008difficulty} & Reports a replication  & \ding{51} &\ding{55} &\ding{55}   \\ \hline
Almqvist \cite{1330459} & SLR replications & \ding{51} &\ding{55} &\ding{55}  \\ \hline

Apa et al. \cite{apa2014effectiveness} & Reports a replication & \ding{51} &\ding{55} &\ding{55}  \\ \hline
Fucci et al.  \cite{fucci2016external} & Reports a replication & \ding{51} &\ding{55} &\ding{55}  \\ \hline
Juristo et al.  \cite{juristo2012comparing} & Reports a replication & \ding{51} &\ding{55} &\ding{55}  \\ \hline
Quesada et al.  \cite{quesada2016empirical} & Reports a replication & \ding{51} &\ding{55} &\ding{55}  \\ \hline
 
Durán et al. \cite{duran1999requirements} & Elicitation of requirements  & \ding{55} &\ding{51} &\ding{51}  \\ \hline
Del Río et al. \cite{del2012defining} & Performance indicators  & \ding{55} &\ding{51} &\ding{51}   \\ \hline
Segura et al. \cite{segura2017template} & Metamorphic relationships & \ding{55} &\ding{55} &\ding{51}  \\ \hline

Basili \cite{Basili1994} & GQM & \ding{51} &\ding{55} &\ding{51} \\ \hline
Wieringa \cite{38631e0608b54d4299d5707f3a78debf} & DSR & \ding{55} &\ding{55} &\ding{51}  \\ \hline
This work & Templates for changes & \ding{51}  &\ding{51} &\ding{51}  \\ \hline
%\noalign{\smallskip}\hline

\end{tabular}

\end{table}


%76 words


• Improve \textless a problem context \textgreater  \newline
• by \textless (re)designing an artifact \textgreater \newline
• that satisfies \textless some requirements \textgreater \newline
• in order to \textless help stakeholders achieve some goals \textgreater.



