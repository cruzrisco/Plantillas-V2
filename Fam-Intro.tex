%\gls{SE} acuerdos a nivel de servicio \cite{ruiz2005improving},%\textit{ ``E Experiment''}
%Introduction

The Empirical Software Engineering (ESE) allows the evaluation of new methods, techniques and tools to know the convenience of using them in the development process  \cite{sjoberg2005survey}.
%(Sjøberg \emph{et al.}

Once an artefact has been evaluated for the first time, the study needs to be replicated in different contexts and conditions, not only to consolidate the knowledge acquired, but also to know if its results can be generalized \cite{Baldassarre}.
%Baldassarre \emph{et al.}

Despite the importance of replications, and although the practice of replications has increased in recent years  \cite{da2014replication}, the number of replications in Software Engineering (SE) remains low  \cite{solari2017content}.
%(Da Silva \emph{et al.}
%(Solari \emph{et al.}.

When it comes to identifying the causes of this situation, there are several problems encountered. On the one hand, the lack of single criteria for reporting replications \cite{carver2010towards}. On the other hand, \emph{tacit knowledge}, since the researcher has knowledge that is not explicit in the publication \cite{shull2002replicating}. In addition, there is a lack or incompleteness of the \emph{laboratory packages} that are necessary to be able to make replications \cite{solari2017content}. All this is coupled with the effort and resources needed to carry out an experiment \cite{da2014replication}.

In other areas of knowledge such as psychology, this situation is known as the \emph{replicability crisis} \cite{blanco2017psicologia}. It also includes problems with the use of statistics, large numbers of published false positives and lack of rigour due to individual biases in evidence collection. This situation is also present in ESE \cite{reyes2018statistical} . In order to reduce this crisis, to be rigorous in methodological aspects and an adequate transparency in the transmission of research results is suggested \cite{blanco2017psicologia}. %In order to reduce this crisis, it is suggested to be rigorous in the methodological aspects and an adequate transparency in the transmission of the research results . 

According to \cite{gomez2010replications}, the first replications of the original experiment might be carried out by the same experimenters (\emph{internal replications}), to avoid the influence of variables when sites or experimenters are changed. After carrying out \emph{internal replications}, \emph{external replications} might be performed because they allow to know the range of conditions under which the results are maintained. Furthermore, they demonstrate that the results are independent of the conditions of the original study \cite{brooks1996replication,shull2008role}.

Both \emph{internal} and \emph{external replications} need to incorporate changes to the original experiment for various reasons: i) \emph{to avoid propagating problems} from the original experiment \cite{kitchenham2008role}, ii) \emph{to adapt the replication} to a different environment where the original experiment was carried out \cite{Baldassarre}, and iii) \emph{to generalize the results} of the original experiment (other population, other measures) \cite{shull2008role}.  

Carver \emph{guidelines} \cite{carver2010towards} highlight the need to describe these changes through a \textcolor[rgb]{1,0,0}{specific section} on changes to the original experiment. For each change in the original experiment, Carver recommends recording its description and the situation that caused it. %\textcolor[rgb]{1,0,0}{suprimido threats}
%, and determine its impact on the threats to the validity of the replication. 

%Cambiado
In this work, a template for the definition of replication changes is applied to thirteen replications and empirically evaluated. %me lo he traido de template. 
The use of templates helps to structure the information in a fixed form, reduces ambiguity, facilitates reusability and also serves as a guide to avoid missing relevant information \cite{duran1999requirements}. For some fields in the templates, phrases that are common have been identified and parameterized. These phrases are called lingustic patterns (L–patterns) \cite{toro2000metodologia}.

This template is based on a metamodel to formalize the information and is completed with L-patterns.
%This template is based on a metamodel to formalize the information and is completed with the L-patterns.% to facilitate the writing process \cite{cruz2018} \textcolor[rgb]{1,0,0}{artículo JisBD}. %use of the proposed
The template has a double purpose: on the one hand, it invites the researcher to make the change and its details explicit (reducing \emph{tacit knowledge}) and, on the other hand, it helps the reader to better understand the replication and to follow the trace of how the original experiment evolves in the succession of experiments across a family of them.
%%%%

%sistematiza la concepción de los cambios y la manera de documentarlos será más fácil su comprensión para otros investigadores
Design Science Research (DSR) is adopted as a research methodology. DSR creates and evaluates \emph{artefacts} in order to solve \emph{identified organizational problems} \cite{von2004design}. The phases followed in our research are those defined in the DSR methodology proposed in \cite{Vaishnavi}. Table \ref{tab:fases} shows these phases and their correspondence with the sections of this article.
%\begin{table}
\caption{Phases of the DSR methodology \cite{Vaishnavi} and corresponding sections}
\label{tab:fases}
\begin{minipage}{6cm}

%\begin{tabular}{| l |p{11cm} |}
\begin{tabular}{| p{5.3 cm} | p{5.3 cm} |p{1.2cm} |} \hline

\textbf{Phase of DSR} & \textbf{Description}   &  \textbf{Section}  \\ \hline

\textbf{Awareness of Problem}. The problem that gives rise to a research proposal is identified. & The importance of documenting changes in replications is highlighted. 

&  \ref{sec:intro}  \\ \hline
%\textbf{Introduction}
\textbf{Suggestion}.  A \emph{tentative design} is proposed for the solution to the problem. & The proposed artefact is the metamodel for formalizing information on replications and changes.%The proposed artefact is a template, i.e., a method for reporting changes in experimental replications. 

& \ref{sec:metamodelo} \\ \hline
\textbf{Development}. The \emph{tentative design} is implemented in this phase. & A possible implementation of the metamodel is the template. The template is completed with L-patterns to facilitate the writing.

& \ref{sec:plantilla} \\ \hline
%\textbf{Template for specify changes in replications}
\textbf{Evaluation}. The artefact is evaluated. & The artefact is evaluated by means of its use in thirteen replications belonging to three series of experiments.
& \ref{sec:brief} and \ref{sec:procedure} \\ \hline

\textbf{Conclusion}. Allows you to make adjustments to the artefact and repeat the cycle if necessary. & The  template has  been  improved  by  introducing  new  fields  and  clarifying  their content.

& %\ref{sec:LessonsMI}, \ref{sec:LessonsAL},  \ref{sec:LessonsVV},
\ref{sec:findings} ,  \ref{sec:impact}, \ref{sec:missing} and \ref{sec:summary}   \\ \hline

\end{tabular}
\end{minipage}

\end{table}


% ------------------------------------------------------
% File    : DRS-Fases2.tex
% Content : DRS Fases
% Date    : 1/12/2018
% Version : 1.0
% Authors : M.Cruz
% ------------------------------------------------------

\begin{table}[h]
  %\renewcommand{\arraystretch}{1.45}
  \caption{Phases of the DSR methodology \cite{Vaishnavi} and corresponding sections}
\label{tab:fases}
  \centering
	\scriptsize
  %begin{tabularx}{\textwidth}{cXX}
\begin{tabularx}{0.9\textwidth}{
  >{\hsize=0.88\hsize}X
  >{\hsize=0.02\hsize}X
  >{\hsize=0.88\hsize}X
  >{\hsize=0.02\hsize}X
  >{\hsize=0.20\hsize}X}
  
	\hline\noalign{\smallskip}
		
    %\textbf{Id\#} & 
    %\textbf{Problems detected in the use of template} & \textbf{Proposed solutions} \\
    
    Phase of DSR & &
    Description  & &
    Section \\

	\noalign{\smallskip}\hline\noalign{\smallskip}

    \textbf{Awareness of Problem}. The problem that gives rise to a research proposal is identified & & The importance of documenting changes in replications is highlighted & & 
    \ref{sec:intro}  \\ \\
    
    \textbf{Suggestion}.  A \emph{tentative design} is proposed for the solution to the problem. & & The proposed artefact is the metamodel for formalizing information on replications and changes.& & \ref{sec:metamodelo} \\ \\
    
    \textbf{Development}. The \emph{tentative design} is implemented in this phase. & &
    A possible implementation of the metamodel is the template. The template is completed with L-patterns to facilitate the writing. & &
    \ref{sec:plantilla} \\ \\
    
    \textbf{Evaluation}. The artefact is evaluated. & &
    The artefact is evaluated by means of its use in thirteen replications belonging to three series of experiments. & & 
    \ref{sec:brief}, \ref{sec:procedure} \\ \\

    \textbf{Conclusion}. Allows you to make adjustments to the artefact and repeat the cycle if necessary. & & The  template has  been  improved  by  introducing  new  fields  and  clarifying  their content. & & \ref{sec:findings},  \ref{sec:impact}, \ref{sec:missing}, \ref{sec:summary}   \\ 

	\noalign{\smallskip\smallskip}\hline
	\end{tabularx}  
\end{table}


%Cambiado
This paper is organized as follows: Section \ref{sec:metamodelo} presents the metamodel on which the template is based; Section \ref{sec:plantilla} describes the template for specifying replication changes; In Section \ref{sec:aplica} the template is applied to three series of experiments and analyses the difficulties encountered in defining each change when using the proposed template; Section \ref{sec:trabajos} discusses the related works; and Section \ref{sec:conclusions} presents the concluding remarks and future work.

%Section \ref{sec:DSR} explains the methodology followed;



 

