% ------------------------------------------------------
% File    : Carmen2018-2.tex
% Content : Replication Soil-2018 Secont part 
% Date    : 2/4/2019
% Version : 1.0
% Authors : C.Florido and M.Cruz
% ------------------------------------------------------
\begin{table*}[h]
%\begin{table}[h]
  %\renewcommand{\arraystretch}{1.45}
  %\caption{Soil-2019 replication specification using the template}
\label{tab:plantEng}
  \centering
%	\scriptsize

\begin{tabularx}{\textwidth}{
  >{\hsize=0.25\hsize}X
 %>{\hsize=0.4\hsize}X 2Cols
  >{\hsize=0.8\hsize}X}
  
    \noalign{\smallskip}\hline\noalign{\smallskip}
  
  Field &  Value  \\ 
  \noalign{\smallskip}\hline\noalign{\smallskip}
%%%%%     
  Change \textit{6}   & \textbf{Originally}, the soil aging time (from the time Cu is applied until the plant is sown) is 45 days  \\& \textbf{In replication}, soil aging time is 15 days 
    \\& \textbf{Because of} time constraints and so that Cu is not so much retained\\  
 
    Modified Dimension & 
    \textbf{Protocol}, specifically the guides \\   
    Threat to validity  & The change does not affect validity  \\   
\hline 
%%%%%%
 Change \textit{7} & \textbf{Originally}, there were 6 treatments corresponding to the 3 levels of Cu and with/without \emph{surfactant} (to facilitate Cu extraction). There were 2 soils and 2 types of plants. This represents 24 experimental units (3x2x2x2). For each experimental unit, 3 pots were prepared. In total there are 72 pots (3x2x2x2x3)\\
& \textbf{In replication}, there were 8 treatments corresponding to 4 level of Cu and with/without \emph{surfactant}. There were 1 soil and 1 type of plant. This represents 8 experimental units. For each experimental unit, 4 pots were prepared and placed on a tray. In total there were 32 pots (4x2x4) distributed in 8 trays with 4 pots each. The trays are distributed completely randomly. This is repeated 3 times. The experimental unit was the tray 
    \\& \textbf{Because} by cultivating 4 pots in each tray sufficient biomass can be obtained \\  
    Modified Dimension & 
    \textbf{Protocol}, specifically experimental
    design \\   
    Threat to validity  & The change increases internal validity \\ & because more biomass is obtained for further analysis\\
\hline
%%%%%%
 Change \textit{8}   & \textbf{Originally}, the biomass is collected when the plants have between 2 and 3 true leaves \\& \textbf{In replication}, the plants are rinsed when they have between 2 and 3 real leaves and 4 plants are left by pot. The biomass is collected when the plants reach the fructification stage \\& \textbf{In order to}  avoid competition between plants, let the plants complete their vegetative cycle and thus obtain more biomass \\
    
     Modified Dimension & 
    \textbf{Protocol}, specifically the guides \\   
   Threat to validity  & The change increases internal validity \\ & due to more biomass being obtained for further analysis\\ 
\hline
%%%%%%
 Change \textit{9}   & \textbf{Originally}, the pots are 300 ml tube type \\& \textbf{In replication}, the pots are 500 ml bucket type. \\& \textbf{Because} a greater volume of soil allows for greater root development and greater biomass production \\
    
     Modified Dimension & 
    \textbf{Protocol}, specifically experimental
    material \\   
    Threat to validity  & The change increases internal validity \\ & due to more biomass being obtained for further analysis\\    
%%%%%%    
	\noalign{\smallskip\smallskip}\hline
	\end{tabularx}  
%\end{table}
\end{table*}
