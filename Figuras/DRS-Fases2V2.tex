% ------------------------------------------------------
% File    : DRS-Fases2.tex
% Content : DRS Fases
% Date    : 1/12/2018
% Version : 1.0
% Authors : M.Cruz
% ------------------------------------------------------

\begin{table}[h]
  %\renewcommand{\arraystretch}{1.45}
  \caption{Phases of the DSR methodology \cite{Vaishnavi} and corresponding sections}
\label{tab:fases}
  \centering
	\scriptsize
  %begin{tabularx}{\textwidth}{cXX}
\begin{tabularx}{0.9\textwidth}{
  >{\hsize=0.88\hsize}X
  >{\hsize=0.02\hsize}X
  >{\hsize=0.88\hsize}X
  >{\hsize=0.02\hsize}X
  >{\hsize=0.20\hsize}X}
  
	\hline\noalign{\smallskip}
		
    %\textbf{Id\#} & 
    %\textbf{Problems detected in the use of template} & \textbf{Proposed solutions} \\
    
    Phase of DSR & &
    Description  & &
    Section \\

	\noalign{\smallskip}\hline\noalign{\smallskip}

    \textbf{Awareness of Problem}. The problem that gives rise to a research proposal is identified & & The importance of documenting changes in replications is highlighted & & 
    \ref{sec:intro}  \\ \\
    
    \textbf{Suggestion}.  A \emph{tentative design} is proposed for the solution to the problem. & & The proposed artefact is the metamodel for formalizing information on replications and changes.& & \ref{sec:metamodelo} \\ \\
    
    \textbf{Development}. The \emph{tentative design} is implemented in this phase. & &
    A possible implementation of the metamodel is the template. The template is completed with L-patterns to facilitate the writing. & &
    \ref{sec:plantilla} \\ \\
    
    \textbf{Evaluation}. The artefact is evaluated. & &
    The artefact is evaluated by means of its use in thirteen replications belonging to three series of experiments. & & 
    \ref{sec:brief}, \ref{sec:procedure} \\ \\

    \textbf{Conclusion}. Allows you to make adjustments to the artefact and repeat the cycle if necessary. & & The  template has  been  improved  by  introducing  new  fields  and  clarifying  their content. & & \ref{sec:findings},  \ref{sec:impact}, \ref{sec:missing}, \ref{sec:summary}   \\ 

	\noalign{\smallskip\smallskip}\hline
	\end{tabularx}  
\end{table}
