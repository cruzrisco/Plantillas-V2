
%\begin{table}[htpb]
\begin{table}
\caption{Replicación técnicas de verificación y validación VV-Uds}
%\label{tab:1}       % Give a unique label
%\centering
%\small
\begin{tabular}{| p{3.3cm} | p{9cm} |}
\hline

%\textbf { \textit{Mind}\textbf   { \textit{\#1} }} & Replicación del experimento   \textit{Mind \#0}    \\  \hline
\textbf {\textit{VV-Uds}} & Replicación del experimento \textit{técnicas de verificación y validación (VV-UPM }    \\  \hline

Método empírico &  Experimento   \\  \hline
Tipo replicación &  Externa   \\  \hline
Lugar &  Universidad de Sevilla   \\  \hline
Fecha &     \\  \hline
%Objetivo  & El objetivo de la replicación es  \textless \textit{objetivo} \textgreater  \\  \hline 
Objetivo  &  Entender la efectividad de tres técnicas de verificación y validación en diferentes contextos \\  \hline \hline

Cambio- \textit{1}   & \parbox[t]{9cm} {Originalmente,  \textit{ la duración de las tres sesiones es de cuatro horas cada una; es decir el tiempo es ilimitado } } \parbox[t]{9cm}{en la replicación \textit{ la duración de cada una de las tres sesiones es de 2 horas  } }  debido a  \textit{restricciones de tiempo } \\  \hline
Dimensión modificada & 
Protocolo, en concreto, las guías  \\  \hline 
Amenaza abordada  & El cambio incrementa la validez interna  \\  \hline \hline

Cambio- \textit{2}   & \parbox[t]{9cm} {Originalmente,  \textit{los sujetos ejecutan casos de prueba con la aplicación de la técnica; es decir en cada sesión} } \parbox[t]{9cm}{en la replicación \textit{los sujetos ejecutan casos de prueba para uno de los programas que han probado en una sesión posterior; es decir en la sesión 4} }  debido a  \textit{restricciones de tiempo} \\  \hline
Dimensión modificada & 
Protocolo, en concreto, las guías \\  \hline 
Amenaza abordada  & El cambio incrementa la validez interna  \\  \hline \hline

Cambio- \textit{3}   & \parbox[t]{9cm} {Originalmente,  \textit{los sujetos trabajan de forma individual }} \parbox[t]{9cm}{en la replicación \textit{los sujetos trabajan en parejas} }  debido a  \textit{no hay suficientes ordenadores para todos} \\  \hline
Dimensión modificada & 
Protocolo, en concreto, las guías \\  \hline 
Amenaza abordada  & El cambio incrementa la validez interna  \\  \hline \hline  

Cambio- \textit{4}   & \parbox[t]{9cm} {Originalmente,  \textit{ los sujetos reciben tres sesiones de entrenamiento de cuatro horas para aprender a aplicar las técnicas  } } \parbox[t]{9cm}{en la replicación \textit{ la formación consiste en dos breves tutoriales de dos horas } }  debido a  \textit{los sujetos ya están familiarizados con las técnicas } \\  \hline
Dimensión modificada & 
Operacionalización en concreto, la variable independiente \textit {técnica} \\  \hline 
Amenaza abordada  & El cambio incrementa la validez del constructo  \\  \hline\hline

Cambio- \textit{5}   & \parbox[t]{9cm} {Originalmente,  \textit{el entrenamiento en el uso de las técnicas es antes de que se ejecute el experimento } } \parbox[t]{9cm}{en la replicación \textit{cada tutorial se lleva a cabo antes de la aplicación de la técnica en cada una de las tres sesiones en que se examina cada técnica; es decir, el entrenamiento es intercalado con la operación del experimento } }  debido a  \textit{los sujetos ya están familiarizados con las técnicas} \\  \hline
Dimensión modificada & Operacionalización, en concreto, una variable de contexto \\  \hline 
Amenaza abordada  & El cambio incrementa la validez del constructo  \\  \hline
 
\end{tabular}
%\caption{Comparison of previous reviews}
\label{tab:plantillaUdS}
\end{table}


%Cambio-1 Se aumenta la cantidad de Minfulness, parece que mas que cambiar la variable independiente cambio la forma de aplicarlo, sería instrumentalización ??

%76 words


