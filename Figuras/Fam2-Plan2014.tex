
%\begin{table}[htpb]
\begin{table}
\caption{Replicación E-2014}
%\label{tab:1}       % Give a unique label
%\centering
%\small
\begin{tabular}{| p{3.3cm} | p{9cm} |}
\hline

%\textbf { \textit{Mind}\textbf   { \textit{\#1} }} & Replicación del experimento   \textit{Mind \#0}    \\  \hline
\textbf {\textit{E-2014}} & Replicación del experimento \textit{E-2013 }    \\  \hline

Método empírico &  Experimento   \\  \hline
Tipo &  Interna   \\  \hline
%Objetivo  & El objetivo de la replicación es  \textless \textit{objetivo} \textgreater  \\  \hline 
Objetivo  &  Con la finalidad de obtener más puntos de datos \\  \hline \hline

 Cambio- \textit{1}   & \parbox[t]{9cm} {Originalmente,  \textit{hay dos respondedores} } \parbox[t]{9cm}{en la replicación \textit{ el número de respondedores se reduce a uno} }  debido a  \textit{la indisponibilidad de unos de los respondedores} \\  \hline
Dimensión modificada & 
 Experimentador Monitor  \\  \hline 
Amenaza a la validez abordada  & El cambio incrementa la validez interna  \\  \hline
 \hline
 
Cambio- \textit{2}   & \parbox[t]{9cm} {Originalmente,  \textit{la formación breve (warming up) en actividades relacionadas con requisitos  previa es de 1 semana} } \parbox[t]{9cm}{en la replicación \textit{ la formación breve (warming up) es de 6 semanas} }  con el fin de \textit{explorar un posible efecto de warming up } \\  \hline
Dimensión modificada & 
Operacionalización \\  \hline 
Amenaza a la validez abordada  & El cambio incrementa la validez del constructo  \\  \hline
 \hline


\end{tabular}
%\caption{Comparison of previous reviews}
\label{tab:plantilla}
\end{table}


%Cambio-1 Se aumenta la cantidad de Minfulness, parece que mas que cambiar la variable independiente cambio la forma de aplicarlo, sería instrumentalización ??

%76 words


