% ------------------------------------------------------
% File    : FamTChangesSolV2.tex
% Content : Problems and solutions
% Date    : 1/12/2018
% Version : 1.0
% Authors : M.Cruz
% ------------------------------------------------------

\begin{table}[h]
  %\renewcommand{\arraystretch}{1.45}
  \caption{Types of problems identified and proposed solutions}
  \label{tab:tipos}
  \centering
	\scriptsize
  %begin{tabularx}{\textwidth}{cXX}
\begin{tabularx}{0.9\textwidth}{c
  >{\hsize=0.88\hsize}X
  >{\hsize=0.02\hsize}X
  >{\hsize=1.10\hsize}X}
  
	\hline\noalign{\smallskip}
		
    %\textbf{Id\#} & 
    %\textbf{Problems detected in the use of template} & \textbf{Proposed solutions} \\
    
    Id\# & 
    Problems detected in the use of template &  &
    Proposed solutions \\

	\noalign{\smallskip}\hline\noalign{\smallskip}
	P1 & A \emph{dependent/independent variable} is suppressed or added. The \emph{operationalization dimension} is modified; no \emph{threat to validity} is identified & & \textbf{Modify the template} so that the L-pattern allows you to specify that the change does not affect any \emph{threat to validity} \\ \\

    P2 & The  \emph{reason for the change}  is not specified  & &
    The use of the template avoids this missing information \\ \\
    
    P3 & The  \emph{name of the modified variable} is unknown. The change modifies the  \emph{operationalization dimension}  & &
    The use of the template avoids this missing information \\ \\
	
	P4 & The change consists of modifying a \emph{context variable}. The affected dimension is not identified  & &
	\textbf{Modify the metamodel} considering \emph{context variables} included in the \emph{operationalization dimension}. Therefore, it also implies \textbf{modifying the template} \\ 

	\noalign{\smallskip\smallskip}\hline
	\end{tabularx}  
\end{table}
