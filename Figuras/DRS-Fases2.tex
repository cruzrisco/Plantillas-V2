\begin{table}
\caption{Phases of the DSR methodology \cite{Vaishnavi} and corresponding sections}
\label{tab:fases}
\begin{minipage}{6cm}

%\begin{tabular}{| l |p{11cm} |}
\begin{tabular}{| p{5.3 cm} | p{5.3 cm} |p{1.2cm} |} \hline

\textbf{Phase of DSR} & \textbf{Description}   &  \textbf{Section}  \\ \hline

\textbf{Awareness of Problem}. The problem that gives rise to a research proposal is identified. & The importance of documenting changes in replications is highlighted. 

&  \ref{sec:intro}  \\ \hline
%\textbf{Introduction}
\textbf{Suggestion}.  A \emph{tentative design} is proposed for the solution to the problem. & The proposed artefact is the metamodel for formalizing information on replications and changes.%The proposed artefact is a template, i.e., a method for reporting changes in experimental replications. 

& \ref{sec:metamodelo} \\ \hline
\textbf{Development}. The \emph{tentative design} is implemented in this phase. & A possible implementation of the metamodel is the template. The template is completed with L-patterns to facilitate the writing.

& \ref{sec:plantilla} \\ \hline
%\textbf{Template for specify changes in replications}
\textbf{Evaluation}. The artefact is evaluated. & The artefact is evaluated by means of its use in thirteen replications belonging to three series of experiments.
& \ref{sec:brief} and \ref{sec:procedure} \\ \hline

\textbf{Conclusion}. Allows you to make adjustments to the artefact and repeat the cycle if necessary. & The  template has  been  improved  by  introducing  new  fields  and  clarifying  their content.

& %\ref{sec:LessonsMI}, \ref{sec:LessonsAL},  \ref{sec:LessonsVV},
\ref{sec:findings} ,  \ref{sec:impact}, \ref{sec:missing} and \ref{sec:summary}   \\ \hline

\end{tabular}
\end{minipage}

\end{table}

