
%\begin{table}[htpb]
\begin{table}
\caption{Replicación técnicas de verificación y validación VV-ORT}
%\label{tab:1}       % Give a unique label
%\centering
%\small
\begin{tabular}{| p{3.3cm} | p{9cm} |}
\hline

%\textbf { \textit{Mind}\textbf   { \textit{\#1} }} & Replicación del experimento   \textit{Mind \#0}    \\  \hline
\textbf {\textit{VV-ORT}} & Replicación del experimento \textit{técnicas de verificación y validación (VV-UPM }    \\  \hline

Método empírico &  Experimento   \\  \hline
Tipo replicación &  Externa   \\  \hline
Lugar & Universidad ORT (Uruguay) \\  \hline
Fecha &     \\  \hline
%Objetivo  & El objetivo de la replicación es  \textless \textit{objetivo} \textgreater  \\  \hline 
Objetivo  &  Entender la efectividad de tres técnicas de verificación y validación en diferentes contextos \\  \hline \hline

Cambio- \textit{1}   & \parbox[t]{9cm} {Originalmente,  \textit{ se utilizan las tres técnicas de verificación y validación: lectura de códigos (code reading), partición de equivalencia (equivalence partitioning) y prueba de rama (branch testing) } } \parbox[t]{9cm}{en la replicación \textit{ se omite la técnica de lectura de códigos (code reading)  } }  debido a  \textit{restricciones de tiempo } \\  \hline
Dimensión modificada & 
Operacionalización en concreto, la variable independiente \textit {técnica} \\  \hline 
Amenaza abordada  & El cambio incrementa la validez del constructo  \\  \hline \hline

Cambio- \textit{2}   & \parbox[t]{9cm} {Originalmente,  \textit{ se utilizan tres programas } } \parbox[t]{9cm}{en la replicación \textit{ se omite uno de los programas } }  debido a  \textit{restricciones de tiempo } \\  \hline
Dimensión modificada & 
Protocolo, en concreto, el material experimental \\  \hline 
Amenaza abordada  & El cambio incrementa la validez interna  \\  \hline \hline

Cambio- \textit{3}   & \parbox[t]{9cm} {Originalmente,  \textit{ el experimento se ejecuta en tres sesiones de cuatro horas cada una; es decir el tiempo es ilimitado } } \parbox[t]{9cm}{en la replicación \textit{ el experimento se ejecuta en una única sesión } }  debido a  \textit{  } \\  \hline
Dimensión modificada & 
Protocolo, en concreto, las guías \\  \hline 
Amenaza abordada  & El cambio incrementa la validez interna  \\  \hline \hline

Cambio- \textit{4}   & \parbox[t]{9cm} {Originalmente,  \textit{los sujetos aplican una técnica a un programa en cada una de las tres sesiones} } \parbox[t]{9cm}{en la replicación \textit{los sujetos aplican las dos técnicas a los dos programas en una única sesión} }  debido a  \textit{la duración de la sesión es ilimitada, } \\  \hline
Dimensión modificada & 
Protocolo, en concreto, las guías \\  \hline 
Amenaza abordada  & El cambio incrementa la validez interna  \\  \hline \hline

Cambio- \textit{5}   & \parbox[t]{9cm} {Originalmente,  \textit{los sujetos ejecutan casos de prueba con la aplicación de la técnica} } \parbox[t]{9cm}{en la replicación \textit{no se ejecutan casos de prueba} }  debido a  \textit{los sujetos no pueden acceder a los ordenadores } \\  \hline
Dimensión modificada & 
Protocolo, en concreto, las guías \\  \hline 
Amenaza abordada  & El cambio incrementa la validez interna  \\  \hline
 
\end{tabular}
%\caption{Comparison of previous reviews}
\label{tab:plantillaORT}
\end{table}


%Cambio-1 Se aumenta la cantidad de Minfulness, parece que mas que cambiar la variable independiente cambio la forma de aplicarlo, sería instrumentalización ??

%76 words


