
%\begin{table}[htpb]
\begin{table}
\caption{Replicación Q-2011}
%\label{tab:1}       % Give a unique label
%\centering
%\small
\begin{tabular}{| p{3.3cm} | p{9cm} |}
\hline

%\textbf { \textit{Mind}\textbf   { \textit{\#1} }} & Replicación del experimento   \textit{Mind \#0}    \\  \hline
\textbf {\textit{Q-2011}} & Replicación del experimento \textit{Q-2009 }    \\  \hline

Método empírico &  Cuasi experimento   \\  \hline
Tipo &  Interna   \\  \hline
%Objetivo  & El objetivo de la replicación es  \textless \textit{objetivo} \textgreater  \\  \hline 
Objetivo  &      \\  \hline \hline

Cambio- \textit{1}   & \parbox[t]{9cm} {Originalmente,  \textit{las entrevistas son individuales} } \parbox[t]{9cm}{en la replicación \textit{ entrevistas en grupo entre analistas y clientes} }  debido a  \textit{al costo y esfuerzo que implicaría realizar 16 entrevistas individuales y la influencia que podría ejercer el cansancio del experimentador en la efectividad de los sujetos } \\  \hline
Dimensión modificada & 
 Protocolo guías    \\  \hline 
Amenaza a la validez abordada  & El cambio incrementa la validez  interna   \\  \hline
 \hline
Cambio- \textit{2}   & \parbox[t]{9cm} {Originalmente,  \textit{se considera la experiencia en educción} } \parbox[t]{9cm}{en la replicación \textit{la experiencia se operacionalizará en función de los años de experiencia y la habilidad que el propio sujeto cree tener} }  debido a   \textit{ } \\  \hline
Dimensión modificada & 
Operacionalización  en concreto, las variables independientes \textit { Habilidad en requisitos } y\textit { Habilidad en entrevistas }  \\  \hline 
Amenaza a la validez abordada  &    \\  \hline \hline

Cambio- \textit{3}   & \parbox[t]{9cm} {Originalmente,  \textit{el tiempo de educción, es decir la duración de las entrevistas es de 30min.} } \parbox[t]{9cm}{en la replicación \textit{el tiempo de educción es de 60min.} } debido a que \textit{la entrevista es grupal } \\  \hline
Dimensión modificada & 
 Protocolo en concreto, las guías     \\  \hline 
Amenaza a la validez abordada  & El cambio incrementa la validez  interna  \\  \hline \hline

Cambio- \textit{4}   & \parbox[t]{9cm} {Originalmente,  \textit{el tiempo transcurrido entre la sesión de educción y la consolidación es decir, cuando el sujeto presenta por escrito la información recopilada es de 7 días} } \parbox[t]{9cm}{en la replicación \textit{la consolidación es inmediata a la educción} } con el fin de \textit{evitar perdida de información por olvido } \\  \hline
Dimensión modificada & 
 Protocolo en concreto, las guías     \\  \hline 
Amenaza a la validez abordada  & El cambio incrementa la validez  interna  \\  \hline \hline

Cambio- \textit{5}   & \parbox[t]{9cm} {Originalmente,  \textit{el tiempo de consolidación, es decir, el tiempo disponible para que el analista presente por escrito la información adquirida en las sesiones de educción no se media} } \parbox[t]{9cm}{en la replicación \textit{el tiempo de consolidación es de 120min. } } debido a que \textit{la consolidación es inmediata a la educción} \\  \hline
Dimensión modificada & 
 Protocolo en concreto, las guías     \\  \hline 
Amenaza a la validez abordada  & El cambio incrementa la validez  interna  \\  \hline \hline

Cambio- \textit{6}   & \parbox[t]{9cm} {Originalmente,  \textit{ una persona responde en las entrevistas } } \parbox[t]{9cm}{en la replicación \textit{ se cambia la persona encargada de responder las entrevistas} } debido a  \textit{} \\  \hline
Dimensión modificada & 
 Experimentador Monitor    \\  \hline 
Amenaza a la validez abordada  &    \\  \hline

\end{tabular}
%\caption{Comparison of previous reviews}
\label{tab:Ale-2011}
\end{table}
%Cambio-1 Se aumenta la cantidad de Minfulness, parece que mas que cambiar la variable independiente cambio la forma de aplicarlo, sería instrumentalización ??

%76 words


