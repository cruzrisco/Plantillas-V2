
%\begin{table}[htpb]
\begin{table}
\caption{Replicación técnicas de verificación y validación VV-UPM}
%\label{tab:1}       % Give a unique label
%\centering
%\small
\begin{tabular}{| p{3.3cm} | p{9cm} |}
\hline

%\textbf { \textit{Mind}\textbf   { \textit{\#1} }} & Replicación del experimento   \textit{Mind \#0}    \\  \hline
\textbf {\textit{VV-UPV}} & Replicación del experimento \textit{técnicas de verificación y validación (VV-UPM }    \\  \hline

Método empírico &  Experimento   \\  \hline
Tipo replicación &  Externa   \\  \hline
Lugar & Universidad Politécnica de Valencia  \\  \hline
Fecha &     \\  \hline
%Objetivo  & El objetivo de la replicación es  \textless \textit{objetivo} \textgreater  \\  \hline 
Objetivo  &  Entender la efectividad de tres técnicas de verificación y validación en diferentes contextos \\  \hline \hline
 
Cambio- \textit{1}   & \parbox[t]{9cm} {Originalmente,  \textit{ se utilizan las tres técnicas de verificación y validación: lectura de códigos (code reading), partición de equivalencia (equivalence partitioning) y prueba de rama (branch testing) } } \parbox[t]{9cm}{en la replicación \textit{ se omite la técnica de lectura de códigos (code reading)  } }  debido a  \textit{restricciones de tiempo } \\  \hline
Dimensión modificada & 
Operacionalización en concreto, la variable independiente \textit {técnica} \\  \hline 
Amenaza abordada  & El cambio incrementa la validez del constructo  \\  \hline
Comentario  &  El experimento base son las replicaciones de UPM consideradas como una sola. Solo se hacen los cambios estrictamente requeridos por el nuevo entorno en un intento de mantener los cambios al mínimo. \\  \hline \hline

Cambio- \textit{2}   & \parbox[t]{9cm} {Originalmente,  \textit{ la duración de las tres sesiones es de cuatro horas cada una; es decir el tiempo es ilimitado } } \parbox[t]{9cm}{en la replicación \textit{ la duración de cada una de las tres sesiones es de 2 horas  } }  debido a  \textit{restricciones de tiempo } \\  \hline
Dimensión modificada & 
Protocolo en concreto, guías  \\  \hline 
Amenaza abordada  & El cambio incrementa la validez interna  \\  \hline
 \hline

Cambio- \textit{3}   & \parbox[t]{9cm} {Originalmente,  \textit{ los sujetos reciben tres sesiones de entrenamiento de cuatro horas para aprender a aplicar las técnicas  } } \parbox[t]{9cm}{en la replicación \textit{ la formación consiste en dos breves tutoriales de dos horas } }  debido a  \textit{los sujetos ya están familiarizados con las técnicas } \\  \hline
Dimensión modificada & 
Operacionalización en concreto, la variable independiente \textit {técnica} \\  \hline 
Amenaza abordada  & El cambio incrementa la validez del constructo  \\  \hline
 \hline

Cambio- \textit{4}   & \parbox[t]{9cm} {Originalmente,  \textit{el entrenamiento en el uso de las técnicas es antes de que se ejecute el experimento } } \parbox[t]{9cm}{en la replicación \textit{Cada tutorial se lleva a cabo antes de la aplicación de la técnica, en las dos primeras sesiones; es decir, el entrenamiento es intercalado con la operación del experimento } }  debido a  \textit{los sujetos ya están familiarizados con las técnicas} \\  \hline
Dimensión modificada & Operacionalización, en concreto, una variable de contexto \\  \hline 
Amenaza abordada  & El cambio incrementa la validez del constructo  \\  \hline
 

\end{tabular}
%\caption{Comparison of previous reviews}
\label{tab:plantillaUPV}
\end{table}
% En la 2ª replicación cambia el orden de los MC

%Cambio-1 Se aumenta la cantidad de Minfulness, parece que mas que cambiar la variable independiente cambio la forma de aplicarlo, sería instrumentalización ??

%76 words


