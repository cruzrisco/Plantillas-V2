
%\begin{table}[htpb]
\begin{table}
\caption{Replicación técnicas de verificación y validación VV-UPM1}
%\label{tab:1}       % Give a unique label
%\centering
%\small
\begin{tabular}{| p{3.3cm} | p{9cm} |}
\hline

%\textbf { \textit{Mind}\textbf   { \textit{\#1} }} & Replicación del experimento   \textit{Mind \#0}    \\  \hline
\textbf {\textit{VV-UPM1}} & Replicación del experimento \textit{técnicas de verificación y validación (VV-UPM }    \\  \hline

Método empírico &  Experimento   \\  \hline
Tipo replicación &  Interna   \\  \hline
Lugar &  Universidad Politécnica de Madrid \\  \hline
Fecha &  2002   \\  \hline
%Objetivo  & El objetivo de la replicación es  \textless \textit{objetivo} \textgreater  \\  \hline 
Objetivo  &  Obtener conclusiones que no podían extraerse del experimento base debido a limitaciones de diseño. \\  \hline \hline
% Se analiza si el tipo de fallo, la técnica  de detección utilizada y el programa afectan a la efectividad en la detección de fallos
Cambio- \textit{1}   & \parbox[t]{9cm} {Originalmente,  \textit{ no se analiza la visibilidad del fallo  } } \parbox[t]{9cm}{en la replicación \textit{ se analiza la influencia de la visibilidad del fallo } }  debido a  \textit{querer obtener más conclusiones} \\  \hline
Dimensión modificada & Operacionalización
 \\  \hline 
Amenaza abordada  &   \\  \hline
Comentario  & Se utiliza el paquete de laboratorio elaborado por Kamsties y Lott  \\  \hline \hline

Cambio- \textit{2}   & \parbox[t]{9cm} {Originalmente,  \textit{ el programa es un factor (variable independiente) aunque no se estudia su influencia } } \parbox[t]{9cm}{en la replicación \textit{ se implementan dos versiones de cada programa que será un nuevo factor } }  debido a \textit{que los programas no son muy largos y por tanto se pueden colocar pocos errores ya que se enmascaran unos a otros } \\  \hline
Dimensión modificada & Operacionalización, en concreto, de la variable independiente versión
 \\  \hline 
Amenaza abordada  &   \\  \hline \hline
Comentario  & Dos versiones difieren en cuanto a los fallos pero tienen el mismo número de fallos y los fallos tienen que ser del mismo tipo \\  \hline \hline
 
 Cambio- \textit{3}   & \parbox[t]{9cm} {Originalmente,  \textit{ tres de los tipos de fallos aparecen solo una vez mientras que los otros tres tipos aparecen dos veces } } \parbox[t]{9cm}{en la replicación \textit{ se duplican todos los tipos de fallos } }  debido a \textit{hay dos vesiones de cada programa } \\  \hline
Dimensión modificada & Protocolo, en concreto, el material experimental
 \\  \hline 
Amenaza abordada  & El cambio incrementa la validez Interna \\  \hline \hline

Cambio- \textit{4}   & \parbox[t]{9cm} {Originalmente,  \textit{ los sujetos generan sus casos de prueba para detectar los fallos del código} } \parbox[t]{9cm}{en la replicación \textit{en primer lugar, los sujetos aplican la técnica para generar los casos de prueba y posteriormente, ejecutarán los casos de prueba que se les proporcionan para detectar los fallos del programa} }  con el fin de  \textit{comprobar si la visibilidad de los fallos influye en su detección} \\  \hline
Dimensión modificada & Protocolo, en concreto, el material experimental
 \\  \hline 
Amenaza abordada  & El cambio incrementa la validez Interna \\  \hline \hline

\end{tabular}
%\caption{Comparison of previous reviews}
\label{tab:plantillaUPM2}
\end{table}


%Cambio-1 Se aumenta la cantidad de Minfulness, parece que mas que cambiar la variable independiente cambio la forma de aplicarlo, sería instrumentalización ??

%76 words


