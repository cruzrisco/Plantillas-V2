
\begin{table}
\caption{Findings in each field of the template.}
\label{tab:plantillaProblem2}

\begin{tabular}{| p{3.3cm} | p{9cm} |}

\hline

\textbf {\textless\textit{Acronym}\textbf {\textgreater} \#{\textless\textit{n}\textgreater}}  & Although replication is identified in the template by a code or acronym relating to the baseline experiment and followed by a sequential number, it is advisable to follow the author's notation if different.  \\  \hline

Site \& date  & We realise the need to add the \emph{date} and \emph{site} where the replication is carried out to the template. A new \emph{site} dimension is identified in \cite{Juristo2012}; if replication is carried out at a \emph{site} other than the original experiment, the \emph{site} dimension is changed. Nevertheless, we consider \emph{site} as a template field and not as a new dimension.   \\  \hline

Type of replication & \ding{51}  \\  \hline

%Purpose  &  \parbox[t]{9cm} {\{Confirm results \textbar }  \parbox[t]{9cm} {Extend results \textbar} \hspace{1 mm}  Overcome any limitations  \} \\  \hline 

Purpose  & Except for one of the replications in which its objective is not specified, the rest fits into one of the three possible values to be selected in the template. Purposes of replications of all the proposed types have been found. \\  \hline \hline

Change \#\textless\textit  {i..j}\textgreater  &   In 8 of the 59 changes, the reason for the change is not specified.  \\  \hline

[Modified Dimension] & In 6 of the 59 changes, the affected dimension or element is not identified.
   \\  \hline

Threat to validity   &  In 10 of the 59 changes, the threat to validity is not affected or decreases however the pattern does not allow this specification. \\  \hline
[Comments]  &   The \textit{quasi-experiments}, are included in the \textit{controlled experiments} category following \cite{wohlin:experimentation}, nevertheless, if the author defines replication as a \textit{quasi-experiment}, it should be reflected in the template (e.g. in the comments). \\  \hline


\end{tabular}

\end{table}

