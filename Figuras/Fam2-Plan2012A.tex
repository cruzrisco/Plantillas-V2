
%\begin{table}[htpb]
\begin{table}
\caption{Replicación E-2012A}
%\label{tab:1}       % Give a unique label
%\centering
%\small
\begin{tabular}{| p{3.3cm} | p{9cm} |}
\hline

%\textbf { \textit{Mind}\textbf   { \textit{\#1} }} & Replicación del experimento   \textit{Mind \#0}    \\  \hline
\textbf {\textit{E-2012A}} & Replicación del experimento \textit{Q-2012 }    \\  \hline

Método empírico &  Experimento   \\  \hline
Tipo &  Interna   \\  \hline
%Objetivo  & El objetivo de la replicación es  \textless \textit{objetivo} \textgreater  \\  \hline 
Objetivo  &  Estudiar el efecto del conocimiento que poseen los analistas acerca del dominio del problema, en el proceso de educción.   \\  \hline \hline

Cambio- \textit{1}   & \parbox[t]{9cm} {Originalmente,  \textit{el conocimiento se operacionaliza como familiaridad mediante una valoración subjetiva} } \parbox[t]{9cm}{en la replicación \textit{ el conocimiento se operacionaliza  como variable independiente con dos niveles: problema conocido y desconocido} }  debido a que \textit{en la población experimental (estudiantes de post-grado) se puede saber si conocen o no un determinado dominio del problema } \\  \hline
Dimensión modificada & 
 Operacionalización en concreto, la variable independiente conocimiento  \\  \hline 
Amenaza a la validez abordada  &   \\  \hline
 \hline
 
Cambio- \textit{2}   & \parbox[t]{9cm} {Originalmente,  \textit{es un cuasi experimento en el que el proceso de educción de requisitos se realiza en dos días distintos, para evitar efectos de cansancio por parte del experimentador} } \parbox[t]{9cm}{en la replicación \textit{ se cambia a un diseño de medidas repetidas (within-subjects)} }  debido a \textit{problemas de poder estadístico ya que con este diseño no se necesita un número elevado de sujetos } \\  \hline
Dimensión modificada & 
 Protocolo en concreto, el diseño experimental  \\  \hline 
Amenaza a la validez abordada  & El cambio incrementa la validez interna  \\  \hline
 \hline
 
 Cambio- \textit{3}   & \parbox[t]{9cm} {Originalmente,  \textit{entrevistas en grupo entre analistas y clientes  } } \parbox[t]{9cm}{en la replicación \textit{las entrevistas son individuales} }  debido a  \textit{que se ha cambiado el diseño y hay dos analistas (respondedores) con dos idiomas } \\  \hline
Dimensión modificada & 
 Protocolo en concreto, las guías     \\  \hline 
Amenaza a la validez abordada  & El cambio incrementa la validez  interna   \\  \hline
 \hline
 
  Cambio- \textit{4}   & \parbox[t]{9cm} {Originalmente,  \textit{no hay variables de bloqueo} } \parbox[t]{9cm}{en la replicación \textit{ hay variable de bloqueo por idioma} }  debido a \textit{que es esperable que los sujetos que utilicen su lengua materna sean más efectivos que los sujetos que utilizan una segunda lengua } \\  \hline
Dimensión modificada & 
 Protocolo en concreto, el diseño experimental  \\  \hline 
Amenaza a la validez abordada  & El cambio incrementa la validez interna  \\  \hline
 \hline
 
Cambio- \textit{5}   & \parbox[t]{9cm} {Originalmente,  \textit{no hay variables de bloqueo} } \parbox[t]{9cm}{en la replicación \textit{ hay variable de bloqueo por respondedor} }  debido a \textit{Al bloquear los sujetos por respondedor, se evita que haya interacciones con la variable de bloqueo idioma. Asi los sujetos experimentales realizan la educción en su lengua materna } \\  \hline
Dimensión modificada & 
 Protocolo en concreto, el diseño experimental  \\  \hline 
Amenaza a la validez abordada  & El cambio incrementa la validez interna  \\  \hline
 \hline

 Cambio- \textit{6}   & \parbox[t]{9cm} {Originalmente,  \textit{hay un respondedor} } \parbox[t]{9cm}{en la replicación \textit{ el número de respondedores se ha establecido en dos} }  debido a que \textit{para paliar los efectos de cansancio y aprendizaje del respondedor } \\  \hline
Dimensión modificada & 
 Protocolo en concreto, guías   \\  \hline 
Amenaza a la validez abordada  & El cambio incrementa la validez interna  \\  \hline
 
 
 
%Cambio- \textit{7}   & \parbox[t]{9cm} {Originalmente,  \textit{hay un mismo problema (objeto experimental) para todos los sujetos } \parbox[t]{9cm}{en la replicación \textit{hay dos problemas} }  debido a que \textit{ se hacen grupos debido a las variables de bloqueo } \\  \hline
%Dimensión modificada & 
 %Instrumentalización en concreto, el diseño experimental  \\  \hline 
%Amenaza a la validez abordada  & El cambio incrementa la validez interna  \\  \hline
 %\hline \hline \hline

\end{tabular}
%\caption{Comparison of previous reviews}
\label{tab:plantilla}
\end{table}


%Cambio-1 Se aumenta la cantidad de Minfulness, parece que mas que cambiar la variable independiente cambio la forma de aplicarlo, sería instrumentalización ??

%76 words


