
%\begin{table}[htpb]
\begin{table}
\caption{Replicación E-2012B}
%\label{tab:1}       % Give a unique label
%\centering
%\small
\begin{tabular}{| p{3.3cm} | p{9cm} |}
\hline

%\textbf { \textit{Mind}\textbf   { \textit{\#1} }} & Replicación del experimento   \textit{Mind \#0}    \\  \hline
\textbf {\textit{E-2012B}} & Replicación del experimento \textit{E-2012A }    \\  \hline

Método empírico &  Experimento   \\  \hline
Tipo &  Interna   \\  \hline
%Objetivo  & El objetivo de la replicación es  \textless \textit{objetivo} \textgreater  \\  \hline 
Objetivo  &  Confirmar resultados  \\  \hline \hline

Cambio- \textit{1}   & \parbox[t]{9cm} {Originalmente,  \textit{Se utilizan dos dominios de problemas en el experimento, uno de dominio conocido (DC) y el otro de dominio desconocido (DD)} } \parbox[t]{9cm}{en la replicación \textit{ Se han modificado los dominios de problemas utilizados en el experimento, pero uno sigue siendo dominio conocido (DC) y el otro dominio desconocido (DD)} }  debido a que \textit{ } \\  \hline
Dimensión modificada & 
 Protocolo en concreto, los objetos experimentales  \\  \hline 
Amenaza a la validez abordada  & El cambio incrementa la validez interna  \\  \hline
 \hline
 
Cambio- \textit{2}   & \parbox[t]{9cm} {Originalmente,  \textit{Primero se realiza el problema de dominio conocido y luego el de dominio desconocido  } } \parbox[t]{9cm}{en la replicación \textit{ Se permuta el orden de realización de los problemas, primero el problema de dominio desconocido y luego el conocido} }  debido a que \textit{ } \\  \hline
Dimensión modificada & 
 Protocolo. en concreto, las guías  \\  \hline 
Amenaza a la validez abordada  & El cambio incrementa la validez interna  \\  \hline
 \hline
 
Cambio- \textit{3}   & \parbox[t]{9cm} {Originalmente,  \textit{el experimento se ha llevado a cabo al principio del curso  } } \parbox[t]{9cm}{en la replicación \textit{se ha llevado a cabo después de que los sujetos hayan recibido formación en Ingeniería de Requisitos y específicamente en educción} }  debido a que \textit{ } \\  \hline
Dimensión modificada & 
Operacionalización, en concreto, una variable de contexto  \\  \hline 
Amenaza a la validez abordada  & El cambio incrementa la validez del constructo  \\  \hline
 \hline


\end{tabular}
%\caption{Comparison of previous reviews}
\label{tab:plantilla}
\end{table}


%Cambio-1 Se aumenta la cantidad de Minfulness, parece que mas que cambiar la variable independiente cambio la forma de aplicarlo, sería instrumentalización ??

%76 words


