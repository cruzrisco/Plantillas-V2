
%\begin{table}[htpb]
\begin{table}
\caption{Application of the template to the replication \cite{bernardez-jss-2016}}
%\label{tab:1}       % Give a unique label
%\centering
%\small
\begin{tabular}{| p{3.3cm} | p{9cm} |}
%Cuasi Experimento 
%RQ o GQM & \{Del original ?? Yo no lo pondría\}  \\  \hline
\hline
\textbf { \textit{Mind}\textbf   { \textit{\#2} }} & Replication of experiment   \textit{Mind \#1}    \\  \hline
Empirical method &  Controlled experiment  \\  \hline
Type of replication &  Internal   \\  \hline
%Objetivo  & El objetivo de la replicación es  \textless \textit{objetivo} \textgreater  \\  \hline 
Target  &   Confirm results   \\  \hline \hline

Change \textit{\#1}   & \parbox[t]{9cm} {Originally,  \textit{for 4 weeks Mindfulness was practiced 4 days a week in 10-minute sessions.} } \parbox[t]{9cm}{In replication \textit{the sessions were 12 minutes long and for 6 weeks} }   in order to  \textit{make more evident the benefits of Mindfulness..} \\  \hline
Modified Dimension & 
Operationalization, specifically, the independent variable \textit {Training Workshop }  \\  \hline 
Validity threat addressed  &  The change increases the construct validity   \\  \hline
 \hline
Change \textit{\#2}   & \parbox[t]{9cm} {Originally,  \textit{the assignment of subjects to treatment was not randomized.} } \parbox[t]{9cm}{In replication \textit{it becomes random} }  in order to \textit{remedy threats to the internal validity of quasi-experiments.} \\  \hline
Modified Dimension & Protocol, specifically, experimental design \\  \hline 

Validity threat addressed  &  The change increases the internal validity   \\  \hline \hline

Change \textit{\#3}   & \parbox[t]{9cm} {Originally,  \textit{an public speaking workshop was given to the control group as a placebo.} } \parbox[t]{9cm}{In replication \textit{the oratory workshop took place after the experiment} } in order to  \textit{avoid a possible effect of such a workshop on the measurements of dependent variables.} \\  \hline
Modified Dimension & 
Operationalization, specifically, the independent variable \textit {Training Workshop }  \\  \hline 
Validity threat addressed  &  The change increases the construct validity   \\  \hline \hline

Change \textit{\#N}   & \parbox[t]{9cm} {Originally,  \textit{an public speaking workshop was given to the control group as a placebo.} } \parbox[t]{9cm}{In replication \textit{the oratory workshop took place after the experiment} } in order to  \textit{avoid a possible effect of such a workshop on the measurements of dependent variables.} \\  \hline
Modified Dimension & 
Operationalization, specifically, the independent variable \textit {Training Workshop }  \\  \hline 
Validity threat addressed  &  The change increases the construct validity   \\  \hline
\end{tabular}
%Firstly, the problem of conceptual modeling "Erasmus" is carried out. Secondly, the problem of conceptual modeling "EoD project" is carried out.
%\caption{Comparison of previous reviews}
\label{tab:plantilla-mind}
\end{table}
%Cambio-1 Se aumenta la cantidad de Minfulness, parece que mas que cambiar la variable independiente cambio la forma de aplicarlo, sería instrumentalización ??
% en la 2º replicacion cambia Conceptual modeling exercise order

%76 words


