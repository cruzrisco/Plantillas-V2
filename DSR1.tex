%\section{DSR}
%In the present work, [Wieringa]
Design Science Research (DSR) is adopted as a research methodology. DSR creates and evaluates \emph{artefacts} in order to solve \emph{identified organizational problems} \cite{von2004design}. The phases followed in our research are those defined in the DSR methodology proposed in \cite{Vaishnavi}. Table \ref{tab:fases} shows these phases and their correspondence with the sections of this article.
%\input{Figuras/DRS-Fases.tex}
\begin{table}
\caption{Phases of the DSR methodology \cite{Vaishnavi} and corresponding sections}
\label{tab:fases}
\begin{minipage}{6cm}

%\begin{tabular}{| l |p{11cm} |}
\begin{tabular}{| p{5.3 cm} | p{5.3 cm} |p{1.2cm} |} \hline

\textbf{Phase of DSR} & \textbf{Description}   &  \textbf{Section}  \\ \hline

\textbf{Awareness of Problem}. The problem that gives rise to a research proposal is identified. & The importance of documenting changes in replications is highlighted. 

&  \ref{sec:intro}  \\ \hline
%\textbf{Introduction}
\textbf{Suggestion}.  A \emph{tentative design} is proposed for the solution to the problem. & The proposed artefact is the metamodel for formalizing information on replications and changes.%The proposed artefact is a template, i.e., a method for reporting changes in experimental replications. 

& \ref{sec:metamodelo} \\ \hline
\textbf{Development}. The \emph{tentative design} is implemented in this phase. & A possible implementation of the metamodel is the template. The template is completed with L-patterns to facilitate the writing.

& \ref{sec:plantilla} \\ \hline
%\textbf{Template for specify changes in replications}
\textbf{Evaluation}. The artefact is evaluated. & The artefact is evaluated by means of its use in thirteen replications belonging to three series of experiments.
& \ref{sec:brief} and \ref{sec:procedure} \\ \hline

\textbf{Conclusion}. Allows you to make adjustments to the artefact and repeat the cycle if necessary. & The  template has  been  improved  by  introducing  new  fields  and  clarifying  their content.

& %\ref{sec:LessonsMI}, \ref{sec:LessonsAL},  \ref{sec:LessonsVV},
\ref{sec:findings} ,  \ref{sec:impact}, \ref{sec:missing} and \ref{sec:summary}   \\ \hline

\end{tabular}
\end{minipage}

\end{table}




%DRS requires a deep understanding of the problem to be solved, the consequences to be alleviated and the causes to be avoided [O. D´ıaz, J. P. Contell, and J. R. Venable. Strategic].


%It is a web application that allows users to visualize and share their thoughts in the form of mind maps (Quote Wikipedia). Specifically, DScaffolding (Design Science Scaffolding) is used, a mental map template for DSR projects [J. P. Contell, O. D´ıaz, and J. R. Venable. Dscaffolding]. This template consists of a set of tagged nodes that can be expanded by adding new child nodes. DScaffolding is available as a Chrome plug-in.

%understanding the problem and defining its Causes and Consequence.
%The proposed artefact is a template, i.e., a method for reporting changes in replications. In the present work, the evaluation phase of the proposal is included, which consists of the evaluation of the template by means of its use in three series of experiments.