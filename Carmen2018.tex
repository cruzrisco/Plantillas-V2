% ------------------------------------------------------
% File    : Carmen2018.tex
% Content : Replication Soil-2018 
% Date    : 2/4/2019
% Version : 1.0
% Authors : C.Florido and M.Cruz
% ------------------------------------------------------

\begin{table*}[h]
%\begin{table}[h]
  %\renewcommand{\arraystretch}{1.45}
  \caption{Soil-2018 replication specification using the template}
\label{tab:plantEng}
  \centering
%	\scriptsize 

\begin{tabularx}{\textwidth}{
  >{\hsize=0.25\hsize}X
 %>{\hsize=0.4\hsize}X 2Cols
  >{\hsize=0.8\hsize}X}
  
    \noalign{\smallskip}\hline\noalign{\smallskip}
  
  Field &  Value  \\ 
  \noalign{\smallskip}\hline\noalign{\smallskip}
 
 Replication &   \textbf{\emph{Soil-2018}}   Internal replication based on \textbf{\emph{Soil-2016}}  original experiment   \\
     
 Description \newline of experiment &  To evaluate the effect of a bio-surfactant on the assisted phytoremediation of contaminated soil \\  
 
 Site and Date & The base experiment was carried out in  \textit{ETSIA-University of Seville}  in  \textit{October 2015} and this replication, in  \textit{ETSIA-University of Seville} in \textit{March 2018}    \\
    Purpose  &  Extend results \\  
\hline   
%%%%  
    Change \textit{1}   & \textbf{Originally}, the experiment was carried out in a cultivation chamber. \\& \textbf{In replication}, was carried out in a greenhouse \\& \textbf{In order to} simulate natural conditions \\
    
    Modified Dimension & 
    \textbf{Population}, specifically experimental objects   \\   
    Threat to validity  & The change increases the external validity\\
    & since it allows to generalize the results carrying out the replication in conditions closer to the natural ones \\  \hline
  
%%%%
    Change \textit{2}   & \textbf{Originally}, two types of plants were used: \emph{Hordeum vulgare} L. and \emph{Brassica juncea} L. \\& \textbf{In replication}, only \emph{Brassica juncea} L. was used \\& \textbf{Because} in the original experiment it was demonstrated that only \emph{Brassica juncea} L. was a metal accumulator plant\\  

    Modified Dimension & 
    \textbf{Protocol}, specifically experimental
    material \\   
    Threat to validity  & The change does not affect validity  \\  \hline
%%%%
 
    Change \textit{3}   & \textbf{Originally}, there were two types of soil: Coria (pH=7.8) and Constantina (pH=5.5) \\& \textbf{In replication}, only Constantina soil was used \\& \textbf{Because} it was demonstrated that in the soil of Coria the metal was strongly adsorbed and the phytoextraction did not affect the biomass production
    \\  
     Modified Dimension & 
    \textbf{Protocol}, specifically experimental
    material \\   
    Threat to validity  & The change does not affect validity  \\  \hline

%%%%
    Change \textit{4}   & \textbf{Originally}, Copper (Cu) doses were 0, 500 and 1000 mg $kg^{-1}$ \\& \textbf{In replication}, Cu doses were adjusted to 0, 125, 250 and 500 mg $kg^{-1}$ 
    \\& \textbf{Because of} Cu doses of 1000 mg $kg^{-1}$ was toxic to the plant \\  
 
    Modified Dimension & 
    \textbf{Operationalization}, specifically independent variable dosisCu\\   
    Threat to validity  & The change increases internal validity \\ & because the Cu dose is adjusted to non-toxic levels for the plant\\ 
\hline
%%%%%
    Change \textit{5}   & \textbf{Originally}, Cu was applied as Copper Nitrate \\& \textbf{In replication}, Cu was applied as Copper Sulfate
    \\& \textbf{Because of} is more accessible and the concentrations applied do not affect the plant \\  
 
    Modified Dimension & 
    \textbf{Protocol}, specifically experimental
    material \\   
    Threat to validity  & The change does not affect validity  \\  
 
 %%%%%
   

	\noalign{\smallskip\smallskip}\hline
	\end{tabularx}  
%\end{table}
\end{table*}
