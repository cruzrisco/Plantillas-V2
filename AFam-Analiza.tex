%\section{Analysis of the changes}
This section discusses the difficulties encountered in defining each change using the proposed template. Specifically, it analyses: i) if the changes lead or not to modifications in
12
 the template and ii) main lacks of information found when describing changes.
%
\begin{table}
\caption{Analysis of the changes in the family of Requirements analysis experiments}
\label{tab:changesALE}
%\label{tab:1}       % Give a unique label
%\centering
%\small
%\begin{tabular}{| l | c | c | p{2cm} | c | c |c |}
%\begin{tabular}{| l | c | p{2cm} | p{2cm} | c | p{1.5cm} | c |}
\begin{minipage}{6cm}

%\begin{tabular}{| l | l | l | l | l | l | l |}
\begin{tabular}{| l | l | l |p{6cm} | p{1cm}|}
%\hline\noalign{\smallskip}
\hline
\textbf{Id} & \textbf{Original} & \textbf{Change}  & \textbf{Description}& \textbf{Type}\\
\hline
% Mind2 & Mind2  &~& Replications & 2013--2017 & 105 & Current state of replication\\
Mind\# 2 & Mind\# 1  & Ch-1 & Increase in the number of sessions & T0 \\
~ & ~ & Ch-2 & Random subject assignment & T0 \\
~ & ~ & Ch-3 & The public speaking workshop is held at the end of the workshop. & T0 \\ \hline

Q-2009 & Q-2007 & Ch-1 & Effectiveness is not measured  & T2 \\
~ & ~ & Ch-2 & Retention capacity is not measured  & T2 \\
~ & ~ & Ch-3 & The influence of the development experience is analysed  & T3, T8 \\
~ & ~ & Ch-4 & Interviews are conducted in English  & T6 \\
~ & ~ & Ch-5 & The respondent is changed  & T6, T8 \\ \hline
Q-2011 & Q-2009 & Ch-1 & Group interviews  & T6 \\
~ & ~ & Ch-2 & The influence of skill on requirements and skill in interviews is analyzed  & T1, T3, T8 \\
~ & ~ & Ch-3 & Increases the interview time  & T5, T6 \\
~ & ~ & Ch-4 & Decreases the time to present information  & T6 \\
~ & ~ & Ch-5 & The consolidation time is limited  & T6 \\
~ & ~ & Ch-6 & The respondent is changed  & T6, T8 \\ \hline
Q-2012 & Q-2011 & Ch-1 & The subjects are professionals  & T0 \\
~ & ~ & Ch-2 & The influence of development skill is analyzed  & T3 \\
~ & ~ & Ch-3 & Decreases consolidation time  & T6 \\
~ & ~ & Ch-4 & No training period  & T3, T4 \\ \hline
%T8 Operacionalización pero no conozco la variable
% T3 ya que suprimo la variable
% Relacionando con training es Operacionalización (aunque no se como se llama la variable) y es T0)
E-2012A & Q-2012 & Ch-1 & The influence of knowledge is analysed  & T3 \\
~ & ~ & Ch-2 &  Change in experimental design  & T0 \\
~ & ~ & Ch-3 &  Individual interviews  & T6 \\
~ & ~ & Ch-4 &  The language is a blocking variable.  & T0 \\
~ & ~ & Ch-5 &  The respondent is a blocking variable  & T0 \\
~ & ~ & Ch-6 &  There are two responders  & T6 \\% Cambiar en la plantilla a Operacionalización sin elemento
~ & ~ & Ch-7 &  Groups are formed  & T0 \\ 
~ & ~ & Ch-8 &  Decreases elicitation time  & T6 \\
~ & ~ & Ch-9 &  Increases consolidation time  & T6 \\
~ & ~ & Ch-10 &  The influence of the difficulty of the problem is analysed  & T3 \\ \hline
E-2012B & E-2012A & Ch-1 & The problem domains are modified & T8 \\
~ & ~ & Ch-2 &  The order of the problems is changed  & T7, T8 \\
% Ya no es T6 ya que identifico el cambio de orden
~ & ~ & Ch-3 &  The experiment is conducted at the end of the course  & T5, T6,T8 \\ \hline
E-2013 & E-2012B & Ch-1 & Change in experimental design  & T0 \\
~ & ~ & Ch-2 &  Training is provided  & T4 \\ \hline
E-2014 & E-2013 & Ch-1 & There are one responders  & T6 \\
~ & ~ & Ch-2 &  Increases training time  & T4\\ \hline
E-2015 & E-2013 & Ch-1 & Increases training time  & T4 \\ 
\hline \hline
VV-UPM1 & VV-UPM & Ch-1 & The visibility of the error is analysed  & T3, T4 \\
% Es T3
~ & ~ & Ch-2 &  Two versions of each program  & T3 \\
~ & ~ & Ch-3 &  All types of failures are duplicated  & T0 \\
~ & ~ & Ch-4 &  Test cases are provided & T0 \\
~ & ~ & Ch-5 &  A program is discarded & T0 \\
~ & ~ & Ch-6 &  Each subject applies the three techniques & T0 \\ \hline
VV-UPV & VV-UPM & Ch-1 & A reading technique is omitted & T0 \\
~ & ~ & Ch-2 &  The duration of the sessions is limited & T6 \\
~ & ~ & Ch-3 &  The duration of training is reduced & T0 \\
~ & ~ & Ch-4 &  Change the order of application & T7 \\
% T0, 
~ & ~ & Ch-5 &  Changes in the application of techniques & T6 \\
~ & ~ & Ch-6 &  Changes in the application of test cases & T6 \\ \hline

VV-Uds & VV-UPM & Ch-1 & The duration of the sessions is limited & T6 \\
~ & ~ & Ch-2 &  Change at time of test case execution & T7 \\
% T0,  Protocolo en concreto el orden de ejecución
~ & ~ & Ch-3 &  The subjects work in pairs & T6 \\
~ & ~ & Ch-4 &  The duration of training is reduced & T0 \\
~ & ~ & Ch-5 &  Change the order of application & T5 \\ \hline
% T0, No estaba claro 
VV-ORT & VV-UPM & Ch-1 &  A reading technique is omitted & T0 \\
~ & ~ & Ch-2 &  A program is discarded & T0 \\
%Instrumentalizacion en concreto el material experimental
~ & ~ & Ch-3 &  There is only one session & T6, T8 \\
~ & ~ & Ch-4 &  Changes in the application of techniques to programs. & T6 \\
~ & ~ & Ch-5 &  No test cases are executed & T6 \\


%\noalign{\smallskip}\hline
\hline

\end{tabular}
\end{minipage}
%\caption{Comparison of related reviews}


\end{table}


%76 words


 
\begin{table}
\caption{Analysis of the number of changes in the three families of experiments}
\label{tab:agrupa}
%\label{tab:1}       % Give a unique label
%\centering  
%\small
%\begin{tabular}{| l | c | c | p{2cm} | c | c |c |}
%\begin{tabular}{| l | c | p{2cm} | p{2cm} | c | p{1.5cm} | c |}
\begin{minipage}{6cm}

%\begin{tabular}{| l | l | l | l | l | l | l |}
\begin{tabular}{| p{6.5cm}| c | c | c |c |}
%\hline\noalign{\smallskip}
\hline
 & \textbf{Fam\#1} & \textbf{Fam\#2} & \textbf{Fam\#3} & Sum\\ \hline
Number of replications & 2 & 8 & 4 & 14\\ \hline
Number of changes  & 4 & 33 & 22 & 59\\ \hline
Average number of changes per replication & 2 & 4.1 & 5.5 & 4.2\\ \hline

Num. changes fully specified original template & 4 & 17 & 17 & 38\\ \hline
Num. changes with problem P1 & 0 & 6 & 1 & 7\\ \hline  %Template
Num. changes with problem P2 & 0 & 4 & 1 & 5\\ \hline %Incomp
Num. changes with problem P3 & 0 & 3 & 0 & 3\\ \hline %Incomp
Num. changes with problem P4 & 0 & 0 & 2 & 2\\\hline %Meta (Temp)
Num. changes with problems P1\&P2  & 0 & 2 & 0 & 2\\ \hline %Template + Incomp
Num. changes with problems P1\&P3 & 0 & 0 & 1 & 1\\ \hline %Template + Incomp
Num. changes with problems P2\&P4 & 0 & 1 & 0 & 1\\ \hline %Meta (Temp) + Incomp
 \hline 
Number of changes incompletely defined with the template due to missing information & 0 & 10 & 2 & 12\\ \hline % Con T3+T5  lack of information

Number of changes that lead to modify the original template & 0 & 8 & 2 & 10\\ \hline 

Number of changes that lead to modify the metamodel and therefore also the template & 0 & 1 & 2 & 3\\ \hline 

\end{tabular}
\end{minipage}
%\caption{Comparison of related reviews}


\end{table}


%76 words


%
\begin{table}
\caption{\textcolor[rgb]{1,0,0}{Quitar ?.} Analysis of the changes in the replication of the Mindfulness experiment.}
\label{tab:changesMI}
%\label{tab:1}       % Give a unique label
%\centering  
%\small
%\begin{tabular}{| l | c | c | p{2cm} | c | c |c |}
%\begin{tabular}{| l | c | p{2cm} | p{2cm} | c | p{1.5cm} | c |}
\begin{minipage}{6cm}

%\begin{tabular}{| l | l | l | l | l | l | l |}
\begin{tabular}{| l | l | l |p{6cm} |  l |}
%\hline\noalign{\smallskip}
\hline
\textbf{Id} & \textbf{Original} & \textbf{Change}  & \textbf{Description}& \textbf{Type}\\
\hline
% Mind2 & Mind2  &~& Replications & 2013--2017 & 105 & Current state of replication\\
 Mind\# 2 & Mind\# 1  & Ch-1 & Increase in the number of sessions & \ding{51} \\
~ & ~ & Ch-2 & Random subject assignment & \ding{51} \\
~ & ~ & Ch-3 & The public speaking workshop is held at the end of the workshop. & \ding{51} \\ 

%\noalign{\smallskip}\hline
\hline

\end{tabular}
\end{minipage}
%\caption{Comparison of related reviews}


\end{table}


%76 words


%
\begin{table}
\caption{\textcolor[rgb]{1,0,0}{Quitar ?.} Analysis of the changes in the family of Requirements analysis experiments}
\label{tab:changesALE}
%\label{tab:1}       % Give a unique label
%\centering
%\small
%\begin{tabular}{| l | c | c | p{2cm} | c | c |c |}
%\begin{tabular}{| l | c | p{2cm} | p{2cm} | c | p{1.5cm} | c |}
\begin{minipage}{6cm}

%\begin{tabular}{| l | l | l | l | l | l | l |}
\begin{tabular}{| l | l | l |p{6cm} | p{1cm}|}
%\hline\noalign{\smallskip}
\hline
\textbf{Id} & \textbf{Original} & \textbf{Change}  & \textbf{Description}& \textbf{Type}\\
\hline
% Mind2 & Mind2  &~& Replications & 2013--2017 & 105 & Current state of replication\\

Q-2009 & Q-2007 & Ch-1 & Effectiveness is not measured  & P1 \\
~ & ~ & Ch-2 & Retention capacity is not measured  & P1 \\
~ & ~ & Ch-3 & The influence of the development experience is analysed  & P1, P2 \\ % P1, P2
~ & ~ & Ch-4 & Interviews are conducted in English  &\ding{51} \\
~ & ~ & Ch-5 & The respondent is changed in interviews & P2, P5 \\ \hline
Q-2011 & Q-2009 & Ch-1 & Group interviews  & \ding{51} \\
~ & ~ & Ch-2 & The influence of skill on requirements and skill in interviews is analyzed  & P1, P2 \\ % P1, P2
~ & ~ & Ch-3 & Increases the interview time  & \ding{51} \\
~ & ~ & Ch-4 & Decreases the time to present information  & \ding{51} \\
~ & ~ & Ch-5 & The consolidation time is limited  & \ding{51} \\
~ & ~ & Ch-6 & The respondent is changed in interviews & P2, P5 \\ \hline
Q-2012 & Q-2011 & Ch-1 & The subjects are professionals  & \ding{51} \\
~ & ~ & Ch-2 & The influence of development skill is analyzed  & P1 \\
~ & ~ & Ch-3 & Decreases consolidation time  & \ding{51} \\
~ & ~ & Ch-4 & No training period  & P1 \\ % P1, P3 
\hline
%T8 Operacionalización pero no conozco la variable
% T3 ya que suprimo la variable
% Relacionando con training es Operacionalización (aunque no se como se llama la variable) y es T0)
E-2012A & Q-2012 & Ch-1 & The influence of knowledge is analysed  & P1 \\
~ & ~ & Ch-2 &  Change in experimental design  & \ding{51} \\
~ & ~ & Ch-3 &  Individual interviews  & \ding{51} \\
~ & ~ & Ch-4 &  The language is a blocking variable.  & \ding{51} \\
~ & ~ & Ch-5 &  The respondent is a blocking variable  & \ding{51} \\
~ & ~ & Ch-6 &  There are two responders  & \ding{51} \\% Cambiar en la plantilla a Operacionalización sin elemento
~ & ~ & Ch-7 &  Groups are formed  & \ding{51} \\ 
~ & ~ & Ch-8 &  Decreases elicitation time  & \ding{51} \\
~ & ~ & Ch-9 &  Increases consolidation time  & \ding{51} \\
~ & ~ & Ch-10 &  The influence of the difficulty of the problem is analysed  & P1 \\ \hline
E-2012B & E-2012A & Ch-1 & The problem domains are modified & P2 \\
~ & ~ & Ch-2 &  The order of the problems is changed  & P2 \\
% Ya no es T6 ya que identifico el cambio de orden
~ & ~ & Ch-3 &  \textcolor[rgb]{1,0,0}{The experiment is conducted at the end of the course}   & P2, P4\\  \hline %P2, P4
E-2013 & E-2012B & Ch-1 & Change in experimental design  & \ding{51} \\
~ & ~ & Ch-2 &  Training is provided  & P3 \\ \hline
E-2014 & E-2013 & Ch-1 & There are one responders  & \ding{51} \\
~ & ~ & Ch-2 &  Increases training time  & P3\\ \hline
E-2015 & E-2013 & Ch-1 & Increases training time  & P3 \\ 


%\noalign{\smallskip}\hline
\hline

\end{tabular}
\end{minipage}
%\caption{Comparison of related reviews}


\end{table}


%76 words


%
\begin{table}
\caption{\textcolor[rgb]{1,0,0}{Quitar ?.} Analysis of the changes in the series of experiments on Code evaluation techniques}
\label{tab:changesVV}
\label{tab:1}       % Give a unique label
%\centering
%\small
%\begin{tabular}{| l | c | c | p{2cm} | c | c |c |}
%\begin{tabular}{| l | c | p{2cm} | p{2cm} | c | p{1.5cm} | c |}
\begin{minipage}{6cm}

%\begin{tabular}{| l | l | l | l | l | l | l |}
\begin{tabular}{| l | l | l |p{6cm} |  l |}
%\hline\noalign{\smallskip}
\hline
\textbf{Id} & \textbf{Original} & \textbf{Change}  & \textbf{Description}& \textbf{Type}\\
\hline

VV-UPM1 & VV-UPM & Ch-1 & The visibility of the error is analysed  & P1, P3 \\% Es P1 y P3 
 
~ & ~ & Ch-2 &  Two versions of each program  & P1 \\
~ & ~ & Ch-3 &  All types of failures are duplicated  & \ding{51} \\
~ & ~ & Ch-4 &  Test cases are provided & \ding{51} \\
~ & ~ & Ch-5 &  A program is discarded & \ding{51} \\
~ & ~ & Ch-6 &  Each subject applies the three techniques & \ding{51} \\ \hline
VV-UPV & VV-UPM & Ch-1 & A reading technique is omitted & \ding{51} \\
~ & ~ & Ch-2 &  The duration of the sessions is limited & \ding{51} \\
~ & ~ & Ch-3 &  The duration of training is reduced & \ding{51} \\
~ & ~ & Ch-4 &  \textcolor[rgb]{1,0,0}{The moment of application of the treatment is changed} & P4 \\
% T0, 
~ & ~ & Ch-5 &  Changes in the application of techniques & \ding{51} \\
~ & ~ & Ch-6 &  Changes in the application of test cases & \ding{51} \\ \hline

VV-Uds & VV-UPM & Ch-1 & The duration of the sessions is limited & \ding{51} \\
~ & ~ & Ch-2 &  Change at time of test case execution & \ding{51} \\
% T0,  Protocolo en concreto el orden de ejecución
~ & ~ & Ch-3 &  The subjects work in pairs & \ding{51} \\
~ & ~ & Ch-4 &  The duration of training is reduced & \ding{51} \\
~ & ~ & Ch-5 &  \textcolor[rgb]{1,0,0}{The moment of application of the treatment is changed} & P4 \\ \hline
% T0, No estaba claro 
VV-ORT & VV-UPM & Ch-1 &  A reading technique is omitted & \ding{51} \\
~ & ~ & Ch-2 &  A program is discarded & \ding{51} \\
%Instrumentalizacion en concreto el material experimental
~ & ~ & Ch-3 &  There is only one session & P2 \\
~ & ~ & Ch-4 &  Changes in the application of techniques to programs. & \ding{51} \\
~ & ~ & Ch-5 &  No test cases are executed & \ding{51} \\

%\noalign{\smallskip}\hline
\hline

\end{tabular}
\end{minipage}
%\caption{Comparison of related reviews}


\end{table}


%76 words


%\textcolor[rgb]{1,0,0}{Quitar ?.} Table \ref{tab:changesMI} shows the data related to the changes in the replication of the \textit {Mindfulness} experiment.
%For each change, the acronym of the replication and the original, an identifier of the change and a brief description appear. It also includes the difficulty in describing the change according to the table \ref{tab:tipos}.
%It can be seen that 100\% of the changes fit the template.

%\textcolor[rgb]{1,0,0}{Quitar ?.} Table \ref{tab:changesALE} shows data to changes in the family of \textit{Requirements analysis} experiments. Of the 33 changes included, 6 have been fully defined according to the template. 
%\textcolor[rgb]{1,0,0}{Quitar ?.} Table \ref{tab:changesVV} shows data to changes in the series of experiments on  \textit{Code evaluation techniques}. In this series, 9 of the 22 changes have been defined without difficulty. In the rest, there has been some difficulty in specifying some fields, such as the modified dimension, the threat affected or the reason for the change.

Table \ref{tab:agrupa} totals the changes defined fitting the template and those that have presented difficulties.
By analyzing this table, two types of situations can be identified:
\begin{itemize}
\item \textbf{Due to limitations in the template}. Implies a change in the template (e.g. modify the L-pattern in case the threat is not identified). Some of these situations result in changes in the metamodel.
\item  \textbf{Due to lack of information}. This situation does not imply changes in the template.  It supports the idea that the template is necessary so that applications are not \textbf{under-specified}.
%This situation does not imply changes in the template and supports the idea that using the template the changes are not under specified.

\end{itemize}
\textbf{Main lacks of information found when describing the changes.}  
Due to the difficulty of their specification, some of the fields of the template are optional.  In the three families analysed, among the fields not specified by the authors, \emph{reason for the change introduced} and \emph{validity threats addressed} stand out.   
%two of the fields that present difficulties to be filled in are the \emph{reason for the change} and \emph{the validity threats addressed}.
However, these fields are easily identifiable by the researchers who participate in the replication and fill in the template.

This leads us to reinforce the template by making the fields that were optional mandatory so that this information is not missing.
%Although in some cases it is not specified, it is important to make clear both the purpose of the replication and the reason for each change introduced.  In the template, for the purpose of replication, allows you to choose between three possible prefixed values.
The \textbf{reason for the change} is described by a L-pattern.

In order to facilitate the selection of the \textbf{validity threat addressed}, in \cite{gomez2014understanding} the threats are identified based on changes in the experimental baseline.

\begin{itemize}
\item \emph{Changing the experimenters} does not control any threat. It only ensures that the experimenters do not influence the results \cite{Juristo2012}.
\item \emph{Changing the protocol:} Internal validity is addressed. Results are independent of experimental conditions.
\item \emph{Changing the operationalization:} Construct validity is addressed. %Determine limits for Operationalizations (range of variation of treatments and measures used ) \cite{Juristo2012}.
\item \emph{Changing the population:} External validity is addressed. %Determine limits in the population properties \cite{Juristo2012}.

%\textcolor[rgb]{1,0,0}{Lo de disminuir la validez no lo he visto, no me atrevo a ponerlo.}

% The change  \{ \{increases \textbar decreases\}  \{the construct  \textbar external \textbar internal \textbar conclusion\} validity \textbar not affect \} 
    
\end{itemize}

