
% This is LLNCS.DEM the demonstration file of
% the LaTeX macro package from Springer-Verlag
% for Lecture Notes in Computer Science,
% version 2.4 for LaTeX2e as of 16. April 2010
%
%\documentclass{llncs}
\documentclass[runningheads]{llncs}
\usepackage[utf8]{inputenc} %Codificacion utf-8
%\usepackage[spanish,activeacute]{babel} %Definir idioma español
\usepackage[pdftex]{graphicx}
%\usepackage{paralist}	%Enum
%\usepackage{enumerate}
%\usepackage[acronym]{glossaries}  %Acronym
%\usepackage{jss_cruz}       % other packages and definitions
\usepackage{xspace} % automatic space after macros

% Para los checkmark uso esto 2 package y luego \ding{51}
\usepackage{amssymb}% http://ctan.org/pkg/amssymb
\usepackage{pifont}% http://ctan.org/pkg/pifont
% Para los checkmark uso esto 2 package y luego \ding{51}
%
\usepackage{tabularx} % Para tablas
%\usepackage{makeidx}  % allows for indexgeneration
%
\usepackage{color}
\usepackage[none]{hyphenat} % Nuevo para que no corte las palabras
\usepackage{hyperref}
\begin{document}
\sloppy % Para justificar los párrafos
%
%\frontmatter          % for the preliminaries
%
%\pagestyle{headings}  % switches on printing of running heads
%\addtocmark{Hamiltonian Mechanics} % additional mark in the TOC

%\newacronym{ESE}{ISE}{Ingeniería del Software Empírica}
%newacronym{IS}{IS}{Ingeniería del Software}
\newcommand{\gls}[1]{#1\xspace}

\vspace{1cm}

%\title{Un Enfoque basado en Plantillas para Especificar Cambios en Replicaciones de Experimentos Controlados de estudios empíricos en Ingeniería del software}

\title{An empirical review of a Template to Specify Changes in Controlled Experiment Replications}
\titlerunning{Validation of a Template}
%Controlados

%\author{Ivar Ekeland\inst{1}\orcidID{0000-1111-2222-3333} \and
%Roger Temam\inst{2}\orcidID{1111-2222-3333-4444} \and
%Jeffrey~Dean\orcidID{2222-3333-4444-5555}}
 %José A. Galindo \and  Antonio Ruiz–Cort\'{e}s
%\author{Margarita Cruz\orcidID{0000-0001-8334-6039} \and         Beatriz Bernárdez\orcidID{0000-0002-9390-3772} \and        Amador Durán\orcidID{0000-0003-3630-5511} \and  Antonio Ruiz–Cortés\orcidID{0000-0001-9827-1834}}

\authorrunning{M. Cruz et al.}
\author{Margarita Cruz,         Beatriz Bernárdez,         Amador Durán, José Antonio Parejo \and  Antonio Ruiz–Cortés}
%

\institute{Dept. Lenguajes y Sistemas Informáticos, University of Seville, Seville, Spain.\\
%\institute{Dept. Lenguajes y Sistemas Informáticos, University of Seville, \\  Avda. Reina Mercedes s/n, 41012, Seville, Spain.\\
%\email{I.Ekeland@princeton.edu}}
 %  \email{I.Ekeland@princeton.edu}}
 \email{ \{cruz,beat,amador,japarejo,aruiz\}@us.es}   }      

\maketitle              % typeset the title of the contribution

\begin{abstract} % \gls{ny} sera  \gls{ny}  y  \gls{ESE} sera  \gls{ESE} 
\emph{Context}: The replication of empirical studies in Software Engineering is necessary to consolidate the acquired knowledge.
%To demonstrate the validity of a empirical study in Software Engineering it is necessary to replicate it in different contexts and conditions.
Nevertheless, in order to facilitate successive replications, to improve the understanding of replications and to increase the knowledge generated, the information should be published in a way that allows a \textcolor[rgb]{1,0,0}{deep understandng} of the study.
\emph{Objective}: When designing a replication, the need to make changes to the original study setup arises. In order to represent the relevant information about each change a metamodel and a template were developed. \textcolor[rgb]{1,0,0}{The template is based on the metamodel which formalizes the information involved in the replications}. The goal of the present work is to evaluate both metamodel and template. \emph{Method}: The template has been applied to specify the changes introduced in 14 replications that compound 3 families of experiments.
\emph{Results}: Using the template, we have been able to specify the 59 changes defined in these replications. \textcolor[rgb]{1,0,0}{When analyzing the results, we found that} 13 of them lead us to make modifications in the template and in the way of specifying the detailed information of the change. This implies modifications in some nuances of the metamodel. %With the feedback obtained, the metamodel and the template have been improved by introducing new information and specifying with more details. 
The improved template might allow changes be defined systematically and documented homogeneously.
\emph{Potential audience}: experimenters and researchers in Software Engineering. \emph{Added value}: \textcolor[rgb]{1,0,0}{The proposed template allows the systematic definition of changes, facilitating their specification and avoiding the loss of relevant information}.
%Sistematizar la concepción de los cambios y la manera de documentarlos facilita su comprensión para otros investigadores
 %y que se pueda aplicar en el ámbito general de la Ingeniería del Software experimental y en otras áreas empíricas. \emph{feedback} necesario para su validació
%76 words El patrón se ha validado en la replicación de un experimento llevado a cabo por parte de los autores del presente estudio.
%\keywords{Replication, Template, Linguistic Patterns, Empirical Study, Empirical Software Engineering}
\keywords{Empirical software engineering \and Replication \and Template \and  Linguistic patterns.}
\end{abstract}

% 1. Introduction
\section{Introduction}
\label{sec:intro}
%\gls{SE} acuerdos a nivel de servicio \cite{ruiz2005improving},%\textit{ ``E Experiment''}
%Introduction

The Empirical Software Engineering (ESE) allows the evaluation of new methods, techniques and tools to know the convenience of using them in the development process  \cite{sjoberg2005survey}.
%(Sjøberg \emph{et al.}

Once an artefact has been evaluated for the first time, the study needs to be replicated in different contexts and conditions, not only to consolidate the knowledge acquired, but also to know if its results can be generalized \cite{Baldassarre}.
%Baldassarre \emph{et al.}

Despite the importance of replications, and although the practice of replications has increased in recent years  \cite{da2014replication}, the number of replications in Software Engineering (SE) remains low  \cite{solari2017content}.
%(Da Silva \emph{et al.}
%(Solari \emph{et al.}.

When it comes to identifying the causes of this situation, there are several problems encountered. On the one hand, the lack of single criteria for reporting replications \cite{carver2010towards}. On the other hand, \emph{tacit knowledge}, since the researcher has knowledge that is not explicit in the publication \cite{shull2002replicating}. In addition, there is a lack or incompleteness of the \emph{laboratory packages} that are necessary to be able to make replications \cite{solari2017content}. All this is coupled with the effort and resources needed to carry out an experiment \cite{da2014replication}.

In other areas of knowledge such as psychology, this situation is known as the \emph{replicability crisis} \cite{blanco2017psicologia}. It also includes problems with the use of statistics, large numbers of published false positives and lack of rigour due to individual biases in evidence collection. This situation is also present in ESE \cite{reyes2018statistical} . In order to reduce this crisis, to be rigorous in methodological aspects and an adequate transparency in the transmission of research results is suggested \cite{blanco2017psicologia}. %In order to reduce this crisis, it is suggested to be rigorous in the methodological aspects and an adequate transparency in the transmission of the research results . 

According to \cite{gomez2010replications}, the first replications of the original experiment might be carried out by the same experimenters (\emph{internal replications}), to avoid the influence of variables when sites or experimenters are changed. After carrying out \emph{internal replications}, \emph{external replications} might be performed because they allow to know the range of conditions under which the results are maintained. Furthermore, they demonstrate that the results are independent of the conditions of the original study \cite{brooks1996replication,shull2008role}.

Both \emph{internal} and \emph{external replications} need to incorporate changes to the original experiment for various reasons: i) \emph{to avoid propagating problems} from the original experiment \cite{kitchenham2008role}, ii) \emph{to adapt the replication} to a different environment where the original experiment was carried out \cite{Baldassarre}, and iii) \emph{to generalize the results} of the original experiment (other population, other measures) \cite{shull2008role}.  

Carver \emph{guidelines} \cite{carver2010towards} highlight the need to describe these changes through a \textcolor[rgb]{1,0,0}{specific section} on changes to the original experiment. For each change in the original experiment, Carver recommends recording its description and the situation that caused it. %\textcolor[rgb]{1,0,0}{suprimido threats}
%, and determine its impact on the threats to the validity of the replication. 

%Cambiado
In this work, a template for the definition of replication changes is applied to thirteen replications and empirically evaluated. %me lo he traido de template. 
The use of templates helps to structure the information in a fixed form, reduces ambiguity, facilitates reusability and also serves as a guide to avoid missing relevant information \cite{duran1999requirements}. For some fields in the templates, phrases that are common have been identified and parameterized. These phrases are called lingustic patterns (L–patterns) \cite{toro2000metodologia}.

This template is based on a metamodel to formalize the information and is completed with L-patterns.
%This template is based on a metamodel to formalize the information and is completed with the L-patterns.% to facilitate the writing process \cite{cruz2018} \textcolor[rgb]{1,0,0}{artículo JisBD}. %use of the proposed
The template has a double purpose: on the one hand, it invites the researcher to make the change and its details explicit (reducing \emph{tacit knowledge}) and, on the other hand, it helps the reader to better understand the replication and to follow the trace of how the original experiment evolves in the succession of experiments across a family of them.
%%%%

%sistematiza la concepción de los cambios y la manera de documentarlos será más fácil su comprensión para otros investigadores
Design Science Research (DSR) is adopted as a research methodology. DSR creates and evaluates \emph{artefacts} in order to solve \emph{identified organizational problems} \cite{von2004design}. The phases followed in our research are those defined in the DSR methodology proposed in \cite{Vaishnavi}. Table \ref{tab:fases} shows these phases and their correspondence with the sections of this article.
%\begin{table}
\caption{Phases of the DSR methodology \cite{Vaishnavi} and corresponding sections}
\label{tab:fases}
\begin{minipage}{6cm}

%\begin{tabular}{| l |p{11cm} |}
\begin{tabular}{| p{5.3 cm} | p{5.3 cm} |p{1.2cm} |} \hline

\textbf{Phase of DSR} & \textbf{Description}   &  \textbf{Section}  \\ \hline

\textbf{Awareness of Problem}. The problem that gives rise to a research proposal is identified. & The importance of documenting changes in replications is highlighted. 

&  \ref{sec:intro}  \\ \hline
%\textbf{Introduction}
\textbf{Suggestion}.  A \emph{tentative design} is proposed for the solution to the problem. & The proposed artefact is the metamodel for formalizing information on replications and changes.%The proposed artefact is a template, i.e., a method for reporting changes in experimental replications. 

& \ref{sec:metamodelo} \\ \hline
\textbf{Development}. The \emph{tentative design} is implemented in this phase. & A possible implementation of the metamodel is the template. The template is completed with L-patterns to facilitate the writing.

& \ref{sec:plantilla} \\ \hline
%\textbf{Template for specify changes in replications}
\textbf{Evaluation}. The artefact is evaluated. & The artefact is evaluated by means of its use in thirteen replications belonging to three series of experiments.
& \ref{sec:brief} and \ref{sec:procedure} \\ \hline

\textbf{Conclusion}. Allows you to make adjustments to the artefact and repeat the cycle if necessary. & The  template has  been  improved  by  introducing  new  fields  and  clarifying  their content.

& %\ref{sec:LessonsMI}, \ref{sec:LessonsAL},  \ref{sec:LessonsVV},
\ref{sec:findings} ,  \ref{sec:impact}, \ref{sec:missing} and \ref{sec:summary}   \\ \hline

\end{tabular}
\end{minipage}

\end{table}


% ------------------------------------------------------
% File    : DRS-Fases2.tex
% Content : DRS Fases
% Date    : 1/12/2018
% Version : 1.0
% Authors : M.Cruz
% ------------------------------------------------------

\begin{table}[h]
  %\renewcommand{\arraystretch}{1.45}
  \caption{Phases of the DSR methodology \cite{Vaishnavi} and corresponding sections}
\label{tab:fases}
  \centering
	\scriptsize
  %begin{tabularx}{\textwidth}{cXX}
\begin{tabularx}{0.9\textwidth}{
  >{\hsize=0.88\hsize}X
  >{\hsize=0.02\hsize}X
  >{\hsize=0.88\hsize}X
  >{\hsize=0.02\hsize}X
  >{\hsize=0.20\hsize}X}
  
	\hline\noalign{\smallskip}
		
    %\textbf{Id\#} & 
    %\textbf{Problems detected in the use of template} & \textbf{Proposed solutions} \\
    
    Phase of DSR & &
    Description  & &
    Section \\

	\noalign{\smallskip}\hline\noalign{\smallskip}

    \textbf{Awareness of Problem}. The problem that gives rise to a research proposal is identified & & The importance of documenting changes in replications is highlighted & & 
    \ref{sec:intro}  \\ \\
    
    \textbf{Suggestion}.  A \emph{tentative design} is proposed for the solution to the problem. & & The proposed artefact is the metamodel for formalizing information on replications and changes.& & \ref{sec:metamodelo} \\ \\
    
    \textbf{Development}. The \emph{tentative design} is implemented in this phase. & &
    A possible implementation of the metamodel is the template. The template is completed with L-patterns to facilitate the writing. & &
    \ref{sec:plantilla} \\ \\
    
    \textbf{Evaluation}. The artefact is evaluated. & &
    The artefact is evaluated by means of its use in thirteen replications belonging to three series of experiments. & & 
    \ref{sec:brief}, \ref{sec:procedure} \\ \\

    \textbf{Conclusion}. Allows you to make adjustments to the artefact and repeat the cycle if necessary. & & The  template has  been  improved  by  introducing  new  fields  and  clarifying  their content. & & \ref{sec:findings},  \ref{sec:impact}, \ref{sec:missing}, \ref{sec:summary}   \\ 

	\noalign{\smallskip\smallskip}\hline
	\end{tabularx}  
\end{table}


%Cambiado
This paper is organized as follows: Section \ref{sec:metamodelo} presents the metamodel on which the template is based; Section \ref{sec:plantilla} describes the template for specifying replication changes; In Section \ref{sec:aplica} the template is applied to three series of experiments and analyses the difficulties encountered in defining each change when using the proposed template; Section \ref{sec:trabajos} discusses the related works; and Section \ref{sec:conclusions} presents the concluding remarks and future work.

%Section \ref{sec:DSR} explains the methodology followed;



 



% 2. DSR
%\section{Research Method Used}
%\label{sec:DSR}
%%\section{DSR}
%In the present work, [Wieringa]
Design Science Research (DSR) is adopted as a research methodology. DSR creates and evaluates \emph{artefacts} in order to solve \emph{identified organizational problems} \cite{von2004design}. The phases followed in our research are those defined in the DSR methodology proposed in \cite{Vaishnavi}. Table \ref{tab:fases} shows these phases and their correspondence with the sections of this article.
%\input{Figuras/DRS-Fases.tex}
\begin{table}
\caption{Phases of the DSR methodology \cite{Vaishnavi} and corresponding sections}
\label{tab:fases}
\begin{minipage}{6cm}

%\begin{tabular}{| l |p{11cm} |}
\begin{tabular}{| p{5.3 cm} | p{5.3 cm} |p{1.2cm} |} \hline

\textbf{Phase of DSR} & \textbf{Description}   &  \textbf{Section}  \\ \hline

\textbf{Awareness of Problem}. The problem that gives rise to a research proposal is identified. & The importance of documenting changes in replications is highlighted. 

&  \ref{sec:intro}  \\ \hline
%\textbf{Introduction}
\textbf{Suggestion}.  A \emph{tentative design} is proposed for the solution to the problem. & The proposed artefact is the metamodel for formalizing information on replications and changes.%The proposed artefact is a template, i.e., a method for reporting changes in experimental replications. 

& \ref{sec:metamodelo} \\ \hline
\textbf{Development}. The \emph{tentative design} is implemented in this phase. & A possible implementation of the metamodel is the template. The template is completed with L-patterns to facilitate the writing.

& \ref{sec:plantilla} \\ \hline
%\textbf{Template for specify changes in replications}
\textbf{Evaluation}. The artefact is evaluated. & The artefact is evaluated by means of its use in thirteen replications belonging to three series of experiments.
& \ref{sec:brief} and \ref{sec:procedure} \\ \hline

\textbf{Conclusion}. Allows you to make adjustments to the artefact and repeat the cycle if necessary. & The  template has  been  improved  by  introducing  new  fields  and  clarifying  their content.

& %\ref{sec:LessonsMI}, \ref{sec:LessonsAL},  \ref{sec:LessonsVV},
\ref{sec:findings} ,  \ref{sec:impact}, \ref{sec:missing} and \ref{sec:summary}   \\ \hline

\end{tabular}
\end{minipage}

\end{table}




%DRS requires a deep understanding of the problem to be solved, the consequences to be alleviated and the causes to be avoided [O. D´ıaz, J. P. Contell, and J. R. Venable. Strategic].


%It is a web application that allows users to visualize and share their thoughts in the form of mind maps (Quote Wikipedia). Specifically, DScaffolding (Design Science Scaffolding) is used, a mental map template for DSR projects [J. P. Contell, O. D´ıaz, and J. R. Venable. Dscaffolding]. This template consists of a set of tagged nodes that can be expanded by adding new child nodes. DScaffolding is available as a Chrome plug-in.

%understanding the problem and defining its Causes and Consequence.
%The proposed artefact is a template, i.e., a method for reporting changes in replications. In the present work, the evaluation phase of the proposal is included, which consists of the evaluation of the template by means of its use in three series of experiments.

%% 2. Revisión artículos sobre replicaciones
%\section{Revisión de artículos sobre replicaciones}
%\label{sec:revision}
%\input{JBD-revision}
%%
% 3. Metamodelo
\section{Metamodel about replications and changes}
\label{sec:metamodelo}
% ------------------------------------------------------
% File    : JBD_metamodelo.tex
% Content : Conclusions of the article for 
% Date    : 28/2/2018
% Version : 1.0
% Authors : 
% ------------------------------------------------------
% ------------------------------------------------------
% Last update: ...// 
% - Added ..
% ------------------------------------------------------
%\textit{ ``E Experiment''} in \textit{``Ch.2 Software requirements''}
%\gls{SE} ingeniería del software 

%\subsection{Metamodelo}
%\cite{wiki:Mind} 
This section presents the metamodel, already proposed in \cite{cruz2018}, on which the template is based (Fig.~\ref{fig:F1}) using UML class diagram notation. %\textcolor[rgb]{1,0,0}{}. 

%+-------------------- Fig.1
\begin{figure}

%\includegraphics[width=1\textwidth]
%\includegraphics[width=0.8\textwidth] {Figuras/UML-V3.png}
%\includegraphics[width=\textwidth] {Figuras/UML-V3.png}
%\includegraphics[width=\textwidth] {Figuras/UML-V2-P1.png}
\caption{UML class diagram}
\label{fig:F1} % Unique label
\centering
%\includegraphics[width=0.75\textwidth] {Figuras/UML-V4.png}
\includegraphics[width=0.8\textwidth] {Figuras/UMLV10.png}

\end{figure}
 
In the metamodel, a replication is a type of empirical study based on another empirical study (either the original or an other replication). The inheritance is partial since objects that are instances of the ``original experiment'' are considered in the supertype.
%Su tipo es parcial ya que hay otros tipos de estudios empíricos que no aparecen por no ser de interés en el modelo.
%Among the attributes of the replication class, they appear:
%\begin{itemize} 
%\item \textbf {Acronym}. A code or acronym related to the original experiment is used.
%\item \textbf {\textcolor[rgb]{1,0,0}{Site and Date}}.
%Site and Date of replication.
%\item \textbf {Empirical method}. The three main empirical methods are \textit{controlled experiment}, \textit{survey} y \textit{case study}. The empirical method \textit{quasi-experiment}, in which the assignment of subjects to treatment is not random, are included in the \textit{controlled experiments} category following \cite{wohlin:experimentation}. Although less frequent, we also consider two other empirical methods: \textit{pilot study} and \textit{observational study}.

%\item \textbf {Type of replication}.According to who carries out the replications, these are classified as \textit{internal} (they are performed by the original experimenters) and \textit{external} (they are performed by independent experimenters and therefore their power of confirmation is greater than in the internal ones) \cite{brooks1996replication}.

%There are other taxonomies based on various criteria, for example a possible classification according to the degree to which the original experiment procedure is followed. In this regard, there are authors who speak of \textit{exact} and \textit{conceptual} replications \cite{shull2008role},  \textit{closed} and \textit{differentiated}  \cite{1330459,lindsay1993design,juristo2011role}. Basili et al. \cite{basili1999building} refer to \textit{strict} replications and present a classification into three main groups depending on whether or not the research hypothesis is varied or the theory is expanded. In \cite{gomez2014understanding} a classification of the replications in \textit{literal}, \textit{operational} and \textit{conceptual} is proposed depending on the changes carried out and the purpose of the replication. Due to the lack of agreement on the terminology \cite{gomez2014understanding} and the differentiating nuances, we have chosen to describe the changes by classifying the replications as  \textit{internal} or  \textit{external} without entering into other more confusing categories.

%\item \textbf {Purpose}. A enumerated type that can take three possible values has been defined: i) \textit{Confirm results}, to verify the results obtained in previous experiments, ii) \textit{Extend results} includes, among others, changes in the experimenters to analyze if the results are a consequence of the intervention of these experimenters, changes in the population and measures to see if the results are still met or if the hypothesis is still valid, and iii) \textit{Overcome some limitations of the original experiment}, reflects the changes that occur to avoid the propagation of problems detected in the original experiment (kitchenham \cite{kitchenham2008role}). %\textit{Overcoming some limitations of the original experiment}, reflects the changes that occur as a result of the constraints imposed and to which the experiment needs to be adapted.
%\textcolor[rgb]{1,0,0}{Custom target}. 

%\end{itemize}
To reflect the relationship between the replication class and the class that describes its \textbf {changes}, a composition has been used since the changes are an intrinsic part of the replication. 
%Its attributes are: 
%\begin{itemize} 
%\item \textbf {Description of the change}. It allows to describe what the change is about.

%\item \textbf {Threat to validity}.  
%It's a type enumerated with the values: \textit{increases the  construct, external, internal or conclusion validity}. 
%This attribute serves to indicate how the change introduced mitigates the threat to validity of a given type.

%\item \textbf {Comments}. Additional information about the change. 
%\end{itemize}
Gómez \emph{et al.} \cite{gomez2014understanding} define a set of categories for experimental configuration in replications. Each change belongs to one of these categories, called \emph{dimensions}.
In the class diagram, a hierarchy has been used to classify the configuration elements into the four dimensions. The classification is  \emph{complete} and \emph{without overlap}. The \emph{dimensions} are represented in the metamodel by the subtypes: \textit{Operationalization}, \textit{Population}, \textit{Protocol} and \textit{Stakeholder}.
\textcolor[rgb]{1,0,0}{identified in  \cite{gomez2014understanding,santos2018analyzing}}
%\begin{itemize} 
%\item \textbf {Dimension and elements}. 

%\begin{itemize}
%\item \textit{Operationalization}: includes changes related to \emph{dependent variables} (changes in metrics and measurement procedures) and \emph{ independent variables} (changes in the way treatment is applied). 
%It corresponds to the  EC\_Operationalization subtype with the attributes: variable (name of the variable to be modified) and type that can take the values \emph{dependent} or \emph{independent}.
%\item \textit{Population}: includes changes in the properties of the \emph{experimental subjects} (e.g. different experience, age, etc.) and \emph{objects} (e.g. different programming language, difficulty level, etc.). It corresponds to the  EC\_Population  subtype with the population attribute that can take the values \emph{subjects} or \emph{objects}.
%\item \textit{Protocol}: includes changes in the \emph{experimental design}, in the \emph{experimental material} (when using different instances of the same type of experimental object, e.g. changing the formulation of one problem to another of equal difficulty), in the \emph{guides} (e.g. change the instructions provided to subjects) and in the \emph{measuring instruments} (e.g. change the questionnaire for data collection). It corresponds to the EC\_Protocol subtype with the instrument attribute that can take the values \emph{experimental design}, \emph{experimental material}, \emph{guides} o \emph{measuring instruments}.
%\textcolor[rgb]{1,0,0}{order of application}
%\item \textit{Experimenter}: regarding changes in the roles of experimenters. It corresponds to the EC\_Experimenter subtype with the role attribute that can take the values \emph{designer}, \emph{analyst}, \emph{trainer}, \emph{monitor} o \emph{measurer}.


%Las dimensiones operacionalización, población y experimentadores se corresponden con las identificadas en  \cite{gomez2014understanding}. Debido a su importancia, hemos considerado que el elemento diseño experimental está en una dimensión aparte de la dimensión protocolo. 
%\end{enumerate}
%\end{itemize}
%\end{itemize}



% 4. Plantilla
\section{Template for specify changes in replications}
\label{sec:plantilla}
% ------------------------------------------------------
% File    : JBD-plantillas.tex
% Content : Platilla propuesta
% Date    : Enero/2018
% Version : 1.0
% Authors : 
% ------------------------------------------------------
% ------------------------------------------------------
% Last update: ...// 
% - Added ..
% ------------------------------------------------------
%\gls{SE} acuerdos a nivel de servicio \cite{ruiz2005improving},%\textit{ ``E Experiment''}
This section summarises the template to define the specification of replication changes already proposed in \cite{cruz2018} \textcolor[rgb]{1,0,0}{artículo JisBD}. 

In the notation used to describe the L-patterns, words or phrases between \textless \hspace{0.5 mm} and \textgreater \hspace{1 mm} must be properly replaced, words or phrases between \{ and \} and separated by \textbar \hspace{1 mm} represents options; only one option must be chosen and words between [ and ]  are optional, that is, they may or may not appear when the template is instantiated. Filling the blanks in pre–written sentences, i.e. L–patterns, is easier and faster than writing a whole paragraph. 
%% ------------------------------------------------------
% File    : Jbd-Tplantilla.tex
% Content : Plantilla
% Date    : 12/Feb/2018
% Version : 1.0
% Authors : 
% ------------------------------------------------------
% ------------------------------------------------------
% Last update: ...// 
% - Added ..
% ------------------------------------------------------


%\begin{table}[htpb]
\begin{table}
\caption{Template to specify changes in a replication}
\label{tab:plantilla-V1}

%\label{tab:1}       % Give a unique label
%\centering
%\small
\begin{tabular}{| p{3.3cm} | p{9cm} |}

%Cuasi Experimento 
%RQ o GQM & \{Del original ?? Yo no lo pondría\}  \\  \hline
\hline


\textbf {\textless\textit{Acronym}\textbf {\textgreater} \#{\textless\textit{n}\textgreater}} & Replication of experiment \textless \textit{Acronym of experiment} \textgreater \#{\textless\textit{m}\textgreater} \\  \hline

%Environment  &   {\textless\textit{Date}\textgreater}  \\   &  \textless  \textit{Place}\textgreater \\    \hline

%Empirical method & \{Controlled experiment  \hspace{1 mm} \textbar \hspace{1 mm}  Survey \textbar \hspace{1 mm}  Case study \textbar \hspace{1 mm} \\   &  Observational study \textbar \hspace{1 mm}  Pilot study \}  \\  \hline
Type of replication & \{Internal \textbar \\ & External\}  \\  \hline

%Purpose  &  \parbox[t]{9cm} {\{Confirm results \textbar }  \parbox[t]{9cm} {Extend results \textbar} \hspace{1 mm}  Overcome any limitations  \} \\  \hline 

Purpose  &  \{Confirm results \textbar \\ & Extend results \textbar \\ & Overcome any limitations\} \\  \hline \hline


%Change \#\textless\textit  {i..j}\textgreater  & \parbox[t]{9cm} {\textbf{Originally}, \textless\textit{description of the situation in original experiment}\textgreater} \parbox[t]{9cm}{In replication \textless\textit{description of the situation in replication}\textgreater}  \{in order to  \textbar \hspace{1 mm} because of  \}  \textless\textit{cause of change}\textgreater  \\  \hline

Change \#\textless\textit  {i..j}\textgreater  &  \textbf{Originally}, \textless\textit{description of situation in original experiment}\textgreater \\& \textbf{In replication} \textit{\textless description of the situation in replication}\textgreater \\& \{\textbf{in order to  \textbar because of}\}  \textless\textit{cause of change}\textgreater  \\  \hline
%& \{in order to  \textbar because of  \}  \textless\textit{cause of change}\textgreater  \\  \hline
%%%%
[Modified Dimension] & 
 \{ \textbf{Operationalization} [(, specifically, the  \{dependent \textbar  independent\} variable \textless \textit{variable name}\textgreater)]  \textbar  \\ &
 \textbf{Population}[(, in particular, experimental \{subjects\textbar objects\})]\textbar \\ & 
\textbf{Protocol} [(, in particular, \{experimental design \textbar experimental material \textbar guides \textbar measuring instruments\})] \textbar \\ & \textbf{Experimenters} [(, in particular, \{ designer \textbar   analyst \textbar  trainer \textbar  monitor \textbar  measurer\})]\}    \\  \hline
%%%%
%[Modified Dimension] & \parbox[t]{9cm}  { \{Operationalization  [(, specifically, the  \{dependent \textbar  independent\} variable \textless \textit{variable name}\textgreater)]  \textbar }  \parbox[t]{9cm} {Population[(, in particular, experimental \{subjects \textbar objects\} )]\textbar} \parbox[t]{9cm} {Protocol [(, in particular, \{experimental design \textbar experimental material \textbar guides \textbar measuring instruments  \} )] \textbar } \parbox[t]{9cm} {Experimenters [(, in particular, \{ designer \textbar   analyst \textbar  trainer \textbar  monitor \textbar  measurer\})]  \} }   \\  \hline 
%addressed
Threat to validity   &  The change increases  \{the construct  \textbar external \textbar internal \textbar conclusion\} validity \\  \hline
[Comments]  &   {\textless\textit{Comments}\textgreater}  \\  \hline
% se utilizan distintas instancias del mismo tipo de objeto experimental

\end{tabular}

%\caption{Template for specifying changes in a replication}
%\label{tab:plantilla-V1}
\end{table}


%76 words


% ------------------------------------------------------
% File    : Fam-TplantillaV2.tex
% Content : Problems and solutions
% Date    : 1/12/2018
% Version : 1.0
% Authors : M.Cruz
% ------------------------------------------------------

\begin{table}[h]
  %\renewcommand{\arraystretch}{1.45}
  \caption{Template to specify changes in a replication}
\label{tab:plantilla-V1}
  \centering
	\scriptsize
  %begin{tabularx}{\textwidth}{cXX}
\begin{tabularx}{\textwidth}{
  >{\hsize=0.55\hsize}X
  %>{\hsize=0.01\hsize}X
  >{\hsize=1.45\hsize}X}
  
    \noalign{\smallskip}\hline\noalign{\smallskip}
  
  Field &  Value  \\ 
  \noalign{\smallskip}\hline\noalign{\smallskip}
  
  \textbf {\textless\textit{Acronym}\textbf {\textgreater} \#{\textless\textit{n}\textgreater}} & 
  Replication of experiment \textless \textit{Acronym of experiment} \textgreater \#{\textless\textit{m}\textgreater} \\ 
  
  Type of replication &  \{Internal \textbar External\}  \\  

    Purpose  &  \{Confirm results \textbar Extend results \textbar Overcome any limitations\} \\  \hline
    
    Change \#\textless\textit  {i..j}\textgreater  & 
    \textbf{Originally}, \textless\textit{description of situation in original experiment}\textgreater \\& 
    \textbf{In replication} \textit{\textless description of the situation in replication}\textgreater \\&  \{\textbf{in order to  \textbar because of}\}  \textless\textit{cause of change}\textgreater  \\ 
  
  [Modified Dimension] & 
 \{ \textbf{Operationalization} [(, specifically, the  \{dependent \textbar  independent\} variable \textless \textit{variable name}\textgreater)]  \textbar  \\ &
 \textbf{Population}[(, in particular, experimental \{subjects\textbar objects\})]\textbar \\ & 
\textbf{Protocol} [(, in particular, \{experimental design \textbar experimental material \textbar guides \textbar measuring instruments\})] \textbar \\ & \textbf{Experimenters} [(, in particular, \{ designer \textbar   analyst \textbar  trainer \textbar  monitor \textbar  measurer\})]\}    \\

Threat to validity   &  The change increases  \{the construct  \textbar external \textbar internal \textbar conclusion\} validity \\  

[Comments]  &   {\textless\textit{Comments}\textgreater}  \\


	\noalign{\smallskip\smallskip}\hline
	\end{tabularx}  
\end{table}


 The template has two parts, a general part for replication and a specific part for each of the changes included in the replication. Table \ref{tab:plantilla-V1} shows the proposed template. The meaning of the fields is as follows:
\begin{itemize}
\item \textbf {Acronym for Replication}: In order to obtain a quick identification of the different experiments of a family, it is useful that the description of the set of replications of the same experiment begins with a code or acronym relative to the reference experiment and followed by a sequential number. 
\item \textbf {Acronym of experiment}: The acronym for experiment and replication is the same. The experiment to which the replication refers can be either an original experiment or a previous replication. When the reference experiment is the original experiment, the sequential number following it should be 1, so that it can be easily identified when the reference experiment is already a replication.


%\item \textbf {Environment}: Site and date of replication.
%\item \textbf {Empirical method}: One of the possible values is selected: \textit{controlled experiment}, \textit{survey}, \textit{case study}, \textit{pilot study} and \textit{observational study}.

\item \textbf {Type of replication}: According to who carries out the replications, these are classified as \textit{internal} (they are performed by the original experimenters) and \textit{external} (they are performed by independent experimenters and therefore their power of confirmation is greater than in the internal ones) \cite{brooks1996replication}.

There are other taxonomies based on different criteria, for example a possible classification according to the degree to which the original experiment procedure is followed. In this regard, there are authors who speak of \textit{exact} and \textit{conceptual} replications \cite{shull2008role},  \textit{closed} and \textit{differentiated}  \cite{1330459,lindsay1993design,juristo2011role}. Basili et al. \cite{basili1999building} refer to \textit{strict} replications and present a classification into three main groups depending on whether or not the research hypothesis is varied or the theory is expanded. In \cite{gomez2014understanding} a classification of the replications in \textit{literal}, \textit{operational} and \textit{conceptual} is proposed depending on the changes carried out and the purpose of the replication. Due to the lack of agreement on the terminology \cite{gomez2014understanding} and the differentiating nuances, we have chosen to describe the changes by classifying the replications as  \textit{internal} or  \textit{external} without entering into other more confusing categories. 

Understanding the changes, i.e. if the changes are clearly described, it is not necessary to label the replication in any of these categories. 

\item \textbf {Purpose}: Allows you to select between three possible values: i) \textit{Confirm results}, to verify the results obtained in previous experiments; ii) \textit{Extend results} includes, among others, changes in the experimenters to analyze if the results are a consequence of the intervention of these experimenters, changes in the population and measures to see if the results are still met or if the hypothesis is still valid; and iii) \textit{Overcome some limitations of the original experiment} reflects the changes that occur to avoid the propagation of problems detected in the original experiment (kitchenham \cite{kitchenham2008role}).  %or \textit{Custom target}. 
 \end{itemize}
 
Each replication incorporates one or more \textbf{changes}:
\begin{itemize}
\item \textbf {Description of the change}: The changes are numbered sequentially and their description consists of a L-pattern that must be completed.  In order to correctly specify each change, the \textit{original situation} in the baseline experiment is shown, \textit{new situation} in replication is defined and finally the \textit{cause or consequence} are recorded.  In this last part of the pattern, you can choose between ``\textit{in order to}'' or ``\textit{because of}'' to complete the sentence.
\textcolor[rgb]{1,0,0}{Tiene que cuadrar con lo dicho en la intro (evitar propagar, adaptar y generalizar)}
\item \textbf {Modified dimension and elements}: The L-pattern is completed by choosing one of the identified dimensions \cite{gomez2014understanding,santos2018analyzing}. Within the selected dimension, you can specify the modified element by choosing one of the options presented. 
%\textcolor[rgb]{1,0,0}{Suprimir si se deja el metamodelo}. 
%%%

The proposed dimensions and elements are:
\begin{itemize}
\item \textit{Operationalization}: includes changes related to \emph{dependent variables} (changes in metrics and measurement procedures) and \emph{ independent variables} (changes in the way treatment is applied). 

\item \textit{Population}: includes changes in the properties of the \emph{experimental subjects} (e.g. different experience, age, etc.) and \emph{objects} (e.g. different programming language, difficulty level, etc.). \textcolor[rgb]{1,0,0}{In \cite{santos2018analyzing} only includes changes in experimental subjects}
\item \textit{Protocol}: Includes changes in the \emph{experimental design}, in the \emph{experimental material} (when using different instances of the same type of experimental object, e.g. changing the formulation of one problem to another of equal difficulty), in the \emph{guides} (e.g. change the instructions provided to subjects) and in the \emph{measuring instruments} (e.g. change the questionnaire for data collection). The experimental protocol is the set up of these elements to observe the effects of treatments \cite{Juristo2012}.
%\textcolor[rgb]{1,0,0}{order of application}
\item \textit{Stakeholder}: regarding changes in the roles of experimenters. It can take the values \emph{designer}, \emph{analyst}, \emph{trainer}, \emph{monitor} o \emph{measurer}.


%Las dimensiones operacionalización, población y experimentadores se corresponden con las identificadas en  \cite{gomez2014understanding}. Debido a su importancia, hemos considerado que el elemento diseño experimental está en una dimensión aparte de la dimensión protocolo. 
%\end{enumerate}
\end{itemize}
%%%
\textcolor[rgb]{1,0,0}{Quitar?}.  Both the dimension and the element concerned are optional. 
In other words, for simplicity's sake, it is possible to describe a change without having to identify the dimension and the element affected by the change.
%One of the threats  is chosen:
\item \textbf {Threat to validity addressed}: Allows you to select between \textit{the construct}, \textit{external}, \textit{internal} or \textit{conclusion} validity \cite{wohlin:experimentation}.
\item \textbf {Comments}: Additional information about the change that could be of interest to the reader.

 \end{itemize}
 






 



% 5. Aplica 
\section{Application of the template to changes in three families of experiments}
\label{sec:aplica}
% 
% Section 4
%
% Application of the template to changes in three families of experiments
%
In this section the template is applied to the definition of the changes made in the replications carried out in three series of experiments in the SE area that deal with: i) Mindfulness \cite{bernardez-jss-2016}; ii) Requirements analysis \cite{aranda2016estudio}, and; iii) Code evaluation techniques \cite{juristo2012comparing,juristo2003functional,juristo2013process}.

%
% Section 4.1
% Brief description of the families of experiments
%
\subsection{Brief description of the families of experiments}
\label{sec:brief}
The first series of replications to apply the template (fam\#1) has been chosen for being familiar with it, as it has been carried out by some of the authors of this work. The other two families (fam\#2, fam\#3) have been chosen for knowing the domain of the problem as they are close to our area of knowledge.
\subsubsection{A family of experiments on Mindfulness.}

The family (fam\#1) described in \cite{bernardez2014controlled,bernardez-jss-2016}, consists of one experiment and two internal replications carried out at the University of Seville by some of the authors of the present article. The study deals with the effect of the practice of \emph{Mindfulness} on the performance of students when developing conceptual models. \emph{Mindfulness} is a meditation technique aimed to increase clearness of mind and awareness.
%
%\begin{table}[htpb]
\begin{table}
\caption{Application of the template to the replication \cite{bernardez-jss-2016}}
%\label{tab:1}       % Give a unique label
%\centering
%\small
\begin{tabular}{| p{3.3cm} | p{9cm} |}
%Cuasi Experimento 
%RQ o GQM & \{Del original ?? Yo no lo pondría\}  \\  \hline
\hline
\textbf { \textit{Mind}\textbf   { \textit{\#2} }} & Replication of experiment   \textit{Mind \#1}    \\  \hline
Empirical method &  Controlled experiment  \\  \hline
Type of replication &  Internal   \\  \hline
%Objetivo  & El objetivo de la replicación es  \textless \textit{objetivo} \textgreater  \\  \hline 
Target  &   Confirm results   \\  \hline \hline

Change \textit{\#1}   & \parbox[t]{9cm} {Originally,  \textit{for 4 weeks Mindfulness was practiced 4 days a week in 10-minute sessions.} } \parbox[t]{9cm}{In replication \textit{the sessions were 12 minutes long and for 6 weeks} }   in order to  \textit{make more evident the benefits of Mindfulness..} \\  \hline
Modified Dimension & 
Operationalization, specifically, the independent variable \textit {Training Workshop }  \\  \hline 
Validity threat addressed  &  The change increases the construct validity   \\  \hline
 \hline
Change \textit{\#2}   & \parbox[t]{9cm} {Originally,  \textit{the assignment of subjects to treatment was not randomized.} } \parbox[t]{9cm}{In replication \textit{it becomes random} }  in order to \textit{remedy threats to the internal validity of quasi-experiments.} \\  \hline
Modified Dimension & Protocol, specifically, experimental design \\  \hline 

Validity threat addressed  &  The change increases the internal validity   \\  \hline \hline

Change \textit{\#3}   & \parbox[t]{9cm} {Originally,  \textit{an public speaking workshop was given to the control group as a placebo.} } \parbox[t]{9cm}{In replication \textit{the oratory workshop took place after the experiment} } in order to  \textit{avoid a possible effect of such a workshop on the measurements of dependent variables.} \\  \hline
Modified Dimension & 
Operationalization, specifically, the independent variable \textit {Training Workshop }  \\  \hline 
Validity threat addressed  &  The change increases the construct validity   \\  \hline \hline

Change \textit{\#N}   & \parbox[t]{9cm} {Originally,  \textit{an public speaking workshop was given to the control group as a placebo.} } \parbox[t]{9cm}{In replication \textit{the oratory workshop took place after the experiment} } in order to  \textit{avoid a possible effect of such a workshop on the measurements of dependent variables.} \\  \hline
Modified Dimension & 
Operationalization, specifically, the independent variable \textit {Training Workshop }  \\  \hline 
Validity threat addressed  &  The change increases the construct validity   \\  \hline
\end{tabular}
%Firstly, the problem of conceptual modeling "Erasmus" is carried out. Secondly, the problem of conceptual modeling "EoD project" is carried out.
%\caption{Comparison of previous reviews}
\label{tab:plantilla-mind}
\end{table}
%Cambio-1 Se aumenta la cantidad de Minfulness, parece que mas que cambiar la variable independiente cambio la forma de aplicarlo, sería instrumentalización ??
% en la 2º replicacion cambia Conceptual modeling exercise order

%76 words



%In Appendix A, 
%The table \ref{tab:plantilla-mind}

This is the first instantiation of the template. When applying the template, the second replication has been defined based on the first replication. With its current structure we are able to make explicit the changes in the two replications cited. However, the template needs to be validated in replications of experiments carried out by other researchers.

\subsubsection{A family of experiments on Requirements Analysis.}

This is an empirical study about  the influence of analyst domain knowledge and experience on the effectiveness of \emph{requirements analysis} process. The family (fam\#2) consists of nine experiments involving postgraduate students at the Polytechnic University of Madrid, as well as professionals from different countries and institutions. The whole family is described in \cite{aranda2016estudio}.

According to the author, the acronyms for the experiments and replicactions are: Q-2007, Q-2009, Q-2011, Q-2012, E-2012A, E-2012B, E-2013, E-2014 and E-2015. Q-2007 is the original experiment. 
Each experiment is a replication of the previous one. Therefore, the template has been applied defining the changes in each replication with respect to the previous replication. %except the E-2015 experiment which is a replication of the E-2013. 
The first step in filling out the template is to identify on which experiment or previous replication the replication is defined.

\subsubsection{A family of experiments on Code Evaluation Techniques.}

It consists of one experiment and its four replications. The original experiment is conducted with students at the Polytechnic University of Madrid to evaluate the effectiveness of three verification and validation techniques. This experiment has been replicated in four different sites: the Polytechnic University of Madrid, the Polytechnic University of Valencia, the University of Seville and the ORT University of Uruguay. The complete family of experiments (fam\#3) is described in \cite{juristo2012comparing,juristo2003functional,juristo2013process}.
%
%\begin{table}[htpb]
\begin{table}
\caption{Instantiation of the proposed template in VV-ORT}
%\label{tab:1}       % Give a unique label
%\centering
%\small
\scalebox{0.95}{
\begin{tabular}{| p{3.3cm} | p{9cm} |}
\hline

%\textbf { \textit{Mind}\textbf   { \textit{\#1} }} & Replicación del experimento   \textit{Mind \#0}    \\  \hline
\textbf {\textit{VV-ORT}} & Replication of experiment \textit{VV-UPM}    \\  \hline

%Método empírico &  Experimento   \\ \hline
Type of replication &  External \\  \hline
Site & Universidad ORT (Uruguay) \\  \hline
%Fecha &     \\  \hline
 
Purpose  &  Extend results \\  \hline \hline

%%%%
Change \textit{1}   & \textbf{Originally}, the three techniques of verification and validation are used: code reading, equivalence partitioning and branch testing \\& \textbf{In replication} the code reading technique is omitted \\& \textbf{because of} time constraints \\ & \\  \hline
%%%%

Modified Dimension & 
\textbf{Operationalization}, specifically the independent variable  \textit {technique} \\  \hline 
Threat to validity & The change decreases the construct validity  \\ & \\  \hline \hline

%%%%
Change \textit{2}   & \textbf{Originally}, three program codes are used \\& \textbf{In replication} one of the programs is  discarded \\& \textbf{because of} time constraints \\ & \\  \hline
%%%%

Modified Dimension & 
\textbf{Protocol}, specifically experimental
material \\  \hline 
Threat to validity  & The change increases the internal validity \\ & \\  \hline \hline


%%%%
Change \textit{3}   & \textbf{Originally}, the experiment is carried out in three sessions each of four hours \\& \textbf{In replication} the experiment is executed in a single session  \\ & \\  \hline
%%%%

Modified Dimension & 
\textbf{Protocol}, specifically the guides \\  \hline 
Threat to validity  & The change increases the internal validity \\ & \\  \hline \hline

%%%%
Change \textit{4}   & \textbf{Originally}, subjects apply a different technique to  evaluate a program in each of the three sessions \\& \textbf{In replication} the subjects apply the two techniques to the two programs in a single session \\&    \\  \hline
%%%%

Modified Dimension & 
\textbf{Protocol}, specifically experimental design \\  \hline 
Threat to validity  & The change increases the internal validity \\ & \\  \hline \hline

%%%%
Change \textit{5}   & \textbf{Originally}, subjects execute test cases with the application of the technique \\& \textbf{In replication} no test cases are executed \\& \textbf{because of} computers are not accessible \\ & \\  \hline
%%%%
Modified Dimension & 
\textbf{Protocol}, specifically the measuring instruments \\  \hline 
Threat to validity  & The change increases the internal validity \\ & \\  \hline 
 
\end{tabular}
}
%\caption{Comparison of previous reviews}
\label{tab:plantEng}
\end{table}


%Cambio-1 Se aumenta la cantidad de Minfulness, parece que mas que cambiar la variable independiente cambio la forma de aplicarlo, sería instrumentalización ??

%76 words



% ------------------------------------------------------
% File    : Fam3-ORT-EngV2.tex
% Content : Problems and solutions
% Date    : 1/12/2018
% Version : 1.0
% Authors : M.Cruz
% ------------------------------------------------------

\begin{table}[h]
  %\renewcommand{\arraystretch}{1.45}
  \caption{Instantiation of the proposed template in VV-ORT}
\label{tab:plantEng}
  \centering
	\scriptsize

\begin{tabularx}{\textwidth}{
  >{\hsize=0.55\hsize}X
  >{\hsize=1.45\hsize}X}
  
    \noalign{\smallskip}\hline\noalign{\smallskip}
  
  Field &  Value  \\ 
  \noalign{\smallskip}\hline\noalign{\smallskip}
  
 \textbf {\textit{VV-ORT}} & 
 Replication of experiment \textit{VV-UPM} \\ 
  
    Type of replication &  External \\  
    Site & Universidad ORT (Uruguay) \\  

    Purpose  &  Extend results \\  \hline
%%%%   
    Change \textit{1}   & \textbf{Originally}, the three techniques of verification and validation are used: code reading, equivalence partitioning and branch testing \\& \textbf{In replication} the code reading technique is omitted \\& \textbf{because of} time constraints \\
    
    Modified Dimension & 
    \textbf{Operationalization}, specifically the independent variable  \textit {technique} \\   
    Threat to validity & The change decreases the construct validity  \\  \hline
  
%%%%
    Change \textit{2}   & \textbf{Originally}, three program codes are used \\& \textbf{In replication} one of the programs is  discarded \\& \textbf{because of} time constraints \\  

    Modified Dimension & 
    \textbf{Protocol}, specifically experimental
    material \\   
    Threat to validity  & The change increases the internal validity \\  \hline
%%%%
 
    Change \textit{3}   & \textbf{Originally}, the experiment is carried out in three sessions each of four hours \\& \textbf{In replication} the experiment is executed in a single session  \\  

    Modified Dimension & 
    \textbf{Protocol}, specifically the guides \\  
    Threat to validity  & The change increases the internal validity \\   \hline

%%%%
    Change \textit{4}   & \textbf{Originally}, subjects apply a different technique to  evaluate a program in each of the three sessions \\& \textbf{In replication} the subjects apply the two techniques to the two programs in a single session \\ 
 
    Modified Dimension & 
    \textbf{Protocol}, specifically experimental design \\   
    Threat to validity  & The change increases the internal validity \\    

	\noalign{\smallskip\smallskip}\hline
	\end{tabularx}  
\end{table}


\begin{table}
\caption{\textcolor[rgb]{1,0,0}{Quitar ?.} Analysis of the changes in the family of Requirements analysis experiments}
\label{tab:changesALE}
%\label{tab:1}       % Give a unique label
%\centering
%\small
%\begin{tabular}{| l | c | c | p{2cm} | c | c |c |}
%\begin{tabular}{| l | c | p{2cm} | p{2cm} | c | p{1.5cm} | c |}
\begin{minipage}{6cm}

%\begin{tabular}{| l | l | l | l | l | l | l |}
\begin{tabular}{| l | l | l |p{6cm} | p{1cm}|}
%\hline\noalign{\smallskip}
\hline
\textbf{Id} & \textbf{Original} & \textbf{Change}  & \textbf{Description}& \textbf{Type}\\
\hline
% Mind2 & Mind2  &~& Replications & 2013--2017 & 105 & Current state of replication\\

Q-2009 & Q-2007 & Ch-1 & Effectiveness is not measured  & P1 \\
~ & ~ & Ch-2 & Retention capacity is not measured  & P1 \\
~ & ~ & Ch-3 & The influence of the development experience is analysed  & P1, P2 \\ % P1, P2
~ & ~ & Ch-4 & Interviews are conducted in English  &\ding{51} \\
~ & ~ & Ch-5 & The respondent is changed in interviews & P2, P5 \\ \hline
Q-2011 & Q-2009 & Ch-1 & Group interviews  & \ding{51} \\
~ & ~ & Ch-2 & The influence of skill on requirements and skill in interviews is analyzed  & P1, P2 \\ % P1, P2
~ & ~ & Ch-3 & Increases the interview time  & \ding{51} \\
~ & ~ & Ch-4 & Decreases the time to present information  & \ding{51} \\
~ & ~ & Ch-5 & The consolidation time is limited  & \ding{51} \\
~ & ~ & Ch-6 & The respondent is changed in interviews & P2, P5 \\ \hline
Q-2012 & Q-2011 & Ch-1 & The subjects are professionals  & \ding{51} \\
~ & ~ & Ch-2 & The influence of development skill is analyzed  & P1 \\
~ & ~ & Ch-3 & Decreases consolidation time  & \ding{51} \\
~ & ~ & Ch-4 & No training period  & P1 \\ % P1, P3 
\hline
%T8 Operacionalización pero no conozco la variable
% T3 ya que suprimo la variable
% Relacionando con training es Operacionalización (aunque no se como se llama la variable) y es T0)
E-2012A & Q-2012 & Ch-1 & The influence of knowledge is analysed  & P1 \\
~ & ~ & Ch-2 &  Change in experimental design  & \ding{51} \\
~ & ~ & Ch-3 &  Individual interviews  & \ding{51} \\
~ & ~ & Ch-4 &  The language is a blocking variable.  & \ding{51} \\
~ & ~ & Ch-5 &  The respondent is a blocking variable  & \ding{51} \\
~ & ~ & Ch-6 &  There are two responders  & \ding{51} \\% Cambiar en la plantilla a Operacionalización sin elemento
~ & ~ & Ch-7 &  Groups are formed  & \ding{51} \\ 
~ & ~ & Ch-8 &  Decreases elicitation time  & \ding{51} \\
~ & ~ & Ch-9 &  Increases consolidation time  & \ding{51} \\
~ & ~ & Ch-10 &  The influence of the difficulty of the problem is analysed  & P1 \\ \hline
E-2012B & E-2012A & Ch-1 & The problem domains are modified & P2 \\
~ & ~ & Ch-2 &  The order of the problems is changed  & P2 \\
% Ya no es T6 ya que identifico el cambio de orden
~ & ~ & Ch-3 &  \textcolor[rgb]{1,0,0}{The experiment is conducted at the end of the course}   & P2, P4\\  \hline %P2, P4
E-2013 & E-2012B & Ch-1 & Change in experimental design  & \ding{51} \\
~ & ~ & Ch-2 &  Training is provided  & P3 \\ \hline
E-2014 & E-2013 & Ch-1 & There are one responders  & \ding{51} \\
~ & ~ & Ch-2 &  Increases training time  & P3\\ \hline
E-2015 & E-2013 & Ch-1 & Increases training time  & P3 \\ 


%\noalign{\smallskip}\hline
\hline

\end{tabular}
\end{minipage}
%\caption{Comparison of related reviews}


\end{table}


%76 words



\begin{table}
\caption{\textcolor[rgb]{1,0,0}{Quitar ?.} Analysis of the changes in the series of experiments on Code evaluation techniques}
\label{tab:changesVV}
\label{tab:1}       % Give a unique label
%\centering
%\small
%\begin{tabular}{| l | c | c | p{2cm} | c | c |c |}
%\begin{tabular}{| l | c | p{2cm} | p{2cm} | c | p{1.5cm} | c |}
\begin{minipage}{6cm}

%\begin{tabular}{| l | l | l | l | l | l | l |}
\begin{tabular}{| l | l | l |p{6cm} |  l |}
%\hline\noalign{\smallskip}
\hline
\textbf{Id} & \textbf{Original} & \textbf{Change}  & \textbf{Description}& \textbf{Type}\\
\hline

VV-UPM1 & VV-UPM & Ch-1 & The visibility of the error is analysed  & P1, P3 \\% Es P1 y P3 
 
~ & ~ & Ch-2 &  Two versions of each program  & P1 \\
~ & ~ & Ch-3 &  All types of failures are duplicated  & \ding{51} \\
~ & ~ & Ch-4 &  Test cases are provided & \ding{51} \\
~ & ~ & Ch-5 &  A program is discarded & \ding{51} \\
~ & ~ & Ch-6 &  Each subject applies the three techniques & \ding{51} \\ \hline
VV-UPV & VV-UPM & Ch-1 & A reading technique is omitted & \ding{51} \\
~ & ~ & Ch-2 &  The duration of the sessions is limited & \ding{51} \\
~ & ~ & Ch-3 &  The duration of training is reduced & \ding{51} \\
~ & ~ & Ch-4 &  \textcolor[rgb]{1,0,0}{The moment of application of the treatment is changed} & P4 \\
% T0, 
~ & ~ & Ch-5 &  Changes in the application of techniques & \ding{51} \\
~ & ~ & Ch-6 &  Changes in the application of test cases & \ding{51} \\ \hline

VV-Uds & VV-UPM & Ch-1 & The duration of the sessions is limited & \ding{51} \\
~ & ~ & Ch-2 &  Change at time of test case execution & \ding{51} \\
% T0,  Protocolo en concreto el orden de ejecución
~ & ~ & Ch-3 &  The subjects work in pairs & \ding{51} \\
~ & ~ & Ch-4 &  The duration of training is reduced & \ding{51} \\
~ & ~ & Ch-5 &  \textcolor[rgb]{1,0,0}{The moment of application of the treatment is changed} & P4 \\ \hline
% T0, No estaba claro 
VV-ORT & VV-UPM & Ch-1 &  A reading technique is omitted & \ding{51} \\
~ & ~ & Ch-2 &  A program is discarded & \ding{51} \\
%Instrumentalizacion en concreto el material experimental
~ & ~ & Ch-3 &  There is only one session & P2 \\
~ & ~ & Ch-4 &  Changes in the application of techniques to programs. & \ding{51} \\
~ & ~ & Ch-5 &  No test cases are executed & \ding{51} \\

%\noalign{\smallskip}\hline
\hline

\end{tabular}
\end{minipage}
%\caption{Comparison of related reviews}


\end{table}


%76 words



%The acronyms used to reference the original and the replications are VV-UPM, VV-UPM1, VV-UPV, VV-Uds and VV-ORT. 

All of the replications are described based on the original experiment.
To illustrate the use of the template, table \ref{tab:plantEng} shows the result of applying the template and L–patterns to define the changes made in one of the replications of this family, specifically in the replication carried out at the University of Uruguay.


%In Appendix A, tables Ta, Tb, Tc and Td  show the result of applying the template and L–patterns to define the changes of the replications of this series. 

%\subsubsection{Lessons learned} %\label{sec:LessonsVV}

%The table \ref{tab:plantilla-V2} shows the template with the modifications after its application to the three series of experiments.

%% ------------------------------------------------------
% File    : Jbd-Tplantilla.tex
% Content : Plantilla
% Date    : 12/Feb/2018
% Version : 1.0
% Authors : 
% ------------------------------------------------------
% ------------------------------------------------------
% Last update: ...// 
% - Added ..
% ------------------------------------------------------


%\begin{table}[htpb]
\begin{table}
\caption{Template for specifying changes in a replication after applying lessons learned}
\label{tab:plantilla-V2}
%\caption{Template for specifying changes in a replication}
%\label{tab:1}       % Give a unique label
%\centering
%\small
\begin{tabular}{| p{3.3cm} | p{9cm} |}
%Cuasi Experimento 
%RQ o GQM & \{Del original ?? Yo no lo pondría\}  \\  \hline
\hline


\textbf {\textless\textit{Acronym}\textbf {\textgreater} \#{\textless\textit{n}\textgreater}} & Replication of experiment \textless \textit{Acronym of experiment} \textgreater \#{\textless\textit{m}\textgreater} \\  \hline
%Environment
Site  &   {\textless\textit{Date}\textgreater}  \\   &  \textless  \textit{Place}\textgreater \\    \hline

%Empirical method & \{Controlled experiment  \hspace{1 mm} \textbar \hspace{1 mm}  Quasi-experiment \textbar  \hspace{1 mm}  Survey \textbar \hspace{1 mm}  Case study \textbar \hspace{1 mm}  Observational study \textbar \hspace{1 mm}  Pilot study \}  \\  \hline
Type of replication & \{Internal \textbar \hspace{0.5 mm}  External  \}  \\  \hline


%Target  &  \parbox[t]{9cm} {\{Confirm results \textbar }  \parbox[t]{9cm} { Generalize the results \textbar}   \\   & \textless\textit{Custom target}\textgreater  \}  \\   \hline \hline
Purpose  &  \parbox[t]{9cm} {\{Confirm results \textbar }  \parbox[t]{9cm} {Extend results \textbar} \hspace{1 mm}  Overcome any limitations  \} \\  \hline \hline

Change \#\textless\textit  {i..j}\textgreater  & \parbox[t]{9cm} {Originally, \textless\textit{description of the situation in original experiment}\textgreater} \parbox[t]{9cm}{In replication \textless\textit{description of the situation in replication}\textgreater}  \{in order to  \textbar \hspace{1 mm} because of  \}  \textless\textit{cause of change}\textgreater  \\  \hline

% con el fin de | debido a
[Modified Dimension] & 
\parbox[t]{9cm}  { \{Operationalization  [(, in particular, of the  \{dependent \textbar  independent\} variable \textless \textit{variable name}\textgreater)]  \textbar }  
\parbox[t]{9cm} {Population[(, in particular, experimental \{subjects \textbar objects\} )]\textbar} 
\parbox[t]{9cm} {Protocol [(, in particular, \{experimental design \textbar experimental material \textbar guides \textbar measuring instruments \textbar order of application \} )] \textbar } 
\parbox[t]{9cm}  {Experimenters [(, in particular, \{ designer \textbar   analyst \textbar  trainer \textbar  monitor \textbar  measurer\})]  \} }   \\  \hline 


Threat to validity addressed  &  The change increases  \{the construct  \textbar external \textbar internal \textbar conclusion  \} validity \\  \hline
[Remarks]  &   {\textless\textit{Remarks}\textgreater}  \\  \hline
% se utilizan distintas instancias del mismo tipo de objeto experimental

\end{tabular}

\end{table}


%76 words


%\subsection{Purpose of the replications}

%% -------------------

%\begin{table}[htpb]
\begin{table}
\label{tab:purpose}
\caption{Purpose of replications}
%\small
\begin{tabular}{| l |c | c | c | c|}

%\begin{tabular}{| l |c | c | c | c| c | }
\hline

%\textbf{Related works} & \textbf{Confirm}  & \textbf{Extend} & \textbf{Overcome} &  \textbf{Not completed}   \\ \hline
\textbf{Related works} & \textbf{Confirm results}  & \textbf{Extend results} & \textbf{Overcome limitations}   \\ \hline
Mind \#2  & \ding{51} &   &    \\ \hline
Q-2009   & \ding{51} &   &    \\ \hline
%Q-2009   &  &   &  & \ding{51}   \\ \hline
Q-2009   &  &   &     \\ \hline
Q-2012   &  & \ding{51}   &    \\ \hline
E-2012A   &  & \ding{51}  &     \\ \hline
E-2012B   & \ding{51} &   &     \\ \hline
E-2013   &  &   & \ding{51}    \\ \hline
E-2014   &  \ding{51} &   &     \\ \hline
E-2015   & \ding{51} &   &      \\ \hline
VV-UPM1   &  &   & \ding{51}     \\ \hline
VV-UPV   &  & \ding{51}   &     \\ \hline
VV-Uds   &  & \ding{51}  &     \\ \hline
VV-ORT   &  & \ding{51}  &     \\ \hline


%Mind   & Guidelines & \ding{51}  &\ding{55} &  \ding{55} \\ \hline

%\noalign{\smallskip}\hline
%Confirm results|Extend results|Overcome any limitations}
\end{tabular}

\end{table}


%76 words

%The template is applied to the first family of experiments. 
\subsection{Procedure for template validation}
\label{sec:procedure}
The application of the template to the three families of experiments is an iterative process. The following phases can be distinguished: 

\begin{itemize}
 \item The first family of experiments is surveyed. 
    \begin{itemize}
        \item For each replication, the template is instantiated.
        \item If there is any problem in the definition of any field of the template, it is noted (see Table \ref{tab:plantillaProblem}).
        \begin{itemize}
            \item For each change, if there is any problem in defining the change, it is analyzed and typified (see Table \ref{tab:tipos}).
            \item The change is associated with one or more of the typified values.
        \end{itemize}
       
        \item The impact of the problems detected in the template and in the metamodel is analysed (see Section \ref{sec:impact}). 
    \end{itemize}

  
 \item Repeat the process with the next family updating Table \ref{tab:plantillaProblem} and Table \ref{tab:tipos} when necessary.
\end{itemize}

\subsection{Findings on the application of the template}
\label{sec:findings}
Table \ref{tab:plantillaProblem} shows, over the template, the difficulties faced when filling in each of its fields.
%
\begin{table}
\caption{Findings in each field of the template.}
\label{tab:plantillaProblem}

\begin{tabular}{| p{3.3cm} | p{9cm} |}

\hline

\textbf {\textless\textit{Acronym}\textbf {\textgreater} \#{\textless\textit{n}\textgreater}}  & \ding{51} (1)  \\  \hline

Site \& date  &  (2)   \\  \hline

Type of replication & \ding{51}  \\  \hline

%Purpose  &  \parbox[t]{9cm} {\{Confirm results \textbar }  \parbox[t]{9cm} {Extend results \textbar} \hspace{1 mm}  Overcome any limitations  \} \\  \hline 

Purpose  & \ding{51} (3) \\  \hline \hline

Change \#\textless\textit  {i..j}\textgreater  &   In 8 of the 59 changes, the reason for the change is not specified.  \\  \hline

[Modified Dimension] & In 6 of the 59 changes, the affected dimension or element is not identified.
   \\  \hline

Threat to validity   &  In 10 of the 59 changes, the threat to validity is not affected or decreases however the pattern does not allow this specification. \\  \hline
[Comments]  &  \ding{51} (4) \\  \hline


\end{tabular}

\end{table}


% ------------------------------------------------------
% File    : Fam-TplantillaProblemV2.tex
% Content : Problems and solutions
% Date    : 1/12/2018
% Version : 1.0
% Authors : M.Cruz
% ------------------------------------------------------

\begin{table}[h]
  %\renewcommand{\arraystretch}{1.45}
  \caption{Findings in each field of the template.}
  \label{tab:plantillaProblem}
  \centering
	\scriptsize
  %begin{tabularx}{\textwidth}{cXX}
\begin{tabularx}{0.9\textwidth}{
  >{\hsize=0.65\hsize}X
  >{\hsize=0.01\hsize}X
  >{\hsize=1.34\hsize}X}
  
	%\hline\noalign{\smallskip}
		
    %\textbf{Id\#} & 
    %\textbf{Problems detected in the use of template} & \textbf{Proposed solutions} \\

	\noalign{\smallskip}\hline\noalign{\smallskip}
	\textbf {\textless\textit{Acronym}\textbf {\textgreater} \#{\textless\textit{n}\textgreater}}  & &\ding{51} (1)  \\  
    Site \& date  & & (2)   \\ 
    Type of replication & & \ding{51}  \\ Purpose  & & \ding{51} (3) \\  \hline 
    Change \#\textless\textit  {i..j}\textgreater  &  & In 8 of the 59 changes, the reason for the change is not specified \\ 
    
    [Modified Dimension] & & In 6 of the 59 changes, the affected dimension or element is not identified \\

    Threat to validity   &  & In 10 of the 59 changes, the threat to validity is not affected or decreases however the pattern does not allow this specification \\
    
    [Comments]  & & \ding{51} (4) \\

	\noalign{\smallskip\smallskip}\hline
	\end{tabularx}  
\end{table}

%
\begin{table}
\caption{Findings in each field of the template.}
\label{tab:plantillaProblem2}

\begin{tabular}{| p{3.3cm} | p{9cm} |}

\hline

\textbf {\textless\textit{Acronym}\textbf {\textgreater} \#{\textless\textit{n}\textgreater}}  & Although replication is identified in the template by a code or acronym relating to the baseline experiment and followed by a sequential number, it is advisable to follow the author's notation if different.  \\  \hline

Site \& date  & We realise the need to add the \emph{date} and \emph{site} where the replication is carried out to the template. A new \emph{site} dimension is identified in \cite{Juristo2012}; if replication is carried out at a \emph{site} other than the original experiment, the \emph{site} dimension is changed. Nevertheless, we consider \emph{site} as a template field and not as a new dimension.   \\  \hline

Type of replication & \ding{51}  \\  \hline

%Purpose  &  \parbox[t]{9cm} {\{Confirm results \textbar }  \parbox[t]{9cm} {Extend results \textbar} \hspace{1 mm}  Overcome any limitations  \} \\  \hline 

Purpose  & Except for one of the replications in which its objective is not specified, the rest fits into one of the three possible values to be selected in the template. Purposes of replications of all the proposed types have been found. \\  \hline \hline

Change \#\textless\textit  {i..j}\textgreater  &   In 8 of the 59 changes, the reason for the change is not specified.  \\  \hline

[Modified Dimension] & In 6 of the 59 changes, the affected dimension or element is not identified.
   \\  \hline

Threat to validity   &  In 10 of the 59 changes, the threat to validity is not affected or decreases however the pattern does not allow this specification. \\  \hline
[Comments]  &   The \textit{quasi-experiments}, are included in the \textit{controlled experiments} category following \cite{wohlin:experimentation}, nevertheless, if the author defines replication as a \textit{quasi-experiment}, it should be reflected in the template (e.g. in the comments). \\  \hline


\end{tabular}

\end{table}


\begin{itemize}

\item (1) Although replication is identified in the template by a code or acronym relating to the baseline experiment and followed by a sequential number, it is advisable to follow the author's notation if different.
\item (2) We realize the need to add the \emph{date} and \emph{site} where the replication is carried out to the template. A new \emph{site} dimension is identified in \cite{Juristo2012}; if replication is carried out at a \emph{site} other than the original experiment, the \emph{site} dimension is changed. Nevertheless, we consider \emph{site} as a template field and not as a new dimension.  
 
%\item When the change affects more than one dimension it is advisable to split into several simple changes

\item (3) Concerning the \emph{Purpose of Replication}, except for one of the replications in which its objective is not specified, the rest fits into one of the three possible values to be selected in the template. Purposes of replications of all the proposed types have been found.

\item (4) The \textit{quasi-experiments}, are included in the \textit{controlled experiments} category following \cite{wohlin:experimentation}, nevertheless, if the author defines replication as a \textit{quasi-experiment}, it should be reflected in the template (e.g. in the comments).
%\item The empirical method has been removed from the template and the use of the template is recommended only for \textit{controlled experiments}.  In \emph{surveys}, \emph{observational studies} and other types of empirical methods, most of the concepts in the template, such as independent variables, do not exist.

%The use of the template avoids the lack of information on the purpose of the replication.
\end{itemize}

\subsection{Impact of the findings on the template and metamodel}
\label{sec:impact}

The problems arising in the definition of changes in replications have been typified in order to be able to analyse whether they affect: i) the metamodel and, therefore, also the template and its L-patterns; ii) only the template and L-patterns (see Table \ref{tab:tipos}).

% shows the typified problems faced through the iterative process for template validation in the replications of each family. The solution to these problems involves changes in the metamodel or changes in the template or its L-patterns. On the other hand, the lack of information when describing the changes will benefit from the use of the template.

\subsubsection{Modifications to the metamodel:}
\begin{itemize}
    \item  Due to problem P4 (see Table 5), it is necessary to modify the metamodel. Changes in \emph{context variables} are included within the \emph{operationalization dimension}. \emph{Context variables} are independent variables that were controlled at a fixed level during the experiment \cite{wohlin:experimentation}.
    
    This modification in the metamodel has repercussions on the definition of the template and L-patterns.
%    \item  Add the \emph{"sequence of steps"} as one of the possible elements to modify in the \emph{protocol dimension}.
\end{itemize}

\subsubsection{Modifications to the template and L-patterns:}
\begin{itemize}
    \item Due to problem P1 (see Table \ref{tab:tipos}), it is necessary to modify the pattern of \emph{threats to validity} to reflect that there are changes that do not affect the validity. 
    
    The new L-pattern to describe the  \emph{threats to validity} is:
    
     The change  \{(\{increases\textbar decreases \} \\
     \{the construct  \textbar external \textbar internal \textbar conclusion\} validity) \\
     \textbar  not affect the validity \}  \\
\end{itemize}

\subsection{Analysis of missing information}
\label{sec:missing}

%Due to the difficulty of their specification, some of the fields of the template are optional.  
In the three families analysed, among the fields not specified by the authors, \emph{"reason for the change"} and \emph{ "threat to validity"} stand out.   
The number of changes incompletely defined with the template due to \emph{"missing information"} is 8 out of 59. However, these fields are easily identifiable by the researchers who carry out the replication and fill in the template.

In order to clearly document the changes it is necessary to fill in all the fields in the template. This leads us to reinforce the template by making the fields that were optional mandatory so that this information is not missing.
It supports the idea that the template is necessary so that replications are not \textbf{under-specified}. 

To facilitate the identification of the \emph{threat to validity}, in \cite{gomez2014understanding} the threats are identified based on changes in the experimental baseline.

\begin{itemize}
%\item \emph{Changing the experimenters} does not control any threat. It only ensures that the experimenters do not influence the results \cite{Juristo2012}.
\item \emph{Changing the protocol:} Internal validity is addressed. Results are independent of experimental conditions.
\item \emph{Changing the operationalization:} Construct validity is addressed. %Determine limits for Operationalizations (range of variation of treatments and measures used ) \cite{Juristo2012}.
\item \emph{Changing the population:} External validity is addressed. %Determine limits in the population properties \cite{Juristo2012}.

\end{itemize}
%
\begin{table}
\caption{Types of problems identified and proposed solutions}
\label{tab:tipos}

\begin{tabular}{| l |p{5cm} |p{6cm} |}

\hline
\textbf{Id\#} & \textbf{Problems detected in the use of template} & \textbf{Proposed solutions
} \\
\hline
%P1-P
P1 & A \emph{dependent/independent variable} is suppressed or added. The \emph{operationalization dimension} is modified; no \emph{threat to validity} is identified. &  \textbf{Modify the template} so that the L-pattern allows you to specify that the change does not affect any \emph{threat to validity}. \\ \hline



P2 & The  \emph{reason for the change}  is not specified.  & The use of the template avoids this missing information. \\ \hline
P3 & The  \emph{name of the modified variable} is unknown. The change modifies the  \emph{operationalization dimension}.  & The use of the template avoids this missing information. \\ \hline
%T5 & It is not clear whether the change affects the operationalization or protocol dimension & The modified dimension is not identified \\ \hline
%T6 & It is not clear which element of the protocol dimension is affected by the change & The affected element is not identified \\ \hline
%T7 & The change modifies the protocol dimension & The affected element  is the change in the order of actions. This element is new \\ \hline
% P4-M Or
%P4 & The change consists of \emph{"alter the sequence of steps"}. It does not fit into any of the elements of the \emph{protocol dimension}. & \textbf{Modify the metamodel} to add the \emph{"sequence of steps"} as one of the possible elements to modify in the \emph{protocol dimension}.  \\ \hline
% P5-FI

% P6-M Co
P4 & The change consists of modifying a \emph{context variable}. The affected dimension is not identified.  & \textbf{Modify the metamodel} considering \emph{context variables} included in the \emph{operationalization dimension}. Therefore, it also implies \textbf{modifying the template}. \\ \hline

%T8 & The change modifies the operationalization dimension; the affected variable is not identified
 %\\ \hline

\end{tabular}
%\end{minipage}


%\caption{Types of problems identified and recommended solutions}
%\label{tab:tipos}
\end{table}





%% ------------------------------------------------------
% File    : FamTChangesSolV2.tex
% Content : Problems and solutions
% Date    : 1/12/2018
% Version : 1.0
% Authors : M.Cruz
% ------------------------------------------------------

\begin{table}[h]
  %\renewcommand{\arraystretch}{1.45}
  \caption{Types of problems identified and proposed solutions}
  \label{tab:tipos}
  \centering
	\scriptsize
  %begin{tabularx}{\textwidth}{cXX}
\begin{tabularx}{0.9\textwidth}{c
  >{\hsize=0.88\hsize}X
  >{\hsize=0.02\hsize}X
  >{\hsize=1.10\hsize}X}
  
	\hline\noalign{\smallskip}
		
    %\textbf{Id\#} & 
    %\textbf{Problems detected in the use of template} & \textbf{Proposed solutions} \\
    
    Id\# & 
    Problems detected in the use of template &  &
    Proposed solutions \\

	\noalign{\smallskip}\hline\noalign{\smallskip}
	P1 & A \emph{dependent/independent variable} is suppressed or added. The \emph{operationalization dimension} is modified; no \emph{threat to validity} is identified & & \textbf{Modify the template} so that the L-pattern allows you to specify that the change does not affect any \emph{threat to validity} \\ \\

    P2 & The  \emph{reason for the change}  is not specified  & &
    The use of the template avoids this missing information \\ \\
    
    P3 & The  \emph{name of the modified variable} is unknown. The change modifies the  \emph{operationalization dimension}  & &
    The use of the template avoids this missing information \\ \\
	
	P4 & The change consists of modifying a \emph{context variable}. The affected dimension is not identified  & &
	\textbf{Modify the metamodel} considering \emph{context variables} included in the \emph{operationalization dimension}. Therefore, it also implies \textbf{modifying the template} \\ 

	\noalign{\smallskip\smallskip}\hline
	\end{tabularx}  
\end{table}


\begin{table}
\caption{\textcolor[rgb]{1,0,0}{Probando} Analysis of the changes in the series of experiments on Code evaluation techniques}
\label{tab:changesVV}
\label{tab:1}       
\begin{minipage}{6cm}

\begin{tabular}{| l | l | l  | l |}

\hline
\textbf{Id} & \textbf{Original} & \textbf{Change}  &  \textbf{Type}\\
\hline

Fam\#1 & VV-UPM1 & Ch-1  & P1, P3 \\% Es P1 y P3 
 
~ & ~ & Ch-2  & P1 \\
~ & ~ & Ch-3 &   \ding{51} \\
~ & ~ & Ch-4 &   \ding{51} \\
~ & ~ & Ch-5 &   \ding{51} \\
~ & ~ & Ch-6 &   \ding{51} \\ 
~ & VV-UPV & Ch-1  & \ding{51} \\
~ & ~ & Ch-2  & \ding{51} \\
~ & ~ & Ch-3  & \ding{51} \\
~ & ~ & \textcolor[rgb]{1,0,0}{Ch-4}  & P4 \\
% T0, 
~ & ~ & Ch-5 & \ding{51} \\
~ & ~ & Ch-6 & \ding{51} \\ 

~ & VV-Uds & Ch-1 & \ding{51} \\
~ & ~ & Ch-2 & \ding{51} \\
% T0,  Protocolo en concreto el orden de ejecución
~ & ~ & Ch-3 & \ding{51} \\
~ & ~ & Ch-4 & \ding{51} \\
~ & ~ & \textcolor[rgb]{1,0,0}{Ch-5}  & P4 \\ 
% T0, No estaba claro 
~ & VV-ORT  & Ch-1  & \ding{51} \\
~ & ~ & Ch-2 & \ding{51} \\
%Instrumentalizacion en concreto el material experimental
~ & ~ & Ch-3  & P2 \\
~ & ~ & Ch-4  & \ding{51} \\
~ & ~ & Ch-5 & \ding{51} \\

%\noalign{\smallskip}\hline
\hline

\end{tabular}
\end{minipage}
%\caption{Comparison of related reviews}


\end{table}


%76 words




\subsection{Summary of specified changes}
\label{sec:summary}

%
\begin{table}
\caption{Analysis of the number of changes in the three families of experiments}
\label{tab:agrupa}
%\label{tab:1}       % Give a unique label
%\centering  
%\small
%\begin{tabular}{| l | c | c | p{2cm} | c | c |c |}
%\begin{tabular}{| l | c | p{2cm} | p{2cm} | c | p{1.5cm} | c |}
\begin{minipage}{6cm}

%\begin{tabular}{| l | l | l | l | l | l | l |}
\begin{tabular}{| p{6.5cm}| c | c | c |c |}
%\hline\noalign{\smallskip}
\hline
 & \textbf{Fam\#1} & \textbf{Fam\#2} & \textbf{Fam\#3} & Sum\\ \hline
Number of replications & 2 & 8 & 4 & 14\\ \hline
Number of changes  & 4 & 33 & 22 & 59\\ \hline
Average number of changes per replication & 2 & 4.1 & 5.5 & 4.2\\ \hline

Num. changes fully specified original template & 4 & 17 & 17 & 38\\ \hline
Num. changes with problem P1 & 0 & 6 & 1 & 7\\ \hline  %Template
Num. changes with problem P2 & 0 & 4 & 1 & 5\\ \hline %Incomp
Num. changes with problem P3 & 0 & 3 & 0 & 3\\ \hline %Incomp
Num. changes with problem P4 & 0 & 0 & 2 & 2\\\hline %Meta (Temp)
Num. changes with problems P1\&P2  & 0 & 2 & 0 & 2\\ \hline %Template + Incomp
Num. changes with problems P1\&P3 & 0 & 0 & 1 & 1\\ \hline %Template + Incomp
Num. changes with problems P2\&P4 & 0 & 1 & 0 & 1\\ \hline %Meta (Temp) + Incomp
 \hline 
Number of changes incompletely defined with the template due to missing information & 0 & 10 & 2 & 12\\ \hline % Con T3+T5  lack of information

Number of changes that lead to modify the original template & 0 & 8 & 2 & 10\\ \hline 

Number of changes that lead to modify the metamodel and therefore also the template & 0 & 1 & 2 & 3\\ \hline 

\end{tabular}
\end{minipage}
%\caption{Comparison of related reviews}


\end{table}


%76 words


% ------------------------------------------------------
% File    : Fam-AgrupaV2.tex
% Content : Analysis of changes
% Date    : 1/12/2018
% Version : 1.0
% Authors : M.Cruz
% ------------------------------------------------------

\begin{table}[h]
  %\renewcommand{\arraystretch}{1.45}
  \caption{Analysis of the number of changes in the three families of experiments}
  \label{tab:agrupa}
  \centering
	\scriptsize
  %begin{tabularx}{\textwidth}{cXX}
\begin{tabularx}{0.9\textwidth}{Xcccc}
  
	\hline\noalign{\smallskip}
		
    %\textbf{Id\#} & 
    %\textbf{Problems detected in the use of template} & \textbf{Proposed solutions} \\
    
    & \textbf{Fam\#1} & \textbf{Fam\#2} & \textbf{Fam\#3} & Sum \\

	\noalign{\smallskip}\hline\noalign{\smallskip}
	Number of replications & 2 & 8 & 4 & 14\\ 
	Number of changes  & 4 & 33 & 22 & 59\\ 
    Average number of changes per replication & 2 & 4.1 & 5.5 & 4.2\\ 
    Num. changes fully specified original template & 4 & 17 & 17 & 38\\ \hline
    Num. changes with problem P1 & 0 & 6 & 1 & 7\\ 
    Num. changes with problem P2 & 0 & 4 & 1 & 5\\ 
    Num. changes with problem P3 & 0 & 3 & 0 & 3 \\ 
    Num. changes with problem P4 & 0 & 0 & 2 & 2 \\
    Num. changes with problems P1\&P2  & 0 & 2 & 0 & 2 \\ 
    Num. changes with problems P1\&P3 & 0 & 0 & 1 & 1 \\ 
    Num. changes with problems P2\&P4 & 0 & 1 & 0 & 1 \\ \hline
    Sum. changes with problems (P1, P2, P3, P4) & 0 & 16 & 5 & 21 \\
    \hline
    \\ Number of changes incompletely defined with the template due to missing information & 0 & 10 & 2 & 12\\ 
    Number of changes that lead to modify the original template & 0 & 8 & 2 & 10\\ 
    Number of changes that lead to modify the metamodel and therefore also the template & 0 & 1 & 2 & 3\\ 
	\noalign{\smallskip\smallskip}\hline
	\end{tabularx}  
\end{table}

 
 Table \ref{tab:agrupa} shows the changes defined fitting the template and those that have presented difficulties.
By analyzing this table, two types of situations can be identified:
\begin{itemize}
\item \textbf{Due to limitations in the template}. Implies a change in the template (e.g. modify the pattern of \emph{threats to validity} or consider \emph{context variables}). Some of these situations also result in changes in the metamodel.
\item  \textbf{Due to missing information }. This situation does not imply changes in the template.  It supports the idea that the template is necessary so that applications are not \textbf{under-specified}.
\end{itemize}


%
% 6. Analiza
%\section{Analysis of the changes}
%\label{sec:analiza}
%%\section{Analysis of the changes}
.
%
\begin{table}
\caption{Analysis of the changes in the family of Requirements analysis experiments}
\label{tab:changesALE}
%\label{tab:1}       % Give a unique label
%\centering
%\small
%\begin{tabular}{| l | c | c | p{2cm} | c | c |c |}
%\begin{tabular}{| l | c | p{2cm} | p{2cm} | c | p{1.5cm} | c |}
\begin{minipage}{6cm}

%\begin{tabular}{| l | l | l | l | l | l | l |}
\begin{tabular}{| l | l | l |p{6cm} | p{1cm}|}
%\hline\noalign{\smallskip}
\hline
\textbf{Id} & \textbf{Original} & \textbf{Change}  & \textbf{Description}& \textbf{Type}\\
\hline
% Mind2 & Mind2  &~& Replications & 2013--2017 & 105 & Current state of replication\\
Mind\# 2 & Mind\# 1  & Ch-1 & Increase in the number of sessions & T0 \\
~ & ~ & Ch-2 & Random subject assignment & T0 \\
~ & ~ & Ch-3 & The public speaking workshop is held at the end of the workshop. & T0 \\ \hline

Q-2009 & Q-2007 & Ch-1 & Effectiveness is not measured  & T2 \\
~ & ~ & Ch-2 & Retention capacity is not measured  & T2 \\
~ & ~ & Ch-3 & The influence of the development experience is analysed  & T3, T8 \\
~ & ~ & Ch-4 & Interviews are conducted in English  & T6 \\
~ & ~ & Ch-5 & The respondent is changed  & T6, T8 \\ \hline
Q-2011 & Q-2009 & Ch-1 & Group interviews  & T6 \\
~ & ~ & Ch-2 & The influence of skill on requirements and skill in interviews is analyzed  & T1, T3, T8 \\
~ & ~ & Ch-3 & Increases the interview time  & T5, T6 \\
~ & ~ & Ch-4 & Decreases the time to present information  & T6 \\
~ & ~ & Ch-5 & The consolidation time is limited  & T6 \\
~ & ~ & Ch-6 & The respondent is changed  & T6, T8 \\ \hline
Q-2012 & Q-2011 & Ch-1 & The subjects are professionals  & T0 \\
~ & ~ & Ch-2 & The influence of development skill is analyzed  & T3 \\
~ & ~ & Ch-3 & Decreases consolidation time  & T6 \\
~ & ~ & Ch-4 & No training period  & T3, T4 \\ \hline
%T8 Operacionalización pero no conozco la variable
% T3 ya que suprimo la variable
% Relacionando con training es Operacionalización (aunque no se como se llama la variable) y es T0)
E-2012A & Q-2012 & Ch-1 & The influence of knowledge is analysed  & T3 \\
~ & ~ & Ch-2 &  Change in experimental design  & T0 \\
~ & ~ & Ch-3 &  Individual interviews  & T6 \\
~ & ~ & Ch-4 &  The language is a blocking variable.  & T0 \\
~ & ~ & Ch-5 &  The respondent is a blocking variable  & T0 \\
~ & ~ & Ch-6 &  There are two responders  & T6 \\% Cambiar en la plantilla a Operacionalización sin elemento
~ & ~ & Ch-7 &  Groups are formed  & T0 \\ 
~ & ~ & Ch-8 &  Decreases elicitation time  & T6 \\
~ & ~ & Ch-9 &  Increases consolidation time  & T6 \\
~ & ~ & Ch-10 &  The influence of the difficulty of the problem is analysed  & T3 \\ \hline
E-2012B & E-2012A & Ch-1 & The problem domains are modified & T8 \\
~ & ~ & Ch-2 &  The order of the problems is changed  & T7, T8 \\
% Ya no es T6 ya que identifico el cambio de orden
~ & ~ & Ch-3 &  The experiment is conducted at the end of the course  & T5, T6,T8 \\ \hline
E-2013 & E-2012B & Ch-1 & Change in experimental design  & T0 \\
~ & ~ & Ch-2 &  Training is provided  & T4 \\ \hline
E-2014 & E-2013 & Ch-1 & There are one responders  & T6 \\
~ & ~ & Ch-2 &  Increases training time  & T4\\ \hline
E-2015 & E-2013 & Ch-1 & Increases training time  & T4 \\ 
\hline \hline
VV-UPM1 & VV-UPM & Ch-1 & The visibility of the error is analysed  & T3, T4 \\
% Es T3
~ & ~ & Ch-2 &  Two versions of each program  & T3 \\
~ & ~ & Ch-3 &  All types of failures are duplicated  & T0 \\
~ & ~ & Ch-4 &  Test cases are provided & T0 \\
~ & ~ & Ch-5 &  A program is discarded & T0 \\
~ & ~ & Ch-6 &  Each subject applies the three techniques & T0 \\ \hline
VV-UPV & VV-UPM & Ch-1 & A reading technique is omitted & T0 \\
~ & ~ & Ch-2 &  The duration of the sessions is limited & T6 \\
~ & ~ & Ch-3 &  The duration of training is reduced & T0 \\
~ & ~ & Ch-4 &  Change the order of application & T7 \\
% T0, 
~ & ~ & Ch-5 &  Changes in the application of techniques & T6 \\
~ & ~ & Ch-6 &  Changes in the application of test cases & T6 \\ \hline

VV-Uds & VV-UPM & Ch-1 & The duration of the sessions is limited & T6 \\
~ & ~ & Ch-2 &  Change at time of test case execution & T7 \\
% T0,  Protocolo en concreto el orden de ejecución
~ & ~ & Ch-3 &  The subjects work in pairs & T6 \\
~ & ~ & Ch-4 &  The duration of training is reduced & T0 \\
~ & ~ & Ch-5 &  Change the order of application & T5 \\ \hline
% T0, No estaba claro 
VV-ORT & VV-UPM & Ch-1 &  A reading technique is omitted & T0 \\
~ & ~ & Ch-2 &  A program is discarded & T0 \\
%Instrumentalizacion en concreto el material experimental
~ & ~ & Ch-3 &  There is only one session & T6, T8 \\
~ & ~ & Ch-4 &  Changes in the application of techniques to programs. & T6 \\
~ & ~ & Ch-5 &  No test cases are executed & T6 \\


%\noalign{\smallskip}\hline
\hline

\end{tabular}
\end{minipage}
%\caption{Comparison of related reviews}


\end{table}


%76 words


% 
\begin{table}
\caption{Analysis of the number of changes in the three families of experiments}
\label{tab:agrupa}
%\label{tab:1}       % Give a unique label
%\centering  
%\small
%\begin{tabular}{| l | c | c | p{2cm} | c | c |c |}
%\begin{tabular}{| l | c | p{2cm} | p{2cm} | c | p{1.5cm} | c |}
\begin{minipage}{6cm}

%\begin{tabular}{| l | l | l | l | l | l | l |}
\begin{tabular}{| p{6.5cm}| c | c | c |c |}
%\hline\noalign{\smallskip}
\hline
 & \textbf{Fam\#1} & \textbf{Fam\#2} & \textbf{Fam\#3} & Sum\\ \hline
Number of replications & 2 & 8 & 4 & 14\\ \hline
Number of changes  & 4 & 33 & 22 & 59\\ \hline
Average number of changes per replication & 2 & 4.1 & 5.5 & 4.2\\ \hline

Num. changes fully specified original template & 4 & 17 & 17 & 38\\ \hline
Num. changes with problem P1 & 0 & 6 & 1 & 7\\ \hline  %Template
Num. changes with problem P2 & 0 & 4 & 1 & 5\\ \hline %Incomp
Num. changes with problem P3 & 0 & 3 & 0 & 3\\ \hline %Incomp
Num. changes with problem P4 & 0 & 0 & 2 & 2\\\hline %Meta (Temp)
Num. changes with problems P1\&P2  & 0 & 2 & 0 & 2\\ \hline %Template + Incomp
Num. changes with problems P1\&P3 & 0 & 0 & 1 & 1\\ \hline %Template + Incomp
Num. changes with problems P2\&P4 & 0 & 1 & 0 & 1\\ \hline %Meta (Temp) + Incomp
 \hline 
Number of changes incompletely defined with the template due to missing information & 0 & 10 & 2 & 12\\ \hline % Con T3+T5  lack of information

Number of changes that lead to modify the original template & 0 & 8 & 2 & 10\\ \hline 

Number of changes that lead to modify the metamodel and therefore also the template & 0 & 1 & 2 & 3\\ \hline 

\end{tabular}
\end{minipage}
%\caption{Comparison of related reviews}


\end{table}


%76 words


%
\begin{table}
\caption{\textcolor[rgb]{1,0,0}{Quitar ?.} Analysis of the changes in the replication of the Mindfulness experiment.}
\label{tab:changesMI}
%\label{tab:1}       % Give a unique label
%\centering  
%\small
%\begin{tabular}{| l | c | c | p{2cm} | c | c |c |}
%\begin{tabular}{| l | c | p{2cm} | p{2cm} | c | p{1.5cm} | c |}
\begin{minipage}{6cm}

%\begin{tabular}{| l | l | l | l | l | l | l |}
\begin{tabular}{| l | l | l |p{6cm} |  l |}
%\hline\noalign{\smallskip}
\hline
\textbf{Id} & \textbf{Original} & \textbf{Change}  & \textbf{Description}& \textbf{Type}\\
\hline
% Mind2 & Mind2  &~& Replications & 2013--2017 & 105 & Current state of replication\\
 Mind\# 2 & Mind\# 1  & Ch-1 & Increase in the number of sessions & \ding{51} \\
~ & ~ & Ch-2 & Random subject assignment & \ding{51} \\
~ & ~ & Ch-3 & The public speaking workshop is held at the end of the workshop. & \ding{51} \\ 

%\noalign{\smallskip}\hline
\hline

\end{tabular}
\end{minipage}
%\caption{Comparison of related reviews}


\end{table}


%76 words


%
\begin{table}
\caption{\textcolor[rgb]{1,0,0}{Quitar ?.} Analysis of the changes in the family of Requirements analysis experiments}
\label{tab:changesALE}
%\label{tab:1}       % Give a unique label
%\centering
%\small
%\begin{tabular}{| l | c | c | p{2cm} | c | c |c |}
%\begin{tabular}{| l | c | p{2cm} | p{2cm} | c | p{1.5cm} | c |}
\begin{minipage}{6cm}

%\begin{tabular}{| l | l | l | l | l | l | l |}
\begin{tabular}{| l | l | l |p{6cm} | p{1cm}|}
%\hline\noalign{\smallskip}
\hline
\textbf{Id} & \textbf{Original} & \textbf{Change}  & \textbf{Description}& \textbf{Type}\\
\hline
% Mind2 & Mind2  &~& Replications & 2013--2017 & 105 & Current state of replication\\

Q-2009 & Q-2007 & Ch-1 & Effectiveness is not measured  & P1 \\
~ & ~ & Ch-2 & Retention capacity is not measured  & P1 \\
~ & ~ & Ch-3 & The influence of the development experience is analysed  & P1, P2 \\ % P1, P2
~ & ~ & Ch-4 & Interviews are conducted in English  &\ding{51} \\
~ & ~ & Ch-5 & The respondent is changed in interviews & P2, P5 \\ \hline
Q-2011 & Q-2009 & Ch-1 & Group interviews  & \ding{51} \\
~ & ~ & Ch-2 & The influence of skill on requirements and skill in interviews is analyzed  & P1, P2 \\ % P1, P2
~ & ~ & Ch-3 & Increases the interview time  & \ding{51} \\
~ & ~ & Ch-4 & Decreases the time to present information  & \ding{51} \\
~ & ~ & Ch-5 & The consolidation time is limited  & \ding{51} \\
~ & ~ & Ch-6 & The respondent is changed in interviews & P2, P5 \\ \hline
Q-2012 & Q-2011 & Ch-1 & The subjects are professionals  & \ding{51} \\
~ & ~ & Ch-2 & The influence of development skill is analyzed  & P1 \\
~ & ~ & Ch-3 & Decreases consolidation time  & \ding{51} \\
~ & ~ & Ch-4 & No training period  & P1 \\ % P1, P3 
\hline
%T8 Operacionalización pero no conozco la variable
% T3 ya que suprimo la variable
% Relacionando con training es Operacionalización (aunque no se como se llama la variable) y es T0)
E-2012A & Q-2012 & Ch-1 & The influence of knowledge is analysed  & P1 \\
~ & ~ & Ch-2 &  Change in experimental design  & \ding{51} \\
~ & ~ & Ch-3 &  Individual interviews  & \ding{51} \\
~ & ~ & Ch-4 &  The language is a blocking variable.  & \ding{51} \\
~ & ~ & Ch-5 &  The respondent is a blocking variable  & \ding{51} \\
~ & ~ & Ch-6 &  There are two responders  & \ding{51} \\% Cambiar en la plantilla a Operacionalización sin elemento
~ & ~ & Ch-7 &  Groups are formed  & \ding{51} \\ 
~ & ~ & Ch-8 &  Decreases elicitation time  & \ding{51} \\
~ & ~ & Ch-9 &  Increases consolidation time  & \ding{51} \\
~ & ~ & Ch-10 &  The influence of the difficulty of the problem is analysed  & P1 \\ \hline
E-2012B & E-2012A & Ch-1 & The problem domains are modified & P2 \\
~ & ~ & Ch-2 &  The order of the problems is changed  & P2 \\
% Ya no es T6 ya que identifico el cambio de orden
~ & ~ & Ch-3 &  \textcolor[rgb]{1,0,0}{The experiment is conducted at the end of the course}   & P2, P4\\  \hline %P2, P4
E-2013 & E-2012B & Ch-1 & Change in experimental design  & \ding{51} \\
~ & ~ & Ch-2 &  Training is provided  & P3 \\ \hline
E-2014 & E-2013 & Ch-1 & There are one responders  & \ding{51} \\
~ & ~ & Ch-2 &  Increases training time  & P3\\ \hline
E-2015 & E-2013 & Ch-1 & Increases training time  & P3 \\ 


%\noalign{\smallskip}\hline
\hline

\end{tabular}
\end{minipage}
%\caption{Comparison of related reviews}


\end{table}


%76 words


%
\begin{table}
\caption{\textcolor[rgb]{1,0,0}{Quitar ?.} Analysis of the changes in the series of experiments on Code evaluation techniques}
\label{tab:changesVV}
\label{tab:1}       % Give a unique label
%\centering
%\small
%\begin{tabular}{| l | c | c | p{2cm} | c | c |c |}
%\begin{tabular}{| l | c | p{2cm} | p{2cm} | c | p{1.5cm} | c |}
\begin{minipage}{6cm}

%\begin{tabular}{| l | l | l | l | l | l | l |}
\begin{tabular}{| l | l | l |p{6cm} |  l |}
%\hline\noalign{\smallskip}
\hline
\textbf{Id} & \textbf{Original} & \textbf{Change}  & \textbf{Description}& \textbf{Type}\\
\hline

VV-UPM1 & VV-UPM & Ch-1 & The visibility of the error is analysed  & P1, P3 \\% Es P1 y P3 
 
~ & ~ & Ch-2 &  Two versions of each program  & P1 \\
~ & ~ & Ch-3 &  All types of failures are duplicated  & \ding{51} \\
~ & ~ & Ch-4 &  Test cases are provided & \ding{51} \\
~ & ~ & Ch-5 &  A program is discarded & \ding{51} \\
~ & ~ & Ch-6 &  Each subject applies the three techniques & \ding{51} \\ \hline
VV-UPV & VV-UPM & Ch-1 & A reading technique is omitted & \ding{51} \\
~ & ~ & Ch-2 &  The duration of the sessions is limited & \ding{51} \\
~ & ~ & Ch-3 &  The duration of training is reduced & \ding{51} \\
~ & ~ & Ch-4 &  \textcolor[rgb]{1,0,0}{The moment of application of the treatment is changed} & P4 \\
% T0, 
~ & ~ & Ch-5 &  Changes in the application of techniques & \ding{51} \\
~ & ~ & Ch-6 &  Changes in the application of test cases & \ding{51} \\ \hline

VV-Uds & VV-UPM & Ch-1 & The duration of the sessions is limited & \ding{51} \\
~ & ~ & Ch-2 &  Change at time of test case execution & \ding{51} \\
% T0,  Protocolo en concreto el orden de ejecución
~ & ~ & Ch-3 &  The subjects work in pairs & \ding{51} \\
~ & ~ & Ch-4 &  The duration of training is reduced & \ding{51} \\
~ & ~ & Ch-5 &  \textcolor[rgb]{1,0,0}{The moment of application of the treatment is changed} & P4 \\ \hline
% T0, No estaba claro 
VV-ORT & VV-UPM & Ch-1 &  A reading technique is omitted & \ding{51} \\
~ & ~ & Ch-2 &  A program is discarded & \ding{51} \\
%Instrumentalizacion en concreto el material experimental
~ & ~ & Ch-3 &  There is only one session & P2 \\
~ & ~ & Ch-4 &  Changes in the application of techniques to programs. & \ding{51} \\
~ & ~ & Ch-5 &  No test cases are executed & \ding{51} \\

%\noalign{\smallskip}\hline
\hline

\end{tabular}
\end{minipage}
%\caption{Comparison of related reviews}


\end{table}


%76 words


%\textcolor[rgb]{1,0,0}{Quitar ?.} Table \ref{tab:changesMI} shows the data related to the changes in the replication of the \textit {Mindfulness} experiment.
%For each change, the acronym of the replication and the original, an identifier of the change and a brief description appear. It also includes the difficulty in describing the change according to the table \ref{tab:tipos}.
%It can be seen that 100\% of the changes fit the template.

%\textcolor[rgb]{1,0,0}{Quitar ?.} Table \ref{tab:changesALE} shows data to changes in the family of \textit{Requirements analysis} experiments. Of the 33 changes included, 6 have been fully defined according to the template. 
%\textcolor[rgb]{1,0,0}{Quitar ?.} Table \ref{tab:changesVV} shows data to changes in the series of experiments on  \textit{Code evaluation techniques}. In this series, 9 of the 22 changes have been defined without difficulty. In the rest, there has been some difficulty in specifying some fields, such as the modified dimension, the threat affected or the reason for the change.


%Although in some cases it is not specified, it is important to make clear both the purpose of the replication and the reason for each change introduced.  In the template, for the purpose of replication, allows you to choose between three possible prefixed values.






% 7. Trabajos
\section{Related work}
\label{sec:trabajos}

This section reviews some templates and patterns defined or used in the context of SE.

The idea of using templates and patterns to define changes in the replication of experiments is based on the templates proposed by Durán et al. \cite{duran1999requirements} for the elicitation of requirements. The templates and L-patterns have been successfully applied in several areas within SE. Del Rio et al. \cite{del2012defining} have used them for the definition of business process performance indicators.

In the description of experiments, the Goal-Question-Metric (GQM) template by Basili \cite{Basili1994}, recommended by Wohlin \cite{wohlin:experimentation}, should be highlighted for the definition of experiment objectives.

Segura et al.  \cite{segura2017template} propose a template inspired by the GQM template for the definition of metamorphic relationships. The format is textual rather than tabular to facilitate integration into research work.

Finally, in the Design Science Research (DSR) methodology, Wieringa \cite{38631e0608b54d4299d5707f3a78debf} defines a template for the specification of a problem that will lead to future research:\newline
%% ------------------------------------------------------
% File    : Jbd-Trabajos.tex
% Content : Comparación de trabajos relacionados
% Date    : 10/Marzo/2018
% Version : 1.0
% Authors : 
% ------------------------------------------------------
% ------------------------------------------------------
% Last update: ...// 
% - Added ..
% ------------------------------------------------------


%\begin{table}[htpb]
\begin{table}
\label{tab:trabajos}
\caption{Comparison of related work}
%\label{tab:1}       % Give a unique label
%\centering
%\small
% p{5 cm} |p{5 cm} |
%\begin{tabular}{| p{1cm} | p{0.6cm} | p{1 cm}| p{1 cm} |  p{1 cm} |}  &\ding{51
%\begin{tabular}{| l | c | c | p{2cm} | c | c |}
\begin{tabular}{| l |c | c | c | c| c | }

\hline

\textbf{Related works} & \textbf{Topic}  & \textbf{ESE} & \textbf{Use template} &  \textbf{Use pattern}   \\ \hline
Carver  \cite{carver2010towards}   & Guidelines & \ding{51}  &\ding{55} &  \ding{55} \\ \hline
Solari \cite{solari2013identifying} & Reports a replication  & \ding{51} &\ding{55} &\ding{55} \\ \hline
Lung et al. \cite{lung2008difficulty} & Reports a replication  & \ding{51} &\ding{55} &\ding{55}   \\ \hline
Almqvist \cite{1330459} & SLR replications & \ding{51} &\ding{55} &\ding{55}  \\ \hline

Apa et al. \cite{apa2014effectiveness} & Reports a replication & \ding{51} &\ding{55} &\ding{55}  \\ \hline
Fucci et al.  \cite{fucci2016external} & Reports a replication & \ding{51} &\ding{55} &\ding{55}  \\ \hline
Juristo et al.  \cite{juristo2012comparing} & Reports a replication & \ding{51} &\ding{55} &\ding{55}  \\ \hline
Quesada et al.  \cite{quesada2016empirical} & Reports a replication & \ding{51} &\ding{55} &\ding{55}  \\ \hline
 
Durán et al. \cite{duran1999requirements} & Elicitation of requirements  & \ding{55} &\ding{51} &\ding{51}  \\ \hline
Del Río et al. \cite{del2012defining} & Performance indicators  & \ding{55} &\ding{51} &\ding{51}   \\ \hline
Segura et al. \cite{segura2017template} & Metamorphic relationships & \ding{55} &\ding{55} &\ding{51}  \\ \hline

Basili \cite{Basili1994} & GQM & \ding{51} &\ding{55} &\ding{51} \\ \hline
Wieringa \cite{38631e0608b54d4299d5707f3a78debf} & DSR & \ding{55} &\ding{55} &\ding{51}  \\ \hline
This work & Templates for changes & \ding{51}  &\ding{51} &\ding{51}  \\ \hline
%\noalign{\smallskip}\hline

\end{tabular}

\end{table}


%76 words


• Improve \textless a problem context \textgreater  \newline
• by \textless (re)designing an artifact \textgreater \newline
• that satisfies \textless some requirements \textgreater \newline
• in order to \textless help stakeholders achieve some goals \textgreater.





%8. Conclusiones
\section{Conclusions and Future work}
\label{sec:conclusions}

\textcolor[rgb]{1,0,0}{Pendiente de terminar.} In this work we validate a template including L-patterns to facilitate the specification of replication changes. By systematizing the conception of changes and how to document them, it will be easier for other researchers to understand them.
%The structure of the information in the form of a template and the proposal of standard phrases facilitates the writing of changes in the the replications.

%\textcolor[rgb]{1,0,0}{Quitar Comentario Bea dice ver 2 cuestiones independientes y ¿en que consiste el principio? } In addition, an in-depth knowledge of the changes makes it easier to see whether the fundamental principle of independence based on aggregated data analysis according to Kitchenham \cite{kitchenham2008role}, which promotes the introduction of changes in replication to avoid propagating problems from the original experiment, is met.

The validation of the template in the three series of experiments has allowed us to realise that relevant information was missing as well as to improve the wording or phrases that are proposed as parameters in some of the defined L-patterns. 

Analysing the difficulties encountered in defining each change, emphasis has been placed on: i) discerning whether the changes lead to modifications in the template and/or metamodel and ii) highlighting the missing information in describing the changes so that the changes are not \textbf{under-specified}.

In summary, the main modifications required are:
i) add the fields \emph{date} and \emph{site} where the replication is carried out; ii) changes in \emph{context variables} are included in the  \emph{operationalization dimension}; and iii) the L-pattern of \emph{threat to validity} is modified in order to be able to specify that the validity is \emph{decreased} or \emph{no threat to validity is addressed}.

%\textbf{Due to limitations in the template}. Implies a change in the template (e.g. modify the L-pattern in case the threat is not identified). Some of these situations result in changes in the metamodel.

In the short term, future work will be aimed at developing a web form or a wizard so that experimenters can define the changes in a friendly way and with this information is generated, for example, the \LaTeX\ code to insert in the articles. The web form will allow the use of the template by parts of other researchers who do not know them and test its usefulness.
Further, our intention is to extend the experimental information repository EXEMPLAR \cite{ParejoExemplar2014} with aspects of replications based on metamodel and template proposed and reviewed in this paper.

\section*{Acknowledgements}
%

This section analyzes related work that: i) presents guidelines on how to report changes in SE replications; ii) presents such changes in an easy-to-interpret way, e.g., through tables; and iii) defines or uses templates or patterns.

Carver \cite{carver2010towards} presents an initial proposal for guidelines on the content of publications reporting replications. In this article, the content that should be included about the original study and about replication is indicated. It is recommended to include a section on changes to the original experiment but without specifying how to report such changes.

Apart from Carver's methodological proposal, several studies have been found that document changes in replication using \emph{ad-hoc} tables. Among them we highlight the following:

Solari \cite{solari2013identifying} refers to identifying experimental incidents that occur during replication. Examples of incidents are unexpected changes in any of the variables defined for the experiment. In the study, five replications of the same experiment carried out by different experimenters in different locations are documented. Incidents are classified by replication and incident category. With the information stored, a summary table is presented that facilitates the analysis and comparison of the replications.

The work of Lung et al. \cite{lung2008difficulty}, is the most related to our proposal. A replication of an experiment with human subjects is documented by presenting a summary table of differences and changes made to adapt the procedure to local circumstances. For each change introduced, the situation in the original experiment, the situation in replication and the reasons for making the change are clearly explained. 

In \cite{1330459}, Almqvist presents a Systematic Literature Review about Replications. It includes a template to collect the configuration of the replications and in one of its rows it jointly describes all the changes made in the replication.

In \cite{apa2014effectiveness}, Apa et al. present a replication of a failure detection experiment. The changes are described, in a section on changes in the original experiment, in textual form and in a table where for each change the situation in the original experiment and in replication is compared.

In \cite{fucci2016external}, Fucci et al. document an external replication of an experiment on evidence-based development. It contains a section of changes to the original configuration where it presents a table with the adjustments made to the base experiment.

In \cite{juristo2012comparing}, Juristo et al. document eight replications of an experiment to evaluate the effectiveness of three verification and validation techniques. A table is presented where for each change, its situation in each replication and the design decision adopted is analyzed.

In \cite{quesada2016empirical}, Quesada et al. describe a controlled experiment on function point analysis and its two replications. The differences in the experimental configuration between the experiment and replications are explained using a table and in textual form.

On the other hand, there are the studies on template definition in other areas within SE. The idea of using templates and patterns to define changes in the replication of experiments is based on the templates proposed by Durán et al. \cite{duran1999requirements} for the elicitation of requirements.  The templates and L-patterns have been successfully applied in several areas; Del Rio et al. \cite{del2012defining} have used them for the definition of business process performance indicators and Segura et al.  \cite{segura2017template} have used them for the definition of metamorphic relationships.

In the description of experiments, the Goal-Question-Metric (GQM) template by Basili \cite{Basili1994}, recommended by Wohlin \cite{wohlin:experimentation}, should be highlighted for the definition of experiment objectives. Finally, in the Design Science Research (DSR) methodology, Wieringa \cite{38631e0608b54d4299d5707f3a78debf} defines a template for the specification of a problem that will lead to future research.
% ------------------------------------------------------
% File    : Jbd-Trabajos.tex
% Content : Comparación de trabajos relacionados
% Date    : 10/Marzo/2018
% Version : 1.0
% Authors : 
% ------------------------------------------------------
% ------------------------------------------------------
% Last update: ...// 
% - Added ..
% ------------------------------------------------------


%\begin{table}[htpb]
\begin{table}
\label{tab:trabajos}
\caption{Comparison of related work}
%\label{tab:1}       % Give a unique label
%\centering
%\small
% p{5 cm} |p{5 cm} |
%\begin{tabular}{| p{1cm} | p{0.6cm} | p{1 cm}| p{1 cm} |  p{1 cm} |}  &\ding{51
%\begin{tabular}{| l | c | c | p{2cm} | c | c |}
\begin{tabular}{| l |c | c | c | c| c | }

\hline

\textbf{Related works} & \textbf{Topic}  & \textbf{ESE} & \textbf{Use template} &  \textbf{Use pattern}   \\ \hline
Carver  \cite{carver2010towards}   & Guidelines & \ding{51}  &\ding{55} &  \ding{55} \\ \hline
Solari \cite{solari2013identifying} & Reports a replication  & \ding{51} &\ding{55} &\ding{55} \\ \hline
Lung et al. \cite{lung2008difficulty} & Reports a replication  & \ding{51} &\ding{55} &\ding{55}   \\ \hline
Almqvist \cite{1330459} & SLR replications & \ding{51} &\ding{55} &\ding{55}  \\ \hline

Apa et al. \cite{apa2014effectiveness} & Reports a replication & \ding{51} &\ding{55} &\ding{55}  \\ \hline
Fucci et al.  \cite{fucci2016external} & Reports a replication & \ding{51} &\ding{55} &\ding{55}  \\ \hline
Juristo et al.  \cite{juristo2012comparing} & Reports a replication & \ding{51} &\ding{55} &\ding{55}  \\ \hline
Quesada et al.  \cite{quesada2016empirical} & Reports a replication & \ding{51} &\ding{55} &\ding{55}  \\ \hline
 
Durán et al. \cite{duran1999requirements} & Elicitation of requirements  & \ding{55} &\ding{51} &\ding{51}  \\ \hline
Del Río et al. \cite{del2012defining} & Performance indicators  & \ding{55} &\ding{51} &\ding{51}   \\ \hline
Segura et al. \cite{segura2017template} & Metamorphic relationships & \ding{55} &\ding{55} &\ding{51}  \\ \hline

Basili \cite{Basili1994} & GQM & \ding{51} &\ding{55} &\ding{51} \\ \hline
Wieringa \cite{38631e0608b54d4299d5707f3a78debf} & DSR & \ding{55} &\ding{55} &\ding{51}  \\ \hline
This work & Templates for changes & \ding{51}  &\ding{51} &\ding{51}  \\ \hline
%\noalign{\smallskip}\hline

\end{tabular}

\end{table}


%76 words



The table \ref{tab:trabajos} analyses, for each of the above-mentioned works: i) the subject matter, ii) their belonging to the Empirical SE area, iii) the use of templates to facilitate reuse and visual presentation, and iv) the use of patterns to facilitate writing.





%\section*{Appendix A. instantiation of the template}
%\label{sec:Instancia}
%
%\begin{table}[htpb]
\begin{table}
\caption{Application of the template to the replication \cite{bernardez-jss-2016}}
%\label{tab:1}       % Give a unique label
%\centering
%\small
\begin{tabular}{| p{3.3cm} | p{9cm} |}
%Cuasi Experimento 
%RQ o GQM & \{Del original ?? Yo no lo pondría\}  \\  \hline
\hline
\textbf { \textit{Mind}\textbf   { \textit{\#2} }} & Replication of experiment   \textit{Mind \#1}    \\  \hline
Empirical method &  Controlled experiment  \\  \hline
Type of replication &  Internal   \\  \hline
%Objetivo  & El objetivo de la replicación es  \textless \textit{objetivo} \textgreater  \\  \hline 
Target  &   Confirm results   \\  \hline \hline

Change \textit{\#1}   & \parbox[t]{9cm} {Originally,  \textit{for 4 weeks Mindfulness was practiced 4 days a week in 10-minute sessions.} } \parbox[t]{9cm}{In replication \textit{the sessions were 12 minutes long and for 6 weeks} }   in order to  \textit{make more evident the benefits of Mindfulness..} \\  \hline
Modified Dimension & 
Operationalization, specifically, the independent variable \textit {Training Workshop }  \\  \hline 
Validity threat addressed  &  The change increases the construct validity   \\  \hline
 \hline
Change \textit{\#2}   & \parbox[t]{9cm} {Originally,  \textit{the assignment of subjects to treatment was not randomized.} } \parbox[t]{9cm}{In replication \textit{it becomes random} }  in order to \textit{remedy threats to the internal validity of quasi-experiments.} \\  \hline
Modified Dimension & Protocol, specifically, experimental design \\  \hline 

Validity threat addressed  &  The change increases the internal validity   \\  \hline \hline

Change \textit{\#3}   & \parbox[t]{9cm} {Originally,  \textit{an public speaking workshop was given to the control group as a placebo.} } \parbox[t]{9cm}{In replication \textit{the oratory workshop took place after the experiment} } in order to  \textit{avoid a possible effect of such a workshop on the measurements of dependent variables.} \\  \hline
Modified Dimension & 
Operationalization, specifically, the independent variable \textit {Training Workshop }  \\  \hline 
Validity threat addressed  &  The change increases the construct validity   \\  \hline \hline

Change \textit{\#N}   & \parbox[t]{9cm} {Originally,  \textit{an public speaking workshop was given to the control group as a placebo.} } \parbox[t]{9cm}{In replication \textit{the oratory workshop took place after the experiment} } in order to  \textit{avoid a possible effect of such a workshop on the measurements of dependent variables.} \\  \hline
Modified Dimension & 
Operationalization, specifically, the independent variable \textit {Training Workshop }  \\  \hline 
Validity threat addressed  &  The change increases the construct validity   \\  \hline
\end{tabular}
%Firstly, the problem of conceptual modeling "Erasmus" is carried out. Secondly, the problem of conceptual modeling "EoD project" is carried out.
%\caption{Comparison of previous reviews}
\label{tab:plantilla-mind}
\end{table}
%Cambio-1 Se aumenta la cantidad de Minfulness, parece que mas que cambiar la variable independiente cambio la forma de aplicarlo, sería instrumentalización ??
% en la 2º replicacion cambia Conceptual modeling exercise order

%76 words



%% 
% Tabla
% T-Fam23-Problem
% Resumen changes Fam-2, Fam-3
%
%\begin{table*}
\begin{table}
   \caption{Problems identified in Fam\#1 when specifying changes using the template}
  \label{tab:ProblemF1}
  \centering

\begin{tabularx}{0.47\textwidth}{cccc}
    \toprule
    Family & Replication & Change & Problem identified\\
    \midrule
    

% Family-3
Fam\#1  & Mind\#2 & Ch-1 & \ding{51}  \\
~ & ~ & Ch-2 &   \ding{51} \\
~ & ~ & Ch-3 &   \ding{51} \\

\hline
~ & Mind\#3 & Ch-1 & \ding{51}  \\

    \bottomrule
  \end{tabularx}
%\end{table*}
\end{table}
% ------------------------------------------------------
% File    : T-ApliORT.tex
% Content : Problems and solutions
% Date    : 1/12/2018
% Version : 1.0
% Authors : M.Cruz
% ------------------------------------------------------
\begin{table*}[h]
%\begin{table}[h]
  %\renewcommand{\arraystretch}{1.45}
  \caption{Instantiation of the proposed template in Mind\#2}
  
\label{tab:plantEng}
  \centering
%	\scriptsize

\begin{tabularx}{\textwidth}{
  >{\hsize=0.3\hsize}X
 %>{\hsize=0.4\hsize}X 2Cols
  >{\hsize=0.8\hsize}X}
  
    \noalign{\smallskip}\hline\noalign{\smallskip}
  
  Field &  Value  \\ 
  \noalign{\smallskip}\hline\noalign{\smallskip}
  
 \textbf {\textit{Mind\#2}} & Replication of experiment   \textit{Mind\#1}    \\
%Empirical method &  Controlled experiment  \\  
Type of replication &  Internal   \\ 
Purpose  &   Confirm results   \\  
%Site & University of Seville \\  
 \hline
%%%%   
    Change \textit{1}   & \textbf{Originally}, for 4 weeks Mindfulness was practiced 4 days a week in 10-minute sessions \\& \textbf{In replication} In replication the sessions were 12 minutes long and for 6 weeks \\& \textbf{in order to}  make more evident the benefits of Mindfulness \\
    
    Modified Dimension & 
    \textbf{Operationalization}, specifically the independent variable  \textit {Training Workshop} \\   
    Threat to validity & The change increases the construct validity  \\  \hline
    %Threat to validity & The change not affect the validity   \\& \textbf{because} only the dose of the treatment applied changes \\\hline
  
   Change \textit{\#2}   &  \textbf{Originally}, the assignment of subjects to treatment was not randomized 
   \\& \textbf{In replication} it becomes random  \\&
    \textbf{in order to} remedy threats to the internal validity of quasi-experiments\\
  
Modified Dimension & \textbf{Protocol}, specifically experimental design \\  
Threat to validity  &  The change increases the internal validity    \\
%& \textbf{because} the results are independent of the protocol \\ 
\hline


 Change \textit{\#3}   &  \textbf{Originally}, an public speaking workshop was given to the control group as a placebo
   \\& \textbf{In replication} the oratory workshop took place after the experiment  \\&
    \textbf{in order to} avoid a possible effect of such a workshop on the measurements of dependent variables\\
  
Modified Dimension & \textbf{Operationalization}, specifically the independent variable  \textit {Training Workshop} \\   
    %Threat to validity & The change not affect the validity   \\%& 
    %\textbf{because} only the levels of  independent variable  \textit {Training Workshop} changes \\
    Threat to validity & The change increases the construct validity   \\
%%%%
   
	\noalign{\smallskip\smallskip}\hline
	\end{tabularx}  
%\end{table}
\end{table*}

% ------------------------------------------------------
% File    : T-ApliORT.tex
% Content : Problems and solutions
% Date    : 1/12/2018
% Version : 1.0
% Authors : M.Cruz
% ------------------------------------------------------
\begin{table*}[h]
%\begin{table}[h]
  %\renewcommand{\arraystretch}{1.45}
  \caption{Instantiation of the proposed template in Mind\#3}
  
\label{tab:plantEng}
  \centering
%	\scriptsize

\begin{tabularx}{\textwidth}{
  >{\hsize=0.3\hsize}X
 %>{\hsize=0.4\hsize}X 2Cols
  >{\hsize=0.8\hsize}X}
  
    \noalign{\smallskip}\hline\noalign{\smallskip}
  
  Field &  Value  \\ 
  \noalign{\smallskip}\hline\noalign{\smallskip}
  
 \textbf {\textit{Mind\#3}} & Replication of experiment   \textit{Mind\#2}    \\
%Empirical method &  Controlled experiment  \\  
Type of replication &  Internal   \\ 
Purpose  &   Confirm results   \\  
%Site & University of Seville \\  
 \hline
%%%%   
    Change \textit{1}   & \textbf{Originally}, students make two exercises of concepual modeling, one before and one after treatment.  \\& \textbf{In replication} the order of the exercises is swapped \\& \textbf{in order to} demonstrate that it does not affect the results \\
    
    Modified Dimension & 
    \textbf{Protocol}, specifically the guides \\   
    Threat to validity  &  The change increases the internal validity    \\
    %& \textbf{because} the results are independent of the protocol \\
   \noalign{\smallskip\smallskip}\hline
	\end{tabularx}  
%\end{table}
\end{table*}

%\input{Figuras/ORT.tex}
% ------------------------------------------------------
% File    : Q-2009.tex
% Content : Q-2009
% Date    : 1/12/2018
% Version : 1.0
% Authors : M.Cruz
% ------------------------------------------------------
\begin{table*}[h]
%\begin{table}[h]
  %\renewcommand{\arraystretch}{1.45}
  \caption{Instantiation of the proposed template in Q-2009}
\label{tab:plantEng}
  \centering
%	\scriptsize

\begin{tabularx}{\textwidth}{
  >{\hsize=0.3\hsize}X
 %>{\hsize=0.4\hsize}X 2Cols
  >{\hsize=0.8\hsize}X}
  
    \noalign{\smallskip}\hline\noalign{\smallskip}
  
  Field &  Value  \\ 
  \noalign{\smallskip}\hline\noalign{\smallskip}
  
\textbf {\textit{Q-2009}} &  Replication of experiment \textit{Q-2007 }    \\
%Empirical method & Quasi-experiment\\ 
Type of replication &  Internal   \\ 
Purpose  &  Confirm results  \\   
%Site & Polytechnic Universityof Madrid \\
%Date & 2009\\ 
\hline 

%%%%   
    Change \textit{1}   & \textbf{Originally}, analysts' effectiveness in interview sessions is analysed \\& \textbf{In replication} effectiveness is not analysed  \\ & \textbf{because of} the high cost of transcribing and analyzing all interviews \\
    
    Modified Dimension & 
    \textbf{Operationalization}, specifically the dependent variable  \textit {effectiveness} \\   
    Threat to validity & The change ***(\textbf{\textit{Threat}})
    %The change decreases the construct validity
    \\
    % &     \textbf{because} a dependent variable is suppressed \\ 
    \hline

%%%%
    Change \textit{2}   & \textbf{Originally}, the retention capacity is analyzed \\& \textbf{In replication} retention capacity is not analysed  \\& \textbf{because of} the high cost of transcribing and analyzing all interviews  \\  

    Modified Dimension & 
   \textbf{Operationalization}, specifically the dependent variable  \textit {retention capacity} \\ 
    Threat to validity  & The change ***(\textbf{\textit{Threat}}) \\ \hline
%%%%
 
    Change \textit{3}   & \textbf{Originally}, no account is taken of development experience \\& \textbf{In replication} experience in development is considered to calculate the independent variable experience  \\  
    & \textbf{because of} ***(\textbf{\textit{Reason}})\\

    Modified Dimension & 
    \textbf{Operationalization}, specifically the dependent variable  \textit {development experience} \\   
     Threat to validity  & The change increases the construct validity \\ \hline

%%%%
    Change \textit{4}   & \textbf{Originally}, interviews are conducted in Spanish \\& \textbf{In replication} interviews are conducted in English \\& \textbf{because of} English was a requirement of the master to which the students belonged \\ 

    Modified Dimension & 
    \textbf{Protocol}, specifically experimental material \\   
    Threat to validity  & The change increases the internal validity \\  \hline
    
     Change \textit{5}   & \textbf{Originally}, a person responds in interviews \\& \textbf{In replication} the person who answers the interviews is changed  \\ 
    & \textbf{because of} ***(\textbf{\textit{Reason}})\\

    Modified Dimension & 
    \textbf{Stakeholder}, specifically the \textit {monitor} \\   
     Threat to validity  & The change ***(\textbf{\textit{Threat}}) \\
   \noalign{\smallskip\smallskip}\hline
	
	\end{tabularx}  
	
%\end{table}
\end{table*}

%
%\begin{table}[htpb]
\begin{table}
\caption{Replicación Q-2009}
%\label{tab:1}       % Give a unique label
%\centering
%\small
\begin{tabular}{| p{3.3cm} | p{9cm} |}
\hline

%\textbf { \textit{Mind}\textbf   { \textit{\#1} }} & Replicación del experimento   \textit{Mind \#0}    \\  \hline
\textbf {\textit{Q-2009}} & Replicación del experimento \textit{Q-2007 }    \\  \hline

Método empírico &  Cuasi experimento   \\  \hline
Tipo &  Interna   \\  \hline
%Objetivo  & El objetivo de la replicación es  \textless \textit{objetivo} \textgreater  \\  \hline 
Objetivo  &   El objetivo de la replicación es confirmar resultados   \\  \hline \hline

Cambio- \textit{1}   & \parbox[t]{9cm} {Originalmente,  \textit{se analiza la efectividad de los analistas en las sesiones de educción} } \parbox[t]{9cm}{en la replicación \textit{no se analiza} }  debido a  \textit{el alto costo de transcribir y analizar todas las entrevistas } \\  \hline
Dimensión modificada & 
 Operacionalización  en concreto, la variable  dependiente \textit { efectividad en la educción }  \\  \hline 
Amenaza a la validez abordada  &     \\  \hline
 \hline
Cambio- \textit{2}   & \parbox[t]{9cm} {Originalmente,  \textit{se analiza la capacidad de retención
} } \parbox[t]{9cm}{en la replicación \textit{no se analiza} }  debido a   \textit{el alto costo de transcribir y analizar todas las entrevistas} \\  \hline
Dimensión modificada & 
Operacionalización  en concreto, la variable  dependiente \textit { capacidad de retención }  \\  \hline 
Amenaza a la validez abordada  &    \\  \hline \hline

Cambio- \textit{3}   & \parbox[t]{9cm} {Originalmente,  \textit{ No se tiene en cuenta la experiencia en desarrollo } } \parbox[t]{9cm}{en la replicación \textit{Se considera la experiencia en desarrollo para calcular la variable independiente experiencia} }  con el fin de  \textit{ } \\  \hline
Dimensión modificada & 
  Operacionalización  en concreto, la variable  independiente \textit {experiencia}  \\  \hline 
Amenaza a la validez abordada  &    \\  \hline \hline 

Cambio- \textit{4}   & \parbox[t]{9cm} {Originalmente,  \textit{ Las entrevistas se hacen en español } } \parbox[t]{9cm}{en la replicación \textit{ Las entrevistas se hacen en idioma inglés} }  debido a que \textit{el inglés era un requisito del master al que pertenecían los sujetos } \\  \hline
Dimensión modificada & 
  Protocolo en concreto, material experimental    \\  \hline 
Amenaza a la validez abordada  & El cambio incrementa la validez  interna   \\  \hline \hline 

Cambio- \textit{5}   & \parbox[t]{9cm} {Originalmente,  \textit{ una persona responde en las entrevistas } } \parbox[t]{9cm}{en la replicación \textit{ se cambia la persona encargada de responder las entrevistas} }  con el fin de  \textit{ } \\  \hline
Dimensión modificada & 
  Experimentador en concreto, el monitor    \\  \hline 
Amenaza a la validez abordada  &    \\  \hline

\end{tabular}
%\caption{Comparison of previous reviews}
\label{tab:Ale-2009}
\end{table}
%Cambio-1 Se aumenta la cantidad de Minfulness, parece que mas que cambiar la variable independiente cambio la forma de aplicarlo, sería instrumentalización ??

%76 words



%
%\begin{table}[htpb]
\begin{table}
\caption{Replicación Q-2011}
%\label{tab:1}       % Give a unique label
%\centering
%\small
\begin{tabular}{| p{3.3cm} | p{9cm} |}
\hline

%\textbf { \textit{Mind}\textbf   { \textit{\#1} }} & Replicación del experimento   \textit{Mind \#0}    \\  \hline
\textbf {\textit{Q-2011}} & Replicación del experimento \textit{Q-2009 }    \\  \hline

Método empírico &  Cuasi experimento   \\  \hline
Tipo &  Interna   \\  \hline
%Objetivo  & El objetivo de la replicación es  \textless \textit{objetivo} \textgreater  \\  \hline 
Objetivo  &      \\  \hline \hline

Cambio- \textit{1}   & \parbox[t]{9cm} {Originalmente,  \textit{las entrevistas son individuales} } \parbox[t]{9cm}{en la replicación \textit{ entrevistas en grupo entre analistas y clientes} }  debido a  \textit{al costo y esfuerzo que implicaría realizar 16 entrevistas individuales y la influencia que podría ejercer el cansancio del experimentador en la efectividad de los sujetos } \\  \hline
Dimensión modificada & 
 Protocolo guías    \\  \hline 
Amenaza a la validez abordada  & El cambio incrementa la validez  interna   \\  \hline
 \hline
Cambio- \textit{2}   & \parbox[t]{9cm} {Originalmente,  \textit{se considera la experiencia en educción} } \parbox[t]{9cm}{en la replicación \textit{la experiencia se operacionalizará en función de los años de experiencia y la habilidad que el propio sujeto cree tener} }  debido a   \textit{ } \\  \hline
Dimensión modificada & 
Operacionalización  en concreto, las variables independientes \textit { Habilidad en requisitos } y\textit { Habilidad en entrevistas }  \\  \hline 
Amenaza a la validez abordada  &    \\  \hline \hline

Cambio- \textit{3}   & \parbox[t]{9cm} {Originalmente,  \textit{el tiempo de educción, es decir la duración de las entrevistas es de 30min.} } \parbox[t]{9cm}{en la replicación \textit{el tiempo de educción es de 60min.} } debido a que \textit{la entrevista es grupal } \\  \hline
Dimensión modificada & 
 Protocolo en concreto, las guías     \\  \hline 
Amenaza a la validez abordada  & El cambio incrementa la validez  interna  \\  \hline \hline

Cambio- \textit{4}   & \parbox[t]{9cm} {Originalmente,  \textit{el tiempo transcurrido entre la sesión de educción y la consolidación es decir, cuando el sujeto presenta por escrito la información recopilada es de 7 días} } \parbox[t]{9cm}{en la replicación \textit{la consolidación es inmediata a la educción} } con el fin de \textit{evitar perdida de información por olvido } \\  \hline
Dimensión modificada & 
 Protocolo en concreto, las guías     \\  \hline 
Amenaza a la validez abordada  & El cambio incrementa la validez  interna  \\  \hline \hline

Cambio- \textit{5}   & \parbox[t]{9cm} {Originalmente,  \textit{el tiempo de consolidación, es decir, el tiempo disponible para que el analista presente por escrito la información adquirida en las sesiones de educción no se media} } \parbox[t]{9cm}{en la replicación \textit{el tiempo de consolidación es de 120min. } } debido a que \textit{la consolidación es inmediata a la educción} \\  \hline
Dimensión modificada & 
 Protocolo en concreto, las guías     \\  \hline 
Amenaza a la validez abordada  & El cambio incrementa la validez  interna  \\  \hline \hline

Cambio- \textit{6}   & \parbox[t]{9cm} {Originalmente,  \textit{ una persona responde en las entrevistas } } \parbox[t]{9cm}{en la replicación \textit{ se cambia la persona encargada de responder las entrevistas} } debido a  \textit{} \\  \hline
Dimensión modificada & 
 Experimentador Monitor    \\  \hline 
Amenaza a la validez abordada  &    \\  \hline

\end{tabular}
%\caption{Comparison of previous reviews}
\label{tab:Ale-2011}
\end{table}
%Cambio-1 Se aumenta la cantidad de Minfulness, parece que mas que cambiar la variable independiente cambio la forma de aplicarlo, sería instrumentalización ??

%76 words



% ------------------------------------------------------
% File    : Q-2011.tex
% Content : Q-2011
% Date    : 1/12/2018
% Version : 1.0
% Authors : M.Cruz
% ------------------------------------------------------
\begin{table*}[h]
%\begin{table}[h]
  %\renewcommand{\arraystretch}{1.45}
  \caption{Instantiation of the proposed template in Q-2011}
\label{tab:plantEng}
  \centering
%	\scriptsize

\begin{tabularx}{\textwidth}{
  >{\hsize=0.3\hsize}X
 %>{\hsize=0.4\hsize}X 2Cols
  >{\hsize=0.8\hsize}X}
  
    \noalign{\smallskip}\hline\noalign{\smallskip}
  
  Field &  Value  \\ 
  \noalign{\smallskip}\hline\noalign{\smallskip}
  
\textbf {\textit{Q-2011}} &  Replication of experiment \textit{Q-2009 }    \\
%Empirical method & Quasi-experiment\\ 
Type of replication &  Internal   \\  
Purpose & Extend results   \\  
\hline
%Site & Polytechnic University of Madrid \\
%Date & 2009\\ 

%%%%   
    Change \textit{1}   & \textbf{Originally}, interviews between subjects (analysts) and experimenter are individual \\& \textbf{In replication} interviews are in groups   \\& \textbf{because of} the cost and effort involved in conducting individual interviews and the experimenter's fatigue \\
    
    Modified Dimension & 
    \textbf{Protocol}, specifically the guides \\   
    Threat to validity & The change increases the internal validity  \\  \hline
  
%%%%
    Change \textit{2}   & \textbf{Originally}, experience in requirements analysis is considered \\& \textbf{In replication} experience is determined by years of experience and the skill the subject claims to have  \\
    & \textbf{because of} ***(\textbf{\textit{Reason}})\\

    Modified Dimension & 
   \textbf{Operationalization}, specifically the independent variable  \textit {skill in requirements} and \textit {skill in interviews} \\ 
     Threat to validity  & The change ***(\textbf{\textit{Threat}}) \\ \hline
    
%%%%
 
    Change \textit{3}   & \textbf{Originally}, the duration of the interviews is 30 min. \\& \textbf{In replication} the duration of the interviews is 60 min  \\.  
    & \textbf{because of} the interview is in group\\

    Modified Dimension & 
    \textbf{Protocol}, specifically the guides \\   
    Threat to validity & The change increases the internal validity  \\  \hline

%%%%
    Change \textit{4}   & \textbf{Originally}, The subject (analyst) has 7 days to present in writing the information gathered in the interview. \\& \textbf{In replication} the written presentation is immediately after the interview. \\& \textbf{in order to} avoid loss of information \\ 

    Modified Dimension & 
    \textbf{Protocol}, specifically the guides \\   
    Threat to validity & The change increases the internal validity  \\  \hline
    
     Change \textit{5}   & \textbf{Originally}, the time elapsed between the interview and the written presentation of the information collected is not measured \\& \textbf{In replication} the time elapsed between the interview and the written presentation of the information is set at 120 min.  \\ 
     & \textbf{because of} the written presentation is immediately after the interview\\
     
    Modified Dimension & 
    \textbf{Protocol}, specifically the guides \\   
    Threat to validity & The change increases the internal validity  \\  \hline
    
    Change \textit{6}   & \textbf{Originally}, a person responds in interviews \\& \textbf{In replication} the person who answers the interviews is changed \\ 
      & \textbf{because of} ***(\textbf{\textit{Reason}})\\
     
     Modified Dimension & 
    \textbf{Stakeholder}, specifically the \textit {monitor} \\   
   Threat to validity  & The change ***(\textbf{\textit{Threat}}) \\
 \noalign{\smallskip\smallskip}\hline
	
	\end{tabularx}  
	
%\end{table}
\end{table*}

% ------------------------------------------------------
% File    : Q-2012.tex
% Content : Q-2012
% Date    : 1/12/2018
% Version : 1.0
% Authors : M.Cruz
% ------------------------------------------------------
\begin{table*}[h]
%\begin{table}[h]
  %\renewcommand{\arraystretch}{1.45}
  \caption{Instantiation of the proposed template in Q-2012}
\label{tab:plantEng}
  \centering
%	\scriptsize

\begin{tabularx}{\textwidth}{
  >{\hsize=0.3\hsize}X
 %>{\hsize=0.4\hsize}X 2Cols
  >{\hsize=0.8\hsize}X}
  
    \noalign{\smallskip}\hline\noalign{\smallskip}
  
  Field &  Value  \\ 
  \noalign{\smallskip}\hline\noalign{\smallskip}
  
\textbf {\textit{Q-2012}} &  Replication of experiment \textit{Q-2011 }    \\
%Empirical method & Quasi-experiment\\ 
Type of replication &  External   \\  
Purpose  & Confirm results \\   \hline
%Site & Polytechnic University of Madrid \\
%Date & 2009\\ 

%%%%   
    Change \textit{1}   & \textbf{Originally}, the subjects are Master's students \\& \textbf{In replication} the subjects are professionals   \\& \textbf{because of} replication is performed at the International Working Conference on Requirements Engineering \\
    
    Modified Dimension & 
    \textbf{Population}, specifically the experimental subjects \\   
    Threat to validity & The change increases the external validity  \\  \hline
  
%%%%
    Change \textit{2}   & \textbf{Originally}, subjects have little or no development experience \\& \textbf{In replication} the subjects are professionals with experience in development  \\& \textbf{because of} replication is performed at the International Working Conference on Requirements Engineering \\ 

    Modified Dimension & 
   \textbf{Operationalization}, specifically the independent variable \textit {development skill} \\ 
    Threat to validity  & The change increases the construct validity \\ \hline
    
%%%%
 
     Change \textit{3}   & \textbf{Originally}, the duration of the interviews is 120 min. \\& \textbf{In replication} the duration of the interviews is 30 min  \\.  
    & \textbf{because of} time constraints \\ 

    Modified Dimension & 
    \textbf{Protocol}, specifically the guides \\   
    Threat to validity & The change increases the internal validity  \\  \hline

%%%%
    Change \textit{4}   & \textbf{Originally}, the experiment is carried out at the end of the course, i.e. after the training period \\& \textbf{In replication} no training period \\& \textbf{because of} replication is performed at the International Working Conference on Requirements Engineering \\ 

    Modified Dimension & 
    \textbf{Protocol}, specifically the guides \\   
     Threat to validity  & The change increases the internal validity \\ \hline
    
	\end{tabularx}  
	
%\end{table}
\end{table*}
% ------------------------------------------------------
% File    : Q-2012A.tex
% Content : Q-2012A
% Date    : 1/12/2018
% Version : 1.0
% Authors : M.Cruz
% ------------------------------------------------------
\begin{table*}[h]
%\begin{table}[h]
  %\renewcommand{\arraystretch}{1.45}
  \caption{Instantiation of the proposed template in E-2012A}
\label{tab:plantEng}
  \centering
%	\scriptsize

\begin{tabularx}{\textwidth}{
  >{\hsize=0.3\hsize}X
 %>{\hsize=0.4\hsize}X 2Cols
  >{\hsize=0.8\hsize}X}
  
    \noalign{\smallskip}\hline\noalign{\smallskip}
  
  Field &  Value  \\ 
  \noalign{\smallskip}\hline\noalign{\smallskip}
  
\textbf {\textit{E-2012A}} &  Replication of experiment \textit{Q-2012 }    \\
%Empirical method & Experiment\\ 
Type of replication &  Internal   \\  
Purpose  & Extend results \\   \hline
%Site & Polytechnic University of Madrid \\
%Date & 2009\\ 

%%%%   
    Change \textit{1}   & \textbf{Originally}, knowledge is defined as familiarity through subjective assessment\\& \textbf{In replication} knowledge is defined as an independent variable with two levels: known and unknown problem \\& \textbf{because of} in the experimental population (post-graduate students) it is possible to know whether or not they know a certain domain of the problem \\
    
    Modified Dimension & 
   \textbf{Operationalization}, specifically the independent variable \textit {knowledge} \\ 
    Threat to validity  & The change ***(\textbf{\textit{Threat}}) \\ \hline
  
%%%%
    Change \textit{2}   & \textbf{Originally}, the interviews to know the requirements are carried out on two different days, to avoid fatigue in the experimenter \\& \textbf{In replication} the design is changed to a design of repeated measurements (within-subjects) \\& \textbf{because of} this design does not require a large number of subjects \\ 

    Modified Dimension & 
   \textbf{Protocol}, specifically the experimental design  \\ 
    Threat to validity & The change increases the internal validity \\  \hline
    
%%%%
 
     Change \textit{3}   & \textbf{Originally}, interviews between subjects (analysts) and experimenters are in groups \\& \textbf{In replication} interviews are individual  \\.  
    & \textbf{because of} there are two experimenters (responders) with two languages \\ 

    Modified Dimension & 
    \textbf{Protocol}, specifically the guides \\   
    Threat to validity & The change increases the internal validity  \\  \hline

%%%%
    Change \textit{4}   & \textbf{Originally}, there are no blocking variables \\& \textbf{In replication} there is a blocking variable per language \\& \textbf{because of} subjects who use their mother tongue will be more effective than subjects who use a second language  \\ 

     Modified Dimension & 
   \textbf{Protocol}, specifically the experimental design  \\ 
    Threat to validity & The change increases the internal validity \\  \hline
    
    %%%%
    Change \textit{5}   & \textbf{Originally}, there are no blocking variables \\& \textbf{In replication} there is one blocking variable per experimenter (respondent) \\& \textbf{because} experimental subjects conduct the interview in their own language.  \\ 

     Modified Dimension & 
   \textbf{Protocol}, specifically the experimental design  \\ 
    Threat to validity & The change increases the internal validity \\  \hline
    
     %%%%
    Change \textit{6}   & \textbf{Originally}, there is a experimenter (respondent) \\& \textbf{In replication} there are two experimenters (respondents) \\& \textbf{In order to} alleviate the effects of fatigue and learning of the experimenter (respondents)  \\ 

     Modified Dimension & 
    \textbf{Protocol}, specifically the guides \\   
    Threat to validity & The change increases the internal validity  \\  \hline
    
	\end{tabularx}  
	
%\end{table}
\end{table*}
% ------------------------------------------------------
% File    : Q-2012A.tex
% Content : Q-2012A
% Date    : 1/12/2018
% Version : 1.0
% Authors : M.Cruz
% ------------------------------------------------------
\begin{table*}[h]
%\begin{table}[h]
  %\renewcommand{\arraystretch}{1.45}
  %\caption{Instantiation of the proposed template in E-2012A}
\label{tab:plantEng}
  \centering
%	\scriptsize

\begin{tabularx}{\textwidth}{
  >{\hsize=0.3\hsize}X
 %>{\hsize=0.4\hsize}X 2Cols
  >{\hsize=0.8\hsize}X}
  
    \noalign{\smallskip}\hline\noalign{\smallskip}
  
  Field &  Value  \\ 
  \noalign{\smallskip}\hline\noalign{\smallskip}
  
%\textbf {\textit{E-2012A}} &  Replication of experiment \textit{Q-2012 }    \\
%Empirical method & Experiment\\ 
%Type of replication &  Internal   \\  
%Purpose  & Extend results \\   \hline
%Site & Polytechnic University of Madrid \\
%Date & 2009\\ 

%%%%   
    Change \textit{7}   & \textbf{Originally}, there is the same problem (experimental object) for all subjects  \\& \textbf{In replication} there are two problems \\& \textbf{because of} groups are made due to blocking variables \\
    
     Modified Dimension & 
   \textbf{Protocol}, specifically the experimental material  \\ 
    Threat to validity & The change increases the internal validity \\  \hline
  
%%%%
    Change \textit{8}   & \textbf{Originally}, the duration of the interviews is 60 min. \\& \textbf{In replication} the duration of the interviews is 30 min. \\& \textbf{because} the interview is individual \\ 

     Modified Dimension & 
    \textbf{Protocol}, specifically the guides \\ 
    Threat to validity & The change increases the internal validity  \\  \hline
    
%%%%
 
     Change \textit{9}   & \textbf{Originally}, the  time  elapsed  between  the  interview  and the  written  presentation is  30 min. \\& \textbf{In replication} the  time  elapsed  between  the  interview  and the  written  presentation is  90 min. \\.  
    & \textbf{because} the recommended duration of 90 minutes \\ 

    Modified Dimension & 
    \textbf{Protocol}, specifically the guides \\ 
    Threat to validity & The change increases the internal validity  \\  \hline

%%%%
    Change \textit{10}   & \textbf{Originally}, the difficulty of the problem is not measured \\& \textbf{In replication} the difficulty variable indicates the difficulty of the problem \\& \textbf{because} there are two problems  \\ 

    Modified Dimension & 
   \textbf{Operationalization}, specifically the independent variable \textit {difficulty} \\ 
     Threat to validity  & The change increases the construct validity \\ 
   \noalign{\smallskip\smallskip}\hline
    
	\end{tabularx}  
	
%\end{table}
\end{table*}
%
%\begin{table}[htpb]
\begin{table}
\caption{Replicación Q-2012}
%\label{tab:1}       % Give a unique label
%\centering
%\small
\begin{tabular}{| p{3.3cm} | p{9cm} |}
\hline

%\textbf { \textit{Mind}\textbf   { \textit{\#1} }} & Replicación del experimento   \textit{Mind \#0}    \\  \hline
\textbf {\textit{Q-2012}} & Replicación del experimento \textit{Q-2011 }    \\  \hline

Método empírico &  Cuasi experimento   \\  \hline
Tipo &  Externa   \\  \hline
%Objetivo  & El objetivo de la replicación es  \textless \textit{objetivo} \textgreater  \\  \hline 
Objetivo  &   Aumentar serie histórica de experimentos
   \\  \hline \hline

Cambio- \textit{1}   & \parbox[t]{9cm} {Originalmente,  \textit{los sujetos son estudiantes de master} } \parbox[t]{9cm}{en la replicación \textit{ los sujetos son profesionales} }  debido a que \textit{se realiza en el International Working Conference on Requirements Engineering } \\  \hline
Dimensión modificada & 
 Población en concreto, los sujetos experimentales   \\  \hline 
Amenaza a la validez abordada  & El cambio incrementa la validez externa  \\  \hline
 \hline
 
 Cambio- \textit{2}   & \parbox[t]{9cm} {Originalmente,  \textit{los sujetos no tienen o tiene poca experiencia en desarrollos } } \parbox[t]{9cm}{en la replicación \textit{ los sujetos son profesionales con experiencia en desarrollos} }  debido a que \textit{se realiza en el International Working Conference on Requirements Engineering } \\  \hline
Dimensión modificada & 
 Operacionalización en concreto,  la variable independiente habilidad en desarrollo  \\  \hline 
Amenaza a la validez abordada  &   \\  \hline
% Es nueva variable
 \hline

Cambio- \textit{3}   & \parbox[t]{9cm} {Originalmente,  \textit{el tiempo de consolidación es de 120min.} } \parbox[t]{9cm}{en la replicación \textit{el tiempo de consolidación es de 30min. } } debido a que \textit{se realiza en el International Working Conference on Requirements Engineering} \\  \hline
Dimensión modificada & 
 Protocolo en concreto, las guías    \\  \hline 
Amenaza a la validez abordada  & El cambio incrementa la validez  interna  \\  \hline \hline

Cambio- \textit{4}   & \parbox[t]{9cm} {Originalmente,  \textit{ el experimento se realiza al final del curso, es decir, del periodo de formación } } \parbox[t]{9cm}{en la replicación \textit{ no hay periodo de formación} } debido a  \textit{e realiza en el International Working Conference on Requirements Engineering} \\  \hline
Dimensión modificada & 
  Operacionalización    \\  \hline 
Amenaza a la validez abordada  &   \\  \hline

\end{tabular}
%\caption{Comparison of previous reviews}
\label{tab:plantilla}
\end{table}
%Cambio-1 Se aumenta la cantidad de Minfulness, parece que mas que cambiar la variable independiente cambio la forma de aplicarlo, sería instrumentalización ??

%76 words



%
%\begin{table}[htpb]
\begin{table}
\caption{Replicación E-2012A}
%\label{tab:1}       % Give a unique label
%\centering
%\small
\begin{tabular}{| p{3.3cm} | p{9cm} |}
\hline

%\textbf { \textit{Mind}\textbf   { \textit{\#1} }} & Replicación del experimento   \textit{Mind \#0}    \\  \hline
\textbf {\textit{E-2012A}} & Replicación del experimento \textit{Q-2012 }    \\  \hline

Método empírico &  Experimento   \\  \hline
Tipo &  Interna   \\  \hline
%Objetivo  & El objetivo de la replicación es  \textless \textit{objetivo} \textgreater  \\  \hline 
Objetivo  &  Estudiar el efecto del conocimiento que poseen los analistas acerca del dominio del problema, en el proceso de educción.   \\  \hline \hline

Cambio- \textit{1}   & \parbox[t]{9cm} {Originalmente,  \textit{el conocimiento se operacionaliza como familiaridad mediante una valoración subjetiva} } \parbox[t]{9cm}{en la replicación \textit{ el conocimiento se operacionaliza  como variable independiente con dos niveles: problema conocido y desconocido} }  debido a que \textit{en la población experimental (estudiantes de post-grado) se puede saber si conocen o no un determinado dominio del problema } \\  \hline
Dimensión modificada & 
 Operacionalización en concreto, la variable independiente conocimiento  \\  \hline 
Amenaza a la validez abordada  &   \\  \hline
 \hline
 
Cambio- \textit{2}   & \parbox[t]{9cm} {Originalmente,  \textit{es un cuasi experimento en el que el proceso de educción de requisitos se realiza en dos días distintos, para evitar efectos de cansancio por parte del experimentador} } \parbox[t]{9cm}{en la replicación \textit{ se cambia a un diseño de medidas repetidas (within-subjects)} }  debido a \textit{problemas de poder estadístico ya que con este diseño no se necesita un número elevado de sujetos } \\  \hline
Dimensión modificada & 
 Protocolo en concreto, el diseño experimental  \\  \hline 
Amenaza a la validez abordada  & El cambio incrementa la validez interna  \\  \hline
 \hline
 
 Cambio- \textit{3}   & \parbox[t]{9cm} {Originalmente,  \textit{entrevistas en grupo entre analistas y clientes  } } \parbox[t]{9cm}{en la replicación \textit{las entrevistas son individuales} }  debido a  \textit{que se ha cambiado el diseño y hay dos analistas (respondedores) con dos idiomas } \\  \hline
Dimensión modificada & 
 Protocolo en concreto, las guías     \\  \hline 
Amenaza a la validez abordada  & El cambio incrementa la validez  interna   \\  \hline
 \hline
 
  Cambio- \textit{4}   & \parbox[t]{9cm} {Originalmente,  \textit{no hay variables de bloqueo} } \parbox[t]{9cm}{en la replicación \textit{ hay variable de bloqueo por idioma} }  debido a \textit{que es esperable que los sujetos que utilicen su lengua materna sean más efectivos que los sujetos que utilizan una segunda lengua } \\  \hline
Dimensión modificada & 
 Protocolo en concreto, el diseño experimental  \\  \hline 
Amenaza a la validez abordada  & El cambio incrementa la validez interna  \\  \hline
 \hline
 
Cambio- \textit{5}   & \parbox[t]{9cm} {Originalmente,  \textit{no hay variables de bloqueo} } \parbox[t]{9cm}{en la replicación \textit{ hay variable de bloqueo por respondedor} }  debido a \textit{Al bloquear los sujetos por respondedor, se evita que haya interacciones con la variable de bloqueo idioma. Asi los sujetos experimentales realizan la educción en su lengua materna } \\  \hline
Dimensión modificada & 
 Protocolo en concreto, el diseño experimental  \\  \hline 
Amenaza a la validez abordada  & El cambio incrementa la validez interna  \\  \hline
 \hline

 Cambio- \textit{6}   & \parbox[t]{9cm} {Originalmente,  \textit{hay un respondedor} } \parbox[t]{9cm}{en la replicación \textit{ el número de respondedores se ha establecido en dos} }  debido a que \textit{para paliar los efectos de cansancio y aprendizaje del respondedor } \\  \hline
Dimensión modificada & 
 Protocolo en concreto, guías   \\  \hline 
Amenaza a la validez abordada  & El cambio incrementa la validez interna  \\  \hline
 
 
 
%Cambio- \textit{7}   & \parbox[t]{9cm} {Originalmente,  \textit{hay un mismo problema (objeto experimental) para todos los sujetos } \parbox[t]{9cm}{en la replicación \textit{hay dos problemas} }  debido a que \textit{ se hacen grupos debido a las variables de bloqueo } \\  \hline
%Dimensión modificada & 
 %Instrumentalización en concreto, el diseño experimental  \\  \hline 
%Amenaza a la validez abordada  & El cambio incrementa la validez interna  \\  \hline
 %\hline \hline \hline

\end{tabular}
%\caption{Comparison of previous reviews}
\label{tab:plantilla}
\end{table}


%Cambio-1 Se aumenta la cantidad de Minfulness, parece que mas que cambiar la variable independiente cambio la forma de aplicarlo, sería instrumentalización ??

%76 words



%
%\begin{table}[htpb]
\begin{table}
%\caption{Replicación Q-2012A}
%\label{tab:1}       % Give a unique label
%\centering
%\small
\begin{tabular}{| p{3.3cm} | p{9cm} |}
\hline

%\textbf { \textit{Mind}\textbf   { \textit{\#1} }} & Replicación del experimento   \textit{Mind \#0}    \\  \hline
 
Cambio- \textit{7}   & \parbox[t]{9cm} {Originalmente,  \textit{hay un mismo problema (objeto experimental) para todos los sujetos }} \parbox[t]{9cm}{en la replicación \textit{hay dos problemas} }  debido a que \textit{ se hacen grupos debido a las variables de bloqueo } \\  \hline
Dimensión modificada & 
Protocolo en concreto, el material experimental  \\  \hline 
Amenaza a la validez abordada  & El cambio incrementa la validez interna  \\  \hline
 \hline 
 
 Cambio- \textit{8}   & \parbox[t]{9cm} {Originalmente,  \textit{el tiempo de educción, es decir la duración de las entrevistas es de 60min.} } \parbox[t]{9cm}{en la replicación \textit{el tiempo de educción es de 30min.} } debido a que \textit{la entrevista es individual } \\  \hline
Dimensión modificada & 
 Protocolo en concreto, guías    \\  \hline 
Amenaza a la validez abordada  & El cambio incrementa la validez  interna  \\  \hline \hline

Cambio- \textit{9}   & \parbox[t]{9cm} {Originalmente,  \textit{el tiempo de consolidación es de 30min.} } \parbox[t]{9cm}{en la replicación \textit{el tiempo de consolidación es de 90min. } } debido a que \textit{la duración recomendada de 90 minutos} \\  \hline
Dimensión modificada & 
Protocolo guías     \\  \hline 
Amenaza a la validez abordada  & El cambio incrementa la validez  interna  \\  \hline \hline

Cambio- \textit{10}   & \parbox[t]{9cm} {Originalmente,  \textit{no se mide la dificultad del problema} } \parbox[t]{9cm}{en la replicación \textit{la variable dificultad indica la dificultad del problema } } debido a que \textit{hay dos problemas} \\  \hline
Dimensión modificada & 
 Operacionalización en concreto,  la variable independiente dificultad  \\  \hline 
Amenaza a la validez abordada  &   \\  \hline

\end{tabular}
%\caption{Comparison of previous reviews}
\label{tab:plantilla}
\end{table}


%Cambio-1 Se aumenta la cantidad de Minfulness, parece que mas que cambiar la variable independiente cambio la forma de aplicarlo, sería instrumentalización ??

%76 words



%
%\begin{table}[htpb]
\begin{table}
\caption{Replicación E-2012B}
%\label{tab:1}       % Give a unique label
%\centering
%\small
\begin{tabular}{| p{3.3cm} | p{9cm} |}
\hline

%\textbf { \textit{Mind}\textbf   { \textit{\#1} }} & Replicación del experimento   \textit{Mind \#0}    \\  \hline
\textbf {\textit{E-2012B}} & Replicación del experimento \textit{E-2012A }    \\  \hline

Método empírico &  Experimento   \\  \hline
Tipo &  Interna   \\  \hline
%Objetivo  & El objetivo de la replicación es  \textless \textit{objetivo} \textgreater  \\  \hline 
Objetivo  &  Confirmar resultados  \\  \hline \hline

Cambio- \textit{1}   & \parbox[t]{9cm} {Originalmente,  \textit{Se utilizan dos dominios de problemas en el experimento, uno de dominio conocido (DC) y el otro de dominio desconocido (DD)} } \parbox[t]{9cm}{en la replicación \textit{ Se han modificado los dominios de problemas utilizados en el experimento, pero uno sigue siendo dominio conocido (DC) y el otro dominio desconocido (DD)} }  debido a que \textit{ } \\  \hline
Dimensión modificada & 
 Protocolo en concreto, los objetos experimentales  \\  \hline 
Amenaza a la validez abordada  & El cambio incrementa la validez interna  \\  \hline
 \hline
 
Cambio- \textit{2}   & \parbox[t]{9cm} {Originalmente,  \textit{Primero se realiza el problema de dominio conocido y luego el de dominio desconocido  } } \parbox[t]{9cm}{en la replicación \textit{ Se permuta el orden de realización de los problemas, primero el problema de dominio desconocido y luego el conocido} }  debido a que \textit{ } \\  \hline
Dimensión modificada & 
 Protocolo. en concreto, las guías  \\  \hline 
Amenaza a la validez abordada  & El cambio incrementa la validez interna  \\  \hline
 \hline
 
Cambio- \textit{3}   & \parbox[t]{9cm} {Originalmente,  \textit{el experimento se ha llevado a cabo al principio del curso  } } \parbox[t]{9cm}{en la replicación \textit{se ha llevado a cabo después de que los sujetos hayan recibido formación en Ingeniería de Requisitos y específicamente en educción} }  debido a que \textit{ } \\  \hline
Dimensión modificada & 
Operacionalización, en concreto, una variable de contexto  \\  \hline 
Amenaza a la validez abordada  & El cambio incrementa la validez del constructo  \\  \hline
 \hline


\end{tabular}
%\caption{Comparison of previous reviews}
\label{tab:plantilla}
\end{table}


%Cambio-1 Se aumenta la cantidad de Minfulness, parece que mas que cambiar la variable independiente cambio la forma de aplicarlo, sería instrumentalización ??

%76 words



% ------------------------------------------------------
% File    : Q-2012B.tex
% Content : Q-2012B
% Date    : 1/12/2018
% Version : 1.0
% Authors : M.Cruz
% ------------------------------------------------------
\begin{table*}[h]
%\begin{table}[h]
  %\renewcommand{\arraystretch}{1.45}
  \caption{Instantiation of the proposed template in E-2012B}
\label{tab:plantEng}
  \centering
%	\scriptsize

\begin{tabularx}{\textwidth}{
  >{\hsize=0.3\hsize}X
 %>{\hsize=0.4\hsize}X 2Cols
  >{\hsize=0.8\hsize}X}
  
    \noalign{\smallskip}\hline\noalign{\smallskip}
  
  Field &  Value  \\ 
  \noalign{\smallskip}\hline\noalign{\smallskip}
  
\textbf {\textit{E-2012B}} &  Replication of experiment \textit{E-2012A }    \\
%Empirical method & Experiment\\ 
Type of replication &  Internal   \\  
Purpose  &  Confirm results \\   \hline
%Site & Polytechnic University of Madrid \\
%Date & 2009\\ 

%%%%   
    Change \textit{1}   & \textbf{Originally}, two problem domains are used in the experiment, one known domain (DC) and the other unknown domain (DD) \\& \textbf{In replication} the problem domains used in the experiment have been modified, but one is still a known domain (DC) and the other is an unknown domain (DD) \\& \textbf{because of} ***(\textbf{\textit{Reason}})\\
    
    Modified Dimension & 
   \textbf{Protocol}, specifically the experimental material  \\ 
    Threat to validity & The change increases the internal validity \\  \hline
  
%%%%
    Change \textit{2}   & \textbf{Originally}, first the known domain problem is performed and then the unknown domain problem. \\& \textbf{In replication} the order of the problems is swapped \\& \textbf{because of} ***(\textbf{\textit{Reason}})\\ 

    Modified Dimension & 
    \textbf{Protocol}, specifically the guides \\   
    Threat to validity & The change increases the internal validity  \\  \hline
    
%%%%
 
    
    Change \textit{3}   & \textbf{Originally}, the experiment was carried out at the beginning of the course;\\
    & \textbf{In replication} the experiment is carried out after the subjects have received training in Requirements Engineering \\& \textbf{because of} ***(\textbf{\textit{Reason}})\\
    
    Modified Dimension & 
    \textbf{Operationalization}, ***(\textbf{\textit{Context}}) \\
     Threat to validity & The change increases the construct validity  \\  
    %Threat to validity & The change does not affect the validity  \\ %&  \textbf{because} change the timing of the experiment. It does not involve changing the way treatment is applied or changing metrics and measurement procedures for the same response variable \\
    \noalign{\smallskip\smallskip}\hline

%%%%
    
    
	\end{tabularx}  
	
%\end{table}
\end{table*}
% ------------------------------------------------------
% File    : Q-2013.tex
% Content : Q-2013
% Date    : 1/12/2018
% Version : 1.0
% Authors : M.Cruz
% ------------------------------------------------------
\begin{table*}[h]
%\begin{table}[h]
  %\renewcommand{\arraystretch}{1.45}
  \caption{Instantiation of the proposed template in E-2013}
\label{tab:plantEng}
  \centering
%	\scriptsize

\begin{tabularx}{\textwidth}{
  >{\hsize=0.3\hsize}X
 %>{\hsize=0.4\hsize}X 2Cols
  >{\hsize=0.8\hsize}X}
  
    \noalign{\smallskip}\hline\noalign{\smallskip}
  
  Field &  Value  \\ 
  \noalign{\smallskip}\hline\noalign{\smallskip}
  
\textbf {\textit{E-2013}} &  Replication of experiment \textit{E-2012B}    \\
%Empirical method & Experiment\\ 
Type of replication &  Internal   \\  
Purpose  &  Extend results \\   \hline
%Site & Polytechnic University of Madrid \\
%Date & 2009\\ 

%%%%%
 Change \textit{1}   & \textbf{Originally}, the design is of repeated measurements \\& \textbf{In replication} the design is between-subjects \\& \textbf{In order to } avoid the learning effect \\ 

    Modified Dimension & 
    \textbf{Protocol}, specifically experimental design \\
    Threat to validity & The change increases the internal validity  \\  \hline
    
%%%%   
    Change \textit{2}   & \textbf{Originally}, no short training (warming up) before the course \\& \textbf{In replication} the brief training (warming up) is 1 week \\
    %& \textbf{In order to} explore the warming up effect \\
    & \textbf{Because of}  ***(\textbf{\textit{Reason}})\\
    
    % ***(\textbf{\textit{Reason}})\\
    
    Modified Dimension & 
   \textbf{Operationalization}, ***(\textbf{\textit{Name}}) \\ 
    Threat to validity & The change increases the construct validity \\  
   \noalign{\smallskip\smallskip}\hline
    
	\end{tabularx}  
	
%\end{table}
\end{table*}
% ------------------------------------------------------
% File    : Q-2014.tex
% Content : Q-2014
% Date    : 1/12/2018
% Version : 1.0
% Authors : M.Cruz
% ------------------------------------------------------
\begin{table*}[h]
%\begin{table}[h]
  %\renewcommand{\arraystretch}{1.45}
  \caption{Instantiation of the proposed template in E-2014}
\label{tab:plantEng}
  \centering
%	\scriptsize

\begin{tabularx}{\textwidth}{
  >{\hsize=0.3\hsize}X
 %>{\hsize=0.4\hsize}X 2Cols
  >{\hsize=0.8\hsize}X}
  
    \noalign{\smallskip}\hline\noalign{\smallskip}
  
  Field &  Value  \\ 
  \noalign{\smallskip}\hline\noalign{\smallskip}
  
\textbf {\textit{E-2014}} &  Replication of experiment \textit{E-2013}    \\
%Empirical method & Experiment\\ 
Type of replication &  Internal   \\  
Purpose  &  Extend results \\   \hline
%Site & Polytechnic University of Madrid \\
%Date & 2009\\ 

%%%%%
 Change \textit{1}   & \textbf{Originally}, in the interviews, there are two respondents\\& \textbf{In replication} there  is only one responder \\& \textbf{because of}  the unavailability of one of the respondents \\ 

    Modified Dimension & 
    \textbf{Protocol}, specifically the guides \\
    Threat to validity & The change increases the internal validity  \\  \hline
    
%%%%   
    Change \textit{2}   & \textbf{Originally}, the brief training (warming up) is 1 week \\& \textbf{In replication} the brief training (warming up) is 6 week \\& \textbf{In order to} explore the warming up effect \\
    
   Modified Dimension & 
   \textbf{Operationalization}, ***(\textbf{\textit{Name}}) \\ 
    Threat to validity & The change increases the construct validity \\  
  \noalign{\smallskip\smallskip}\hline
    
	\end{tabularx}  
	
%\end{table}
\end{table*}
% ------------------------------------------------------
% File    : Q-2015.tex
% Content : Q-2015
% Date    : 1/12/2018
% Version : 1.0
% Authors : M.Cruz
% ------------------------------------------------------
\begin{table*}[h]
%\begin{table}[h]
  %\renewcommand{\arraystretch}{1.45}
  \caption{Instantiation of the proposed template in E-2015}
\label{tab:plantEng}
  \centering
%	\scriptsize

\begin{tabularx}{\textwidth}{
  >{\hsize=0.3\hsize}X
 %>{\hsize=0.4\hsize}X 2Cols
  >{\hsize=0.8\hsize}X}
  
    \noalign{\smallskip}\hline\noalign{\smallskip}
  
  Field &  Value  \\ 
  \noalign{\smallskip}\hline\noalign{\smallskip}
  
\textbf {\textit{E-2015}} &  Replication of experiment \textit{E-2013}    \\
%Empirical method & Experiment\\ 
Type of replication &  Internal   \\  
Purpose  &  Extend results \\   \hline
%Site & Polytechnic University of Madrid \\
%Date & 2009\\ 
    
%%%%   
    Change \textit{1}   & \textbf{Originally}, the brief training (warming up) is 1 week \\& \textbf{In replication} the brief training (warming up) is 2 week \\& \textbf{In order to} explore the warming up effect \\
    
   Modified Dimension & 
   \textbf{Operationalization}, ***(\textbf{\textit{Name}}) \\ 
    Threat to validity & The change increases the construct validity \\ 
   \noalign{\smallskip\smallskip}\hline
    
	\end{tabularx}  
	
%\end{table}
\end{table*}
%
%\begin{table}[htpb]
\begin{table}
\caption{Replicación E-2013}
%\label{tab:1}       % Give a unique label
%\centering
%\small
\begin{tabular}{| p{3.3cm} | p{9cm} |}
\hline

%\textbf { \textit{Mind}\textbf   { \textit{\#1} }} & Replicación del experimento   \textit{Mind \#0}    \\  \hline
\textbf {\textit{E-2013}} & Replicación del experimento \textit{E-2012B }    \\  \hline

Método empírico &  Experimento   \\  \hline
Tipo &  Interna   \\  \hline
%Objetivo  & El objetivo de la replicación es  \textless \textit{objetivo} \textgreater  \\  \hline 
Objetivo  &  Replantearse nuevamente el diseño ya que la influencia del conocimiento en la efectividad de los analistas sigue siendo o muy baja o negativa \\  \hline \hline

Cambio- \textit{1}   & \parbox[t]{9cm} {Originalmente,  \textit{ el diseños era de medidas repetidas } } \parbox[t]{9cm}{en la replicación \textit{ el diseño es entre sujetos (between-subjects) } }  debido a que \textit{El diseños de medidas repetidas da lugar a un sesgo derivado del orden en el que se administran los tratamientos a los sujetos. Esto es, por ejemplo el efecto aprendizaje. En el diseño entre sujetos, cada sujeto sólo se somete a un tratamiento } \\  \hline
Dimensión modificada & 
 Instrumentalización en concreto, el diseño experimental  \\  \hline 
Amenaza a la validez abordada  & El cambio incrementa la validez interna  \\  \hline
 \hline
 
Cambio- \textit{2}   & \parbox[t]{9cm} {Originalmente,  \textit{no hay formación breve (warming up) en actividades relacionadas con requisitos  previa a la propiamente dicha del curso} } \parbox[t]{9cm}{en la replicación \textit{ la formación breve (warming up) es de 1 semana} }  con el fin de \textit{explorar un posible efecto de warming up } \\  \hline
Dimensión modificada & 
Operacionalización \\  \hline 
Amenaza a la validez abordada  & El cambio incrementa la validez del constructo  \\  \hline
 \hline
 



\end{tabular}
%\caption{Comparison of previous reviews}
\label{tab:plantilla}
\end{table}


%Cambio-1 Se aumenta la cantidad de Minfulness, parece que mas que cambiar la variable independiente cambio la forma de aplicarlo, sería instrumentalización ??

%76 words



%
%\begin{table}[htpb]
\begin{table}
\caption{Replicación E-2014}
%\label{tab:1}       % Give a unique label
%\centering
%\small
\begin{tabular}{| p{3.3cm} | p{9cm} |}
\hline

%\textbf { \textit{Mind}\textbf   { \textit{\#1} }} & Replicación del experimento   \textit{Mind \#0}    \\  \hline
\textbf {\textit{E-2014}} & Replicación del experimento \textit{E-2013 }    \\  \hline

Método empírico &  Experimento   \\  \hline
Tipo &  Interna   \\  \hline
%Objetivo  & El objetivo de la replicación es  \textless \textit{objetivo} \textgreater  \\  \hline 
Objetivo  &  Con la finalidad de obtener más puntos de datos \\  \hline \hline

 Cambio- \textit{1}   & \parbox[t]{9cm} {Originalmente,  \textit{hay dos respondedores} } \parbox[t]{9cm}{en la replicación \textit{ el número de respondedores se reduce a uno} }  debido a  \textit{la indisponibilidad de unos de los respondedores} \\  \hline
Dimensión modificada & 
 Experimentador Monitor  \\  \hline 
Amenaza a la validez abordada  & El cambio incrementa la validez interna  \\  \hline
 \hline
 
Cambio- \textit{2}   & \parbox[t]{9cm} {Originalmente,  \textit{la formación breve (warming up) en actividades relacionadas con requisitos  previa es de 1 semana} } \parbox[t]{9cm}{en la replicación \textit{ la formación breve (warming up) es de 6 semanas} }  con el fin de \textit{explorar un posible efecto de warming up } \\  \hline
Dimensión modificada & 
Operacionalización \\  \hline 
Amenaza a la validez abordada  & El cambio incrementa la validez del constructo  \\  \hline
 \hline


\end{tabular}
%\caption{Comparison of previous reviews}
\label{tab:plantilla}
\end{table}


%Cambio-1 Se aumenta la cantidad de Minfulness, parece que mas que cambiar la variable independiente cambio la forma de aplicarlo, sería instrumentalización ??

%76 words



%
%\begin{table}[htpb]
\begin{table}
\caption{Replicación E-2015}
%\label{tab:1}       % Give a unique label
%\centering
%\small
\begin{tabular}{| p{3.3cm} | p{9cm} |}
\hline

%\textbf { \textit{Mind}\textbf   { \textit{\#1} }} & Replicación del experimento   \textit{Mind \#0}    \\  \hline
\textbf {\textit{E-2015}} & Replicación del experimento \textit{E-2013 }    \\  \hline

Método empírico &  Experimento   \\  \hline
Tipo &  Interna   \\  \hline
%Objetivo  & El objetivo de la replicación es  \textless \textit{objetivo} \textgreater  \\  \hline 
Objetivo  &  Conseguir el tamaño muestral necesario \\  \hline \hline
 
Cambio- \textit{1}   & \parbox[t]{9cm} {Originalmente,  \textit{ la formación breve (warming up) en actividades relacionadas con requisitos es de 1 semana} } \parbox[t]{9cm}{en la replicación \textit{ la formación breve (warming up) es de 2 semanas} }  con el fin de \textit{explorar un posible efecto de warming up } \\  \hline
Dimensión modificada & 
Operacionalización \\  \hline 
Amenaza a la validez abordada  & El cambio incrementa la validez del constructo  \\  \hline
 \hline

\end{tabular}
%\caption{Comparison of previous reviews}
\label{tab:plantilla}
\end{table}


%Cambio-1 Se aumenta la cantidad de Minfulness, parece que mas que cambiar la variable independiente cambio la forma de aplicarlo, sería instrumentalización ??

%76 words




%
%\begin{table}[htpb]
\begin{table}
\caption{Replicación técnicas de verificación y validación VV-UPM1}
%\label{tab:1}       % Give a unique label
%\centering
%\small
\begin{tabular}{| p{3.3cm} | p{9cm} |}
\hline

%\textbf { \textit{Mind}\textbf   { \textit{\#1} }} & Replicación del experimento   \textit{Mind \#0}    \\  \hline
\textbf {\textit{VV-UPM1}} & Replicación del experimento \textit{técnicas de verificación y validación (VV-UPM }    \\  \hline

Método empírico &  Experimento   \\  \hline
Tipo replicación &  Interna   \\  \hline
Lugar &  Universidad Politécnica de Madrid \\  \hline
Fecha &  2002   \\  \hline
%Objetivo  & El objetivo de la replicación es  \textless \textit{objetivo} \textgreater  \\  \hline 
Objetivo  &  Obtener conclusiones que no podían extraerse del experimento base debido a limitaciones de diseño. \\  \hline \hline
% Se analiza si el tipo de fallo, la técnica  de detección utilizada y el programa afectan a la efectividad en la detección de fallos
Cambio- \textit{1}   & \parbox[t]{9cm} {Originalmente,  \textit{ no se analiza la visibilidad del fallo  } } \parbox[t]{9cm}{en la replicación \textit{ se analiza la influencia de la visibilidad del fallo } }  debido a  \textit{querer obtener más conclusiones} \\  \hline
Dimensión modificada & Operacionalización
 \\  \hline 
Amenaza abordada  &   \\  \hline
Comentario  & Se utiliza el paquete de laboratorio elaborado por Kamsties y Lott  \\  \hline \hline

Cambio- \textit{2}   & \parbox[t]{9cm} {Originalmente,  \textit{ el programa es un factor (variable independiente) aunque no se estudia su influencia } } \parbox[t]{9cm}{en la replicación \textit{ se implementan dos versiones de cada programa que será un nuevo factor } }  debido a \textit{que los programas no son muy largos y por tanto se pueden colocar pocos errores ya que se enmascaran unos a otros } \\  \hline
Dimensión modificada & Operacionalización, en concreto, de la variable independiente versión
 \\  \hline 
Amenaza abordada  &   \\  \hline \hline
Comentario  & Dos versiones difieren en cuanto a los fallos pero tienen el mismo número de fallos y los fallos tienen que ser del mismo tipo \\  \hline \hline
 
 Cambio- \textit{3}   & \parbox[t]{9cm} {Originalmente,  \textit{ tres de los tipos de fallos aparecen solo una vez mientras que los otros tres tipos aparecen dos veces } } \parbox[t]{9cm}{en la replicación \textit{ se duplican todos los tipos de fallos } }  debido a \textit{hay dos vesiones de cada programa } \\  \hline
Dimensión modificada & Protocolo, en concreto, el material experimental
 \\  \hline 
Amenaza abordada  & El cambio incrementa la validez Interna \\  \hline \hline

Cambio- \textit{4}   & \parbox[t]{9cm} {Originalmente,  \textit{ los sujetos generan sus casos de prueba para detectar los fallos del código} } \parbox[t]{9cm}{en la replicación \textit{en primer lugar, los sujetos aplican la técnica para generar los casos de prueba y posteriormente, ejecutarán los casos de prueba que se les proporcionan para detectar los fallos del programa} }  con el fin de  \textit{comprobar si la visibilidad de los fallos influye en su detección} \\  \hline
Dimensión modificada & Protocolo, en concreto, el material experimental
 \\  \hline 
Amenaza abordada  & El cambio incrementa la validez Interna \\  \hline \hline

\end{tabular}
%\caption{Comparison of previous reviews}
\label{tab:plantillaUPM2}
\end{table}


%Cambio-1 Se aumenta la cantidad de Minfulness, parece que mas que cambiar la variable independiente cambio la forma de aplicarlo, sería instrumentalización ??

%76 words



% ------------------------------------------------------
% File    : F3-UPM.tex
% Content : F3.UPM
% Date    : 1/12/2018
% Version : 1.0
% Authors : M.Cruz
% ------------------------------------------------------

\begin{table*}[h]
%\begin{table}[h]
  %\renewcommand{\arraystretch}{1.45}
  \caption{Instantiation of the proposed template in VV-UPM1}
\label{tab:plantEng}
  \centering
%	\scriptsize

\begin{tabularx}{\textwidth}{
  >{\hsize=0.3\hsize}X
 %>{\hsize=0.4\hsize}X 2Cols
  >{\hsize=0.8\hsize}X}
  
    \noalign{\smallskip}\hline\noalign{\smallskip}
  
  Field &  Value  \\ 
  \noalign{\smallskip}\hline\noalign{\smallskip}
  
\textbf {\textit{VV-UPM1}} &  Replication of experiment \textit{VV-UPM}    \\
%Site &  Polytechnic University of Madrid \\  
%Date &  2002   \\  
%Empirical method & Experiment\\ 
Type of replication &  Internal   \\  
Purpose  &  Extend results \\   \hline
%Site & Polytechnic University of Madrid \\
%Date & 2009\\ 
    
%%%%   
    Change \textit{1}   & \textbf{Originally}, the visibility of the fault is not analysed \\& \textbf{In replication} the influence of the visibility of the fault is analysed \\& \textbf{in order to} draw new conclusions \\
    
     Modified Dimension & 
   \textbf{Operationalization}, ***(\textbf{\textit{Name}}) \\ 
    Threat to validity  & The change ***(\textbf{\textit{Threat}}) \\ 
    Comments & Laboratory package developed by Kamsties and Lott is used\\  \hline
    
    %%%%   
    Change \textit{2}  & \textbf{Originally}, the influence of the programme is not analysed \\& \textbf{In replication} two versions of each program are implemented and is a new factor \\& \textbf{because} the programs are not very long and therefore the errors are masked from each other \\
    
    Modified Dimension & 
   \textbf{Operationalization}, specifically the independent variable  \textit {version} \\ 
    Threat to validity & The change increases the construct validity \\  \hline
    
     %%%%   
    Change \textit{3}   & \textbf{Originally}, three of the fault types appear only once while the other three types appear twice \\& \textbf{In replication} all types of faults are duplicated \\& \textbf{because} there are two versions of each program \\ 

    Modified Dimension & 
   \textbf{Protocol}, specifically the experimental material  \\ 
    Threat to validity & The change increases the internal validity \\  \hline
    
    %%%%   
    Change \textit{4}   & \textbf{Originally}, subjects generate their test cases to detect code failures \\& \textbf{In replication} first, the subjects apply the technique to generate the test cases and then execute the test cases provided to them to detect program failures \\& \textbf{in order to} check whether the visibility of faults influences their detection \\ 

    Modified Dimension & 
   \textbf{Protocol}, specifically the experimental material  \\ 
    Threat to validity & The change increases the internal validity \\  \hline
    
    %%%%   
    Change \textit{5}   & \textbf{Originally}, four programs are used \\& \textbf{In replication} three programs are used, one is discarded \\& \textbf{in order to} balance the design \\ 

    Modified Dimension & 
   \textbf{Protocol}, specifically the experimental material  \\ 
    Threat to validity & The change increases the internal validity \\  \hline
    
    %%%%   
    Change \textit{6}   & \textbf{Originally}, each subject applies a technique \\& \textbf{In replication} each subject applies the three techniques \\& \textbf{because} the design is changed \\ 

    Modified Dimension & 
   \textbf{Protocol}, specifically the experimental design  \\ 
    Threat to validity & The change increases the internal validity \\  
   \noalign{\smallskip\smallskip}\hline
    
	\end{tabularx}  
	
%\end{table}
\end{table*}
% ------------------------------------------------------
% File    : F3-UPV.tex
% Content : F3.UPV
% Date    : 1/12/2018
% Version : 1.0
% Authors : M.Cruz
% ------------------------------------------------------

\begin{table*}[h]
%\begin{table}[h]
  %\renewcommand{\arraystretch}{1.45}
  \caption{Instantiation of the proposed template in VV-UPV}
\label{tab:plantEng}
  \centering
%	\scriptsize

\begin{tabularx}{\textwidth}{
  >{\hsize=0.3\hsize}X
 %>{\hsize=0.4\hsize}X 2Cols
  >{\hsize=0.8\hsize}X}
  
    \noalign{\smallskip}\hline\noalign{\smallskip}
  
  Field &  Value  \\ 
  \noalign{\smallskip}\hline\noalign{\smallskip}
  
\textbf {\textit{VV-UPV}} &  Replication of experiment \textit{VV-UPM}    \\
Site &  Polytechnic University of Valencia \\  
%Empirical method & Experiment\\ 
Type of replication &  External   \\  
Purpose  &  Extend results \\   \hline
%Site & Polytechnic University of Madrid \\
%Date & 2009\\ 
    
%%%%   
    Change \textit{1}   & \textbf{Originally}, the three verification and validation techniques are used: code reading, equivalence partitioning and branch testing \\& \textbf{In replication} the code reading technique is omitted  \\& \textbf{because of} time constraints \\
    
     Modified Dimension & 
   \textbf{Operationalization}, specifically the independent variable \textit {technique} \\ 
    Threat to validity & The change increases the construct validity \\  
    Comment & The baseline experiment are UPM replications treated as one 
    \\ \hline
    %%%%   
    Change \textit{2}  & \textbf{Originally}, the duration of the 3 sessions is 4h. each, i.e. the time is unlimited \\& \textbf{In replication} the duration of each of the 3 sessions is 2h. \\& \textbf{because of} time constraints \\
    
    
    Modified Dimension & 
   \textbf{Protocol}, specifically the guides  \\ 
    Threat to validity & The change increases the internal validity \\  \hline
    
     %%%%   
    Change \textit{3}   & \textbf{Originally}, subjects receive three four-hour training sessions to learn how to apply the techniques   \\& \textbf{In replication} the training consists of two two-hour tutorials \\& \textbf{because} he subjects are already familiar with the techniques \\ 

    Modified Dimension & 
   \textbf{Operationalization}, specifically the independent variable  \textit {training} \\ 
    Threat to validity & The change increases the construct validity \\ \hline
    
    %%%%   
    Change \textit{4}   & \textbf{Originally},the training in the use of the techniques is before the experiment is executed \\& \textbf{In replication} Each tutorial is carried out before the application of the technique, in the first 2 sessions; i. e., the training is interspersed with the operation of the experiment \\& \textbf{because} he subjects are already familiar with the techniques \\  

    Modified Dimension & 
   \textbf{Protocol}, specifically the guides   \\
     Threat to validity & The change increases the internal validity  \\ 
    \hline
    %%%%   
    Change \textit{5}   & \textbf{Originally}, subjects apply a technique to a program in each session \\& \textbf{In replication} subjects apply the same technique to different programs in each session \\& \textbf{because of} time constraints \\

    Modified Dimension & 
   \textbf{Protocol}, specifically the guides  \\ 
    Threat to validity & The change increases the internal validity \\  \hline
    
    %%%%   
    Change \textit{6}   & \textbf{Originally}, The subjects execute test cases with the application of the technique; that is to say in each session \\& \textbf{In replication} Subjects run test cases for one of the programs they have tested in a separate session, i.e. in session 3 \\& \textbf{because of} time constraints \\ 

     Modified Dimension & 
   \textbf{Protocol}, specifically the guides  \\ 
    Threat to validity & The change increases the internal validity \\  
   \noalign{\smallskip\smallskip}\hline
    
	\end{tabularx}  
	
%\end{table}
\end{table*}
% ------------------------------------------------------
% File    : F3-UDS.tex
% Content : F3-UDS
% Date    : 1/12/2018
% Version : 1.0
% Authors : M.Cruz
% ------------------------------------------------------

\begin{table*}[h]
%\begin{table}[h]
  %\renewcommand{\arraystretch}{1.45}
  \caption{Instantiation of the proposed template in VV-Uds}
\label{tab:plantEng}
  \centering
%	\scriptsize

\begin{tabularx}{\textwidth}{
  >{\hsize=0.3\hsize}X
 %>{\hsize=0.4\hsize}X 2Cols
  >{\hsize=0.8\hsize}X}
  
    \noalign{\smallskip}\hline\noalign{\smallskip}
  
  Field &  Value  \\ 
  \noalign{\smallskip}\hline\noalign{\smallskip}
  
\textbf {\textit{VV-Uds}} &  Replication of experiment \textit{VV-UPM}    \\
%Site &  University of Seville \\  
%Empirical method & Experiment\\ 
Type of replication &  External   \\  
Purpose  &  Extend results \\   \hline
%Site & Polytechnic University of Madrid \\
%Date & 2009\\ 

    %%%%   
    Change \textit{1}  & \textbf{Originally}, the duration of the 3 sessions is 4h. each, i.e. the time is unlimited \\& \textbf{In replication} the duration of each of the 3 sessions is 2h. \\& \textbf{because of} time constraints \\
    
    
    Modified Dimension & 
   \textbf{Protocol}, specifically the guides  \\ 
    Threat to validity & The change increases the internal validity \\  \hline
    
%%%%   
%%%%   
    Change \textit{2}   & \textbf{Originally}, the subjects execute test cases with the application of the technique; i. e. in each session \\& \textbf{In replication} the subjects execute test cases for one of the programs they have tested in a later session, i.e. in session 4 \\& \textbf{because of} time constraints \\

     Modified Dimension & 
   \textbf{Protocol}, specifically the guides  \\ 
    Threat to validity & The change increases the internal validity \\  \hline
%%%%%%
    Change \textit{3}   & \textbf{Originally}, subjects work individually \\& \textbf{In replication} subjects work in pairs \\& \textbf{because} there are not enough computers  \\
    
     Modified Dimension & 
   \textbf{Protocol}, specifically the guides  \\ 
    Threat to validity & The change increases the internal validity \\  \hline
    
     %%%%   
    Change \textit{4}   & \textbf{Originally}, subjects receive three four-hour training sessions to learn how to apply the techniques   \\& \textbf{In replication} the training consists of two two-hour tutorials \\& \textbf{because} he subjects are already familiar with the techniques \\

    Modified Dimension & 
   \textbf{Operationalization}, specifically the independent variable  \textit {training} \\ 
    Threat to validity & The change increases the construct validity \\ \hline
    
    %%%%   
    Change \textit{5}   & \textbf{Originally},the training in the use of the techniques is before the experiment is executed \\& \textbf{In replication} each tutorial is conducted before the application of the technique in each of the three sessions in which each technique is examined; i.e., the training is interspersed with the operation of the experiment \\& \textbf{because} he subjects are already familiar with the techniques \\

     Modified Dimension & 
   \textbf{Protocol}, specifically the guides   \\
     Threat to validity & The change increases the internal validity  \\ 
   
   \noalign{\smallskip\smallskip}\hline
   	\end{tabularx}  
	
%\end{table}
\end{table*}
% ------------------------------------------------------
% File    : T-ApliORT.tex
% Content : Problems and solutions
% Date    : 1/12/2018
% Version : 1.0
% Authors : M.Cruz
% ------------------------------------------------------
\begin{table*}[h]
%\begin{table}[h]
  %\renewcommand{\arraystretch}{1.45}
  \caption{Instantiation of the proposed template in VV-ORT}
\label{tab:plantEng}
  \centering
%	\scriptsize

\begin{tabularx}{\textwidth}{
  >{\hsize=0.3\hsize}X
 %>{\hsize=0.4\hsize}X 2Cols
  >{\hsize=0.8\hsize}X}
  
    \noalign{\smallskip}\hline\noalign{\smallskip}
  
  Field &  Value  \\ 
  \noalign{\smallskip}\hline\noalign{\smallskip}
  
 \textbf {\textit{VV-ORT}} & 
 Replication of experiment \textit{VV-UPM} \\ 
  
    Type of replication &  External \\  
%    Site & Universidad ORT (Uruguay) \\  

    Purpose  &  Extend results \\  \hline
%%%%   
    Change \textit{1}   & \textbf{Originally}, the three techniques of verification and validation are used: code reading, equivalence partitioning and branch testing \\& \textbf{In replication} the code reading technique is omitted \\& \textbf{because of} time constraints \\
    
    Modified Dimension & 
    \textbf{Operationalization}, specifically the independent variable  \textit {technique} \\   
    Threat to validity  & The change ***(\textbf{\textit{Threat}}) \\  \hline
  
%%%%
    Change \textit{2}   & \textbf{Originally}, three program codes are used \\& \textbf{In replication} one of the programs is  discarded \\& \textbf{because of} time constraints \\  

    Modified Dimension & 
    \textbf{Protocol}, specifically experimental
    material \\   
    Threat to validity  & The change ***(\textbf{\textit{Threat}}) \\  \hline
%%%%
 
    Change \textit{3}   & \textbf{Originally}, the experiment is carried out in three sessions each of four hours \\& \textbf{In replication} the experiment is executed in a single session \\& \textbf{because of} time constraints \\  

    Modified Dimension & 
    \textbf{Protocol}, specifically the guides \\  
    Threat to validity  & The change increases the internal validity \\   \hline

%%%%
    Change \textit{4}   & \textbf{Originally}, subjects apply a different technique to  evaluate a program in each of the three sessions \\& \textbf{In replication} the subjects apply the two techniques to the two programs in a single session 
    \\& \textbf{because of} time constraints \\  
 
    Modified Dimension & 
    \textbf{Protocol}, specifically experimental design \\   
    Threat to validity  & The change increases the internal validity \\  
	\noalign{\smallskip\smallskip}\hline
	\end{tabularx}  
%\end{table}
\end{table*}


%
%\begin{table}[htpb]

\begin{table}

%\label{tab:1}       % Give a unique label
%\centering
%\small
\begin{tabular}{| p{3.3cm} | p{9cm} |}
 \hline
Cambio- \textit{5}   & \parbox[t]{9cm} {Originalmente,  \textit{se utilizaron cuatro programas} } \parbox[t]{9cm}{en la replicación \textit{se utilizaron tres programas, uno fue descartado } }  con el fin de  \textit{equilibrar el diseño} \\  \hline
Dimensión modificada & Protocolo, en concreto, el material experimental
 \\  \hline 
Amenaza abordada  & El cambio incrementa la validez Interna \\  \hline \hline

Cambio- \textit{6}   & \parbox[t]{9cm} {Originalmente,  \textit{cada sujeto aplica una técnica} } \parbox[t]{9cm}{en la replicación \textit{cada sujeto aplica las tres técnicas } }  debido a  \textit{ que se cambia el diseño} \\  \hline
Dimensión modificada &  Protocolo, en concreto,  el diseño experimental
 \\  \hline 
Amenaza abordada  & El cambio incrementa la validez Interna \\  \hline 

%Cambio- \textit{7}   & \parbox[t]{9cm} {Originalmente,  \textit{Hay 195 sujetos distribuidos en 8 grupos de 16 sujetos y 4 grupos de 25 sujetos } } \parbox[t]{9cm}{en la replicación \textit{Hay 46 sujetos distribuidos en 6 grupos de 7-8 sujetos} }  debido a  \textit{.. } \\  \hline
%Dimensión modificada & Instrumentalización en concreto el diseño\\  \hline 
%Amenaza abordada  & El cambio incrementa la validez Interna \\  \hline



\end{tabular}
%\caption{Comparison of previous reviews}
\label{tab:plantillaUPM2b}
\end{table}


%Cambio-1 Se aumenta la cantidad de Minfulness, parece que mas que cambiar la variable independiente cambio la forma de aplicarlo, sería instrumentalización ??

%76 words



%
%\begin{table}[htpb]
\begin{table}
\caption{Replicación técnicas de verificación y validación VV-UPM}
%\label{tab:1}       % Give a unique label
%\centering
%\small
\begin{tabular}{| p{3.3cm} | p{9cm} |}
\hline

%\textbf { \textit{Mind}\textbf   { \textit{\#1} }} & Replicación del experimento   \textit{Mind \#0}    \\  \hline
\textbf {\textit{VV-UPV}} & Replicación del experimento \textit{técnicas de verificación y validación (VV-UPM }    \\  \hline

Método empírico &  Experimento   \\  \hline
Tipo replicación &  Externa   \\  \hline
Lugar & Universidad Politécnica de Valencia  \\  \hline
Fecha &     \\  \hline
%Objetivo  & El objetivo de la replicación es  \textless \textit{objetivo} \textgreater  \\  \hline 
Objetivo  &  Entender la efectividad de tres técnicas de verificación y validación en diferentes contextos \\  \hline \hline
 
Cambio- \textit{1}   & \parbox[t]{9cm} {Originalmente,  \textit{ se utilizan las tres técnicas de verificación y validación: lectura de códigos (code reading), partición de equivalencia (equivalence partitioning) y prueba de rama (branch testing) } } \parbox[t]{9cm}{en la replicación \textit{ se omite la técnica de lectura de códigos (code reading)  } }  debido a  \textit{restricciones de tiempo } \\  \hline
Dimensión modificada & 
Operacionalización en concreto, la variable independiente \textit {técnica} \\  \hline 
Amenaza abordada  & El cambio incrementa la validez del constructo  \\  \hline
Comentario  &  El experimento base son las replicaciones de UPM consideradas como una sola. Solo se hacen los cambios estrictamente requeridos por el nuevo entorno en un intento de mantener los cambios al mínimo. \\  \hline \hline

Cambio- \textit{2}   & \parbox[t]{9cm} {Originalmente,  \textit{ la duración de las tres sesiones es de cuatro horas cada una; es decir el tiempo es ilimitado } } \parbox[t]{9cm}{en la replicación \textit{ la duración de cada una de las tres sesiones es de 2 horas  } }  debido a  \textit{restricciones de tiempo } \\  \hline
Dimensión modificada & 
Protocolo en concreto, guías  \\  \hline 
Amenaza abordada  & El cambio incrementa la validez interna  \\  \hline
 \hline

Cambio- \textit{3}   & \parbox[t]{9cm} {Originalmente,  \textit{ los sujetos reciben tres sesiones de entrenamiento de cuatro horas para aprender a aplicar las técnicas  } } \parbox[t]{9cm}{en la replicación \textit{ la formación consiste en dos breves tutoriales de dos horas } }  debido a  \textit{los sujetos ya están familiarizados con las técnicas } \\  \hline
Dimensión modificada & 
Operacionalización en concreto, la variable independiente \textit {técnica} \\  \hline 
Amenaza abordada  & El cambio incrementa la validez del constructo  \\  \hline
 \hline

Cambio- \textit{4}   & \parbox[t]{9cm} {Originalmente,  \textit{el entrenamiento en el uso de las técnicas es antes de que se ejecute el experimento } } \parbox[t]{9cm}{en la replicación \textit{Cada tutorial se lleva a cabo antes de la aplicación de la técnica, en las dos primeras sesiones; es decir, el entrenamiento es intercalado con la operación del experimento } }  debido a  \textit{los sujetos ya están familiarizados con las técnicas} \\  \hline
Dimensión modificada & Operacionalización, en concreto, una variable de contexto \\  \hline 
Amenaza abordada  & El cambio incrementa la validez del constructo  \\  \hline
 

\end{tabular}
%\caption{Comparison of previous reviews}
\label{tab:plantillaUPV}
\end{table}
% En la 2ª replicación cambia el orden de los MC

%Cambio-1 Se aumenta la cantidad de Minfulness, parece que mas que cambiar la variable independiente cambio la forma de aplicarlo, sería instrumentalización ??

%76 words



%
%\begin{table}[htpb]
\begin{table}
%\caption{Replicación técnicas de verificación y validación VV-UPM}
%\label{tab:1}       % Give a unique label
%\centering
%\small
\begin{tabular}{| p{3.3cm} | p{9cm} |}
\hline
 
 Cambio- \textit{5}   & \parbox[t]{9cm} {Originalmente,  \textit{los sujetos aplican una técnica a un programa en cada sesión} } \parbox[t]{9cm}{en la replicación \textit{aplican la misma técnica a diferentes programas en cada sesión} }  debido a  \textit{restricciones de tiempo} \\  \hline
Dimensión modificada & 
Protocolo en concreto, las guías \\  \hline 
Amenaza abordada  & El cambio incrementa la validez interna  \\  \hline \hline

 Cambio- \textit{6}   & \parbox[t]{9cm} {Originalmente,  \textit{Los sujetos ejecutan casos de prueba con la aplicación de la técnica; es decir en cada sesión} } \parbox[t]{9cm}{en la replicación \textit{Los sujetos ejecutan casos de prueba para uno de los programas que han probado en sesión aparte; es decir en la sesión 3} }  debido a  \textit{restricciones de tiempo} \\  \hline
Dimensión modificada & 
Protocolo en concreto, las guías \\  \hline 
Amenaza abordada  & El cambio incrementa la validez interna  \\  \hline  


\end{tabular}
%\caption{Comparison of previous reviews}
\label{tab:plantillaUPV2}
\end{table}


%Cambio-1 Se aumenta la cantidad de Minfulness, parece que mas que cambiar la variable independiente cambio la forma de aplicarlo, sería instrumentalización ??

%76 words



%
%\begin{table}[htpb]
\begin{table}
\caption{Replicación técnicas de verificación y validación VV-Uds}
%\label{tab:1}       % Give a unique label
%\centering
%\small
\begin{tabular}{| p{3.3cm} | p{9cm} |}
\hline

%\textbf { \textit{Mind}\textbf   { \textit{\#1} }} & Replicación del experimento   \textit{Mind \#0}    \\  \hline
\textbf {\textit{VV-Uds}} & Replicación del experimento \textit{técnicas de verificación y validación (VV-UPM }    \\  \hline

Método empírico &  Experimento   \\  \hline
Tipo replicación &  Externa   \\  \hline
Lugar &  Universidad de Sevilla   \\  \hline
Fecha &     \\  \hline
%Objetivo  & El objetivo de la replicación es  \textless \textit{objetivo} \textgreater  \\  \hline 
Objetivo  &  Entender la efectividad de tres técnicas de verificación y validación en diferentes contextos \\  \hline \hline

Cambio- \textit{1}   & \parbox[t]{9cm} {Originalmente,  \textit{ la duración de las tres sesiones es de cuatro horas cada una; es decir el tiempo es ilimitado } } \parbox[t]{9cm}{en la replicación \textit{ la duración de cada una de las tres sesiones es de 2 horas  } }  debido a  \textit{restricciones de tiempo } \\  \hline
Dimensión modificada & 
Protocolo, en concreto, las guías  \\  \hline 
Amenaza abordada  & El cambio incrementa la validez interna  \\  \hline \hline

Cambio- \textit{2}   & \parbox[t]{9cm} {Originalmente,  \textit{los sujetos ejecutan casos de prueba con la aplicación de la técnica; es decir en cada sesión} } \parbox[t]{9cm}{en la replicación \textit{los sujetos ejecutan casos de prueba para uno de los programas que han probado en una sesión posterior; es decir en la sesión 4} }  debido a  \textit{restricciones de tiempo} \\  \hline
Dimensión modificada & 
Protocolo, en concreto, las guías \\  \hline 
Amenaza abordada  & El cambio incrementa la validez interna  \\  \hline \hline

Cambio- \textit{3}   & \parbox[t]{9cm} {Originalmente,  \textit{los sujetos trabajan de forma individual }} \parbox[t]{9cm}{en la replicación \textit{los sujetos trabajan en parejas} }  debido a  \textit{no hay suficientes ordenadores para todos} \\  \hline
Dimensión modificada & 
Protocolo, en concreto, las guías \\  \hline 
Amenaza abordada  & El cambio incrementa la validez interna  \\  \hline \hline  

Cambio- \textit{4}   & \parbox[t]{9cm} {Originalmente,  \textit{ los sujetos reciben tres sesiones de entrenamiento de cuatro horas para aprender a aplicar las técnicas  } } \parbox[t]{9cm}{en la replicación \textit{ la formación consiste en dos breves tutoriales de dos horas } }  debido a  \textit{los sujetos ya están familiarizados con las técnicas } \\  \hline
Dimensión modificada & 
Operacionalización en concreto, la variable independiente \textit {técnica} \\  \hline 
Amenaza abordada  & El cambio incrementa la validez del constructo  \\  \hline\hline

Cambio- \textit{5}   & \parbox[t]{9cm} {Originalmente,  \textit{el entrenamiento en el uso de las técnicas es antes de que se ejecute el experimento } } \parbox[t]{9cm}{en la replicación \textit{cada tutorial se lleva a cabo antes de la aplicación de la técnica en cada una de las tres sesiones en que se examina cada técnica; es decir, el entrenamiento es intercalado con la operación del experimento } }  debido a  \textit{los sujetos ya están familiarizados con las técnicas} \\  \hline
Dimensión modificada & Operacionalización, en concreto, una variable de contexto \\  \hline 
Amenaza abordada  & El cambio incrementa la validez del constructo  \\  \hline
 
\end{tabular}
%\caption{Comparison of previous reviews}
\label{tab:plantillaUdS}
\end{table}


%Cambio-1 Se aumenta la cantidad de Minfulness, parece que mas que cambiar la variable independiente cambio la forma de aplicarlo, sería instrumentalización ??

%76 words



%
%\begin{table}[htpb]
\begin{table}
\caption{Replicación técnicas de verificación y validación VV-ORT}
%\label{tab:1}       % Give a unique label
%\centering
%\small
\begin{tabular}{| p{3.3cm} | p{9cm} |}
\hline

%\textbf { \textit{Mind}\textbf   { \textit{\#1} }} & Replicación del experimento   \textit{Mind \#0}    \\  \hline
\textbf {\textit{VV-ORT}} & Replicación del experimento \textit{técnicas de verificación y validación (VV-UPM }    \\  \hline

Método empírico &  Experimento   \\  \hline
Tipo replicación &  Externa   \\  \hline
Lugar & Universidad ORT (Uruguay) \\  \hline
Fecha &     \\  \hline
%Objetivo  & El objetivo de la replicación es  \textless \textit{objetivo} \textgreater  \\  \hline 
Objetivo  &  Entender la efectividad de tres técnicas de verificación y validación en diferentes contextos \\  \hline \hline

Cambio- \textit{1}   & \parbox[t]{9cm} {Originalmente,  \textit{ se utilizan las tres técnicas de verificación y validación: lectura de códigos (code reading), partición de equivalencia (equivalence partitioning) y prueba de rama (branch testing) } } \parbox[t]{9cm}{en la replicación \textit{ se omite la técnica de lectura de códigos (code reading)  } }  debido a  \textit{restricciones de tiempo } \\  \hline
Dimensión modificada & 
Operacionalización en concreto, la variable independiente \textit {técnica} \\  \hline 
Amenaza abordada  & El cambio incrementa la validez del constructo  \\  \hline \hline

Cambio- \textit{2}   & \parbox[t]{9cm} {Originalmente,  \textit{ se utilizan tres programas } } \parbox[t]{9cm}{en la replicación \textit{ se omite uno de los programas } }  debido a  \textit{restricciones de tiempo } \\  \hline
Dimensión modificada & 
Protocolo, en concreto, el material experimental \\  \hline 
Amenaza abordada  & El cambio incrementa la validez interna  \\  \hline \hline

Cambio- \textit{3}   & \parbox[t]{9cm} {Originalmente,  \textit{ el experimento se ejecuta en tres sesiones de cuatro horas cada una; es decir el tiempo es ilimitado } } \parbox[t]{9cm}{en la replicación \textit{ el experimento se ejecuta en una única sesión } }  debido a  \textit{  } \\  \hline
Dimensión modificada & 
Protocolo, en concreto, las guías \\  \hline 
Amenaza abordada  & El cambio incrementa la validez interna  \\  \hline \hline

Cambio- \textit{4}   & \parbox[t]{9cm} {Originalmente,  \textit{los sujetos aplican una técnica a un programa en cada una de las tres sesiones} } \parbox[t]{9cm}{en la replicación \textit{los sujetos aplican las dos técnicas a los dos programas en una única sesión} }  debido a  \textit{la duración de la sesión es ilimitada, } \\  \hline
Dimensión modificada & 
Protocolo, en concreto, las guías \\  \hline 
Amenaza abordada  & El cambio incrementa la validez interna  \\  \hline \hline

Cambio- \textit{5}   & \parbox[t]{9cm} {Originalmente,  \textit{los sujetos ejecutan casos de prueba con la aplicación de la técnica} } \parbox[t]{9cm}{en la replicación \textit{no se ejecutan casos de prueba} }  debido a  \textit{los sujetos no pueden acceder a los ordenadores } \\  \hline
Dimensión modificada & 
Protocolo, en concreto, las guías \\  \hline 
Amenaza abordada  & El cambio incrementa la validez interna  \\  \hline
 
\end{tabular}
%\caption{Comparison of previous reviews}
\label{tab:plantillaORT}
\end{table}


%Cambio-1 Se aumenta la cantidad de Minfulness, parece que mas que cambiar la variable independiente cambio la forma de aplicarlo, sería instrumentalización ??

%76 words



%
%\begin{table}[htpb]
\begin{table}
\caption{Instantiation of the proposed template in VV-ORT}
%\label{tab:1}       % Give a unique label
%\centering
%\small
\scalebox{0.95}{
\begin{tabular}{| p{3.3cm} | p{9cm} |}
\hline

%\textbf { \textit{Mind}\textbf   { \textit{\#1} }} & Replicación del experimento   \textit{Mind \#0}    \\  \hline
\textbf {\textit{VV-ORT}} & Replication of experiment \textit{VV-UPM}    \\  \hline

%Método empírico &  Experimento   \\ \hline
Type of replication &  External \\  \hline
Site & Universidad ORT (Uruguay) \\  \hline
%Fecha &     \\  \hline
 
Purpose  &  Extend results \\  \hline \hline

%%%%
Change \textit{1}   & \textbf{Originally}, the three techniques of verification and validation are used: code reading, equivalence partitioning and branch testing \\& \textbf{In replication} the code reading technique is omitted \\& \textbf{because of} time constraints \\ & \\  \hline
%%%%

Modified Dimension & 
\textbf{Operationalization}, specifically the independent variable  \textit {technique} \\  \hline 
Threat to validity & The change decreases the construct validity  \\ & \\  \hline \hline

%%%%
Change \textit{2}   & \textbf{Originally}, three program codes are used \\& \textbf{In replication} one of the programs is  discarded \\& \textbf{because of} time constraints \\ & \\  \hline
%%%%

Modified Dimension & 
\textbf{Protocol}, specifically experimental
material \\  \hline 
Threat to validity  & The change increases the internal validity \\ & \\  \hline \hline


%%%%
Change \textit{3}   & \textbf{Originally}, the experiment is carried out in three sessions each of four hours \\& \textbf{In replication} the experiment is executed in a single session  \\ & \\  \hline
%%%%

Modified Dimension & 
\textbf{Protocol}, specifically the guides \\  \hline 
Threat to validity  & The change increases the internal validity \\ & \\  \hline \hline

%%%%
Change \textit{4}   & \textbf{Originally}, subjects apply a different technique to  evaluate a program in each of the three sessions \\& \textbf{In replication} the subjects apply the two techniques to the two programs in a single session \\&    \\  \hline
%%%%

Modified Dimension & 
\textbf{Protocol}, specifically experimental design \\  \hline 
Threat to validity  & The change increases the internal validity \\ & \\  \hline \hline

%%%%
Change \textit{5}   & \textbf{Originally}, subjects execute test cases with the application of the technique \\& \textbf{In replication} no test cases are executed \\& \textbf{because of} computers are not accessible \\ & \\  \hline
%%%%
Modified Dimension & 
\textbf{Protocol}, specifically the measuring instruments \\  \hline 
Threat to validity  & The change increases the internal validity \\ & \\  \hline 
 
\end{tabular}
}
%\caption{Comparison of previous reviews}
\label{tab:plantEng}
\end{table}


%Cambio-1 Se aumenta la cantidad de Minfulness, parece que mas que cambiar la variable independiente cambio la forma de aplicarlo, sería instrumentalización ??

%76 words



%

% ---- Bibliography ----
%
%\clearpage
\bibliographystyle{plain}%spbasic
%\renewcommand{\refname}{Referencias}
\bibliography{Biblio-V2} 
%\newpage


%
% second contribution with nearly identical text,
\end{document}
