% 
% Section 4
%
% Application of the template to changes in three families of experiments
%
In this section the template is applied to the definition of the changes made in the replications carried out in three series of experiments in the SE area that deal with: i) Mindfulness \cite{bernardez-jss-2016}; ii) Requirements analysis \cite{aranda2016estudio}, and; iii) Code evaluation techniques \cite{juristo2012comparing,juristo2003functional,juristo2013process}.

%
% Section 4.1
% Brief description of the families of experiments
%
\subsection{Brief description of the families of experiments}
\label{sec:brief}
The first series of replications to apply the template (fam\#1) has been chosen for being familiar with it, as it has been carried out by some of the authors of this work. The other two families (fam\#2, fam\#3) have been chosen for knowing the domain of the problem as they are close to our area of knowledge.
\subsubsection{A family of experiments on Mindfulness.}

The family (fam\#1) described in \cite{bernardez2014controlled,bernardez-jss-2016}, consists of one experiment and two internal replications carried out at the University of Seville by some of the authors of the present article. The study deals with the effect of the practice of \emph{Mindfulness} on the performance of students when developing conceptual models. \emph{Mindfulness} is a meditation technique aimed to increase clearness of mind and awareness.
%
%\begin{table}[htpb]
\begin{table}
\caption{Application of the template to the replication \cite{bernardez-jss-2016}}
%\label{tab:1}       % Give a unique label
%\centering
%\small
\begin{tabular}{| p{3.3cm} | p{9cm} |}
%Cuasi Experimento 
%RQ o GQM & \{Del original ?? Yo no lo pondría\}  \\  \hline
\hline
\textbf { \textit{Mind}\textbf   { \textit{\#2} }} & Replication of experiment   \textit{Mind \#1}    \\  \hline
Empirical method &  Controlled experiment  \\  \hline
Type of replication &  Internal   \\  \hline
%Objetivo  & El objetivo de la replicación es  \textless \textit{objetivo} \textgreater  \\  \hline 
Target  &   Confirm results   \\  \hline \hline

Change \textit{\#1}   & \parbox[t]{9cm} {Originally,  \textit{for 4 weeks Mindfulness was practiced 4 days a week in 10-minute sessions.} } \parbox[t]{9cm}{In replication \textit{the sessions were 12 minutes long and for 6 weeks} }   in order to  \textit{make more evident the benefits of Mindfulness..} \\  \hline
Modified Dimension & 
Operationalization, specifically, the independent variable \textit {Training Workshop }  \\  \hline 
Validity threat addressed  &  The change increases the construct validity   \\  \hline
 \hline
Change \textit{\#2}   & \parbox[t]{9cm} {Originally,  \textit{the assignment of subjects to treatment was not randomized.} } \parbox[t]{9cm}{In replication \textit{it becomes random} }  in order to \textit{remedy threats to the internal validity of quasi-experiments.} \\  \hline
Modified Dimension & Protocol, specifically, experimental design \\  \hline 

Validity threat addressed  &  The change increases the internal validity   \\  \hline \hline

Change \textit{\#3}   & \parbox[t]{9cm} {Originally,  \textit{an public speaking workshop was given to the control group as a placebo.} } \parbox[t]{9cm}{In replication \textit{the oratory workshop took place after the experiment} } in order to  \textit{avoid a possible effect of such a workshop on the measurements of dependent variables.} \\  \hline
Modified Dimension & 
Operationalization, specifically, the independent variable \textit {Training Workshop }  \\  \hline 
Validity threat addressed  &  The change increases the construct validity   \\  \hline \hline

Change \textit{\#N}   & \parbox[t]{9cm} {Originally,  \textit{an public speaking workshop was given to the control group as a placebo.} } \parbox[t]{9cm}{In replication \textit{the oratory workshop took place after the experiment} } in order to  \textit{avoid a possible effect of such a workshop on the measurements of dependent variables.} \\  \hline
Modified Dimension & 
Operationalization, specifically, the independent variable \textit {Training Workshop }  \\  \hline 
Validity threat addressed  &  The change increases the construct validity   \\  \hline
\end{tabular}
%Firstly, the problem of conceptual modeling "Erasmus" is carried out. Secondly, the problem of conceptual modeling "EoD project" is carried out.
%\caption{Comparison of previous reviews}
\label{tab:plantilla-mind}
\end{table}
%Cambio-1 Se aumenta la cantidad de Minfulness, parece que mas que cambiar la variable independiente cambio la forma de aplicarlo, sería instrumentalización ??
% en la 2º replicacion cambia Conceptual modeling exercise order

%76 words



%In Appendix A, 
%The table \ref{tab:plantilla-mind}

This is the first instantiation of the template. When applying the template, the second replication has been defined based on the first replication. With its current structure we are able to make explicit the changes in the two replications cited. However, the template needs to be validated in replications of experiments carried out by other researchers.

\subsubsection{A family of experiments on Requirements Analysis.}

This is an empirical study about  the influence of analyst domain knowledge and experience on the effectiveness of \emph{requirements analysis} process. The family (fam\#2) consists of nine experiments involving postgraduate students at the Polytechnic University of Madrid, as well as professionals from different countries and institutions. The whole family is described in \cite{aranda2016estudio}.

According to the author, the acronyms for the experiments and replicactions are: Q-2007, Q-2009, Q-2011, Q-2012, E-2012A, E-2012B, E-2013, E-2014 and E-2015. Q-2007 is the original experiment. 
Each experiment is a replication of the previous one. Therefore, the template has been applied defining the changes in each replication with respect to the previous replication. %except the E-2015 experiment which is a replication of the E-2013. 
The first step in filling out the template is to identify on which experiment or previous replication the replication is defined.

\subsubsection{A family of experiments on Code Evaluation Techniques.}

It consists of one experiment and its four replications. The original experiment is conducted with students at the Polytechnic University of Madrid to evaluate the effectiveness of three verification and validation techniques. This experiment has been replicated in four different sites: the Polytechnic University of Madrid, the Polytechnic University of Valencia, the University of Seville and the ORT University of Uruguay. The complete family of experiments (fam\#3) is described in \cite{juristo2012comparing,juristo2003functional,juristo2013process}.
%
%\begin{table}[htpb]
\begin{table}
\caption{Instantiation of the proposed template in VV-ORT}
%\label{tab:1}       % Give a unique label
%\centering
%\small
\scalebox{0.95}{
\begin{tabular}{| p{3.3cm} | p{9cm} |}
\hline

%\textbf { \textit{Mind}\textbf   { \textit{\#1} }} & Replicación del experimento   \textit{Mind \#0}    \\  \hline
\textbf {\textit{VV-ORT}} & Replication of experiment \textit{VV-UPM}    \\  \hline

%Método empírico &  Experimento   \\ \hline
Type of replication &  External \\  \hline
Site & Universidad ORT (Uruguay) \\  \hline
%Fecha &     \\  \hline
 
Purpose  &  Extend results \\  \hline \hline

%%%%
Change \textit{1}   & \textbf{Originally}, the three techniques of verification and validation are used: code reading, equivalence partitioning and branch testing \\& \textbf{In replication} the code reading technique is omitted \\& \textbf{because of} time constraints \\ & \\  \hline
%%%%

Modified Dimension & 
\textbf{Operationalization}, specifically the independent variable  \textit {technique} \\  \hline 
Threat to validity & The change decreases the construct validity  \\ & \\  \hline \hline

%%%%
Change \textit{2}   & \textbf{Originally}, three program codes are used \\& \textbf{In replication} one of the programs is  discarded \\& \textbf{because of} time constraints \\ & \\  \hline
%%%%

Modified Dimension & 
\textbf{Protocol}, specifically experimental
material \\  \hline 
Threat to validity  & The change increases the internal validity \\ & \\  \hline \hline


%%%%
Change \textit{3}   & \textbf{Originally}, the experiment is carried out in three sessions each of four hours \\& \textbf{In replication} the experiment is executed in a single session  \\ & \\  \hline
%%%%

Modified Dimension & 
\textbf{Protocol}, specifically the guides \\  \hline 
Threat to validity  & The change increases the internal validity \\ & \\  \hline \hline

%%%%
Change \textit{4}   & \textbf{Originally}, subjects apply a different technique to  evaluate a program in each of the three sessions \\& \textbf{In replication} the subjects apply the two techniques to the two programs in a single session \\&    \\  \hline
%%%%

Modified Dimension & 
\textbf{Protocol}, specifically experimental design \\  \hline 
Threat to validity  & The change increases the internal validity \\ & \\  \hline \hline

%%%%
Change \textit{5}   & \textbf{Originally}, subjects execute test cases with the application of the technique \\& \textbf{In replication} no test cases are executed \\& \textbf{because of} computers are not accessible \\ & \\  \hline
%%%%
Modified Dimension & 
\textbf{Protocol}, specifically the measuring instruments \\  \hline 
Threat to validity  & The change increases the internal validity \\ & \\  \hline 
 
\end{tabular}
}
%\caption{Comparison of previous reviews}
\label{tab:plantEng}
\end{table}


%Cambio-1 Se aumenta la cantidad de Minfulness, parece que mas que cambiar la variable independiente cambio la forma de aplicarlo, sería instrumentalización ??

%76 words



% ------------------------------------------------------
% File    : Fam3-ORT-EngV2.tex
% Content : Problems and solutions
% Date    : 1/12/2018
% Version : 1.0
% Authors : M.Cruz
% ------------------------------------------------------

\begin{table}[h]
  %\renewcommand{\arraystretch}{1.45}
  \caption{Instantiation of the proposed template in VV-ORT}
\label{tab:plantEng}
  \centering
	\scriptsize

\begin{tabularx}{\textwidth}{
  >{\hsize=0.55\hsize}X
  >{\hsize=1.45\hsize}X}
  
    \noalign{\smallskip}\hline\noalign{\smallskip}
  
  Field &  Value  \\ 
  \noalign{\smallskip}\hline\noalign{\smallskip}
  
 \textbf {\textit{VV-ORT}} & 
 Replication of experiment \textit{VV-UPM} \\ 
  
    Type of replication &  External \\  
    Site & Universidad ORT (Uruguay) \\  

    Purpose  &  Extend results \\  \hline
%%%%   
    Change \textit{1}   & \textbf{Originally}, the three techniques of verification and validation are used: code reading, equivalence partitioning and branch testing \\& \textbf{In replication} the code reading technique is omitted \\& \textbf{because of} time constraints \\
    
    Modified Dimension & 
    \textbf{Operationalization}, specifically the independent variable  \textit {technique} \\   
    Threat to validity & The change decreases the construct validity  \\  \hline
  
%%%%
    Change \textit{2}   & \textbf{Originally}, three program codes are used \\& \textbf{In replication} one of the programs is  discarded \\& \textbf{because of} time constraints \\  

    Modified Dimension & 
    \textbf{Protocol}, specifically experimental
    material \\   
    Threat to validity  & The change increases the internal validity \\  \hline
%%%%
 
    Change \textit{3}   & \textbf{Originally}, the experiment is carried out in three sessions each of four hours \\& \textbf{In replication} the experiment is executed in a single session  \\  

    Modified Dimension & 
    \textbf{Protocol}, specifically the guides \\  
    Threat to validity  & The change increases the internal validity \\   \hline

%%%%
    Change \textit{4}   & \textbf{Originally}, subjects apply a different technique to  evaluate a program in each of the three sessions \\& \textbf{In replication} the subjects apply the two techniques to the two programs in a single session \\ 
 
    Modified Dimension & 
    \textbf{Protocol}, specifically experimental design \\   
    Threat to validity  & The change increases the internal validity \\    

	\noalign{\smallskip\smallskip}\hline
	\end{tabularx}  
\end{table}


\begin{table}
\caption{\textcolor[rgb]{1,0,0}{Quitar ?.} Analysis of the changes in the family of Requirements analysis experiments}
\label{tab:changesALE}
%\label{tab:1}       % Give a unique label
%\centering
%\small
%\begin{tabular}{| l | c | c | p{2cm} | c | c |c |}
%\begin{tabular}{| l | c | p{2cm} | p{2cm} | c | p{1.5cm} | c |}
\begin{minipage}{6cm}

%\begin{tabular}{| l | l | l | l | l | l | l |}
\begin{tabular}{| l | l | l |p{6cm} | p{1cm}|}
%\hline\noalign{\smallskip}
\hline
\textbf{Id} & \textbf{Original} & \textbf{Change}  & \textbf{Description}& \textbf{Type}\\
\hline
% Mind2 & Mind2  &~& Replications & 2013--2017 & 105 & Current state of replication\\

Q-2009 & Q-2007 & Ch-1 & Effectiveness is not measured  & P1 \\
~ & ~ & Ch-2 & Retention capacity is not measured  & P1 \\
~ & ~ & Ch-3 & The influence of the development experience is analysed  & P1, P2 \\ % P1, P2
~ & ~ & Ch-4 & Interviews are conducted in English  &\ding{51} \\
~ & ~ & Ch-5 & The respondent is changed in interviews & P2, P5 \\ \hline
Q-2011 & Q-2009 & Ch-1 & Group interviews  & \ding{51} \\
~ & ~ & Ch-2 & The influence of skill on requirements and skill in interviews is analyzed  & P1, P2 \\ % P1, P2
~ & ~ & Ch-3 & Increases the interview time  & \ding{51} \\
~ & ~ & Ch-4 & Decreases the time to present information  & \ding{51} \\
~ & ~ & Ch-5 & The consolidation time is limited  & \ding{51} \\
~ & ~ & Ch-6 & The respondent is changed in interviews & P2, P5 \\ \hline
Q-2012 & Q-2011 & Ch-1 & The subjects are professionals  & \ding{51} \\
~ & ~ & Ch-2 & The influence of development skill is analyzed  & P1 \\
~ & ~ & Ch-3 & Decreases consolidation time  & \ding{51} \\
~ & ~ & Ch-4 & No training period  & P1 \\ % P1, P3 
\hline
%T8 Operacionalización pero no conozco la variable
% T3 ya que suprimo la variable
% Relacionando con training es Operacionalización (aunque no se como se llama la variable) y es T0)
E-2012A & Q-2012 & Ch-1 & The influence of knowledge is analysed  & P1 \\
~ & ~ & Ch-2 &  Change in experimental design  & \ding{51} \\
~ & ~ & Ch-3 &  Individual interviews  & \ding{51} \\
~ & ~ & Ch-4 &  The language is a blocking variable.  & \ding{51} \\
~ & ~ & Ch-5 &  The respondent is a blocking variable  & \ding{51} \\
~ & ~ & Ch-6 &  There are two responders  & \ding{51} \\% Cambiar en la plantilla a Operacionalización sin elemento
~ & ~ & Ch-7 &  Groups are formed  & \ding{51} \\ 
~ & ~ & Ch-8 &  Decreases elicitation time  & \ding{51} \\
~ & ~ & Ch-9 &  Increases consolidation time  & \ding{51} \\
~ & ~ & Ch-10 &  The influence of the difficulty of the problem is analysed  & P1 \\ \hline
E-2012B & E-2012A & Ch-1 & The problem domains are modified & P2 \\
~ & ~ & Ch-2 &  The order of the problems is changed  & P2 \\
% Ya no es T6 ya que identifico el cambio de orden
~ & ~ & Ch-3 &  \textcolor[rgb]{1,0,0}{The experiment is conducted at the end of the course}   & P2, P4\\  \hline %P2, P4
E-2013 & E-2012B & Ch-1 & Change in experimental design  & \ding{51} \\
~ & ~ & Ch-2 &  Training is provided  & P3 \\ \hline
E-2014 & E-2013 & Ch-1 & There are one responders  & \ding{51} \\
~ & ~ & Ch-2 &  Increases training time  & P3\\ \hline
E-2015 & E-2013 & Ch-1 & Increases training time  & P3 \\ 


%\noalign{\smallskip}\hline
\hline

\end{tabular}
\end{minipage}
%\caption{Comparison of related reviews}


\end{table}


%76 words



\begin{table}
\caption{\textcolor[rgb]{1,0,0}{Quitar ?.} Analysis of the changes in the series of experiments on Code evaluation techniques}
\label{tab:changesVV}
\label{tab:1}       % Give a unique label
%\centering
%\small
%\begin{tabular}{| l | c | c | p{2cm} | c | c |c |}
%\begin{tabular}{| l | c | p{2cm} | p{2cm} | c | p{1.5cm} | c |}
\begin{minipage}{6cm}

%\begin{tabular}{| l | l | l | l | l | l | l |}
\begin{tabular}{| l | l | l |p{6cm} |  l |}
%\hline\noalign{\smallskip}
\hline
\textbf{Id} & \textbf{Original} & \textbf{Change}  & \textbf{Description}& \textbf{Type}\\
\hline

VV-UPM1 & VV-UPM & Ch-1 & The visibility of the error is analysed  & P1, P3 \\% Es P1 y P3 
 
~ & ~ & Ch-2 &  Two versions of each program  & P1 \\
~ & ~ & Ch-3 &  All types of failures are duplicated  & \ding{51} \\
~ & ~ & Ch-4 &  Test cases are provided & \ding{51} \\
~ & ~ & Ch-5 &  A program is discarded & \ding{51} \\
~ & ~ & Ch-6 &  Each subject applies the three techniques & \ding{51} \\ \hline
VV-UPV & VV-UPM & Ch-1 & A reading technique is omitted & \ding{51} \\
~ & ~ & Ch-2 &  The duration of the sessions is limited & \ding{51} \\
~ & ~ & Ch-3 &  The duration of training is reduced & \ding{51} \\
~ & ~ & Ch-4 &  \textcolor[rgb]{1,0,0}{The moment of application of the treatment is changed} & P4 \\
% T0, 
~ & ~ & Ch-5 &  Changes in the application of techniques & \ding{51} \\
~ & ~ & Ch-6 &  Changes in the application of test cases & \ding{51} \\ \hline

VV-Uds & VV-UPM & Ch-1 & The duration of the sessions is limited & \ding{51} \\
~ & ~ & Ch-2 &  Change at time of test case execution & \ding{51} \\
% T0,  Protocolo en concreto el orden de ejecución
~ & ~ & Ch-3 &  The subjects work in pairs & \ding{51} \\
~ & ~ & Ch-4 &  The duration of training is reduced & \ding{51} \\
~ & ~ & Ch-5 &  \textcolor[rgb]{1,0,0}{The moment of application of the treatment is changed} & P4 \\ \hline
% T0, No estaba claro 
VV-ORT & VV-UPM & Ch-1 &  A reading technique is omitted & \ding{51} \\
~ & ~ & Ch-2 &  A program is discarded & \ding{51} \\
%Instrumentalizacion en concreto el material experimental
~ & ~ & Ch-3 &  There is only one session & P2 \\
~ & ~ & Ch-4 &  Changes in the application of techniques to programs. & \ding{51} \\
~ & ~ & Ch-5 &  No test cases are executed & \ding{51} \\

%\noalign{\smallskip}\hline
\hline

\end{tabular}
\end{minipage}
%\caption{Comparison of related reviews}


\end{table}


%76 words



%The acronyms used to reference the original and the replications are VV-UPM, VV-UPM1, VV-UPV, VV-Uds and VV-ORT. 

All of the replications are described based on the original experiment.
To illustrate the use of the template, table \ref{tab:plantEng} shows the result of applying the template and L–patterns to define the changes made in one of the replications of this family, specifically in the replication carried out at the University of Uruguay.


%In Appendix A, tables Ta, Tb, Tc and Td  show the result of applying the template and L–patterns to define the changes of the replications of this series. 

%\subsubsection{Lessons learned} %\label{sec:LessonsVV}

%The table \ref{tab:plantilla-V2} shows the template with the modifications after its application to the three series of experiments.

%% ------------------------------------------------------
% File    : Jbd-Tplantilla.tex
% Content : Plantilla
% Date    : 12/Feb/2018
% Version : 1.0
% Authors : 
% ------------------------------------------------------
% ------------------------------------------------------
% Last update: ...// 
% - Added ..
% ------------------------------------------------------


%\begin{table}[htpb]
\begin{table}
\caption{Template for specifying changes in a replication after applying lessons learned}
\label{tab:plantilla-V2}
%\caption{Template for specifying changes in a replication}
%\label{tab:1}       % Give a unique label
%\centering
%\small
\begin{tabular}{| p{3.3cm} | p{9cm} |}
%Cuasi Experimento 
%RQ o GQM & \{Del original ?? Yo no lo pondría\}  \\  \hline
\hline


\textbf {\textless\textit{Acronym}\textbf {\textgreater} \#{\textless\textit{n}\textgreater}} & Replication of experiment \textless \textit{Acronym of experiment} \textgreater \#{\textless\textit{m}\textgreater} \\  \hline
%Environment
Site  &   {\textless\textit{Date}\textgreater}  \\   &  \textless  \textit{Place}\textgreater \\    \hline

%Empirical method & \{Controlled experiment  \hspace{1 mm} \textbar \hspace{1 mm}  Quasi-experiment \textbar  \hspace{1 mm}  Survey \textbar \hspace{1 mm}  Case study \textbar \hspace{1 mm}  Observational study \textbar \hspace{1 mm}  Pilot study \}  \\  \hline
Type of replication & \{Internal \textbar \hspace{0.5 mm}  External  \}  \\  \hline


%Target  &  \parbox[t]{9cm} {\{Confirm results \textbar }  \parbox[t]{9cm} { Generalize the results \textbar}   \\   & \textless\textit{Custom target}\textgreater  \}  \\   \hline \hline
Purpose  &  \parbox[t]{9cm} {\{Confirm results \textbar }  \parbox[t]{9cm} {Extend results \textbar} \hspace{1 mm}  Overcome any limitations  \} \\  \hline \hline

Change \#\textless\textit  {i..j}\textgreater  & \parbox[t]{9cm} {Originally, \textless\textit{description of the situation in original experiment}\textgreater} \parbox[t]{9cm}{In replication \textless\textit{description of the situation in replication}\textgreater}  \{in order to  \textbar \hspace{1 mm} because of  \}  \textless\textit{cause of change}\textgreater  \\  \hline

% con el fin de | debido a
[Modified Dimension] & 
\parbox[t]{9cm}  { \{Operationalization  [(, in particular, of the  \{dependent \textbar  independent\} variable \textless \textit{variable name}\textgreater)]  \textbar }  
\parbox[t]{9cm} {Population[(, in particular, experimental \{subjects \textbar objects\} )]\textbar} 
\parbox[t]{9cm} {Protocol [(, in particular, \{experimental design \textbar experimental material \textbar guides \textbar measuring instruments \textbar order of application \} )] \textbar } 
\parbox[t]{9cm}  {Experimenters [(, in particular, \{ designer \textbar   analyst \textbar  trainer \textbar  monitor \textbar  measurer\})]  \} }   \\  \hline 


Threat to validity addressed  &  The change increases  \{the construct  \textbar external \textbar internal \textbar conclusion  \} validity \\  \hline
[Remarks]  &   {\textless\textit{Remarks}\textgreater}  \\  \hline
% se utilizan distintas instancias del mismo tipo de objeto experimental

\end{tabular}

\end{table}


%76 words


%\subsection{Purpose of the replications}

%% -------------------

%\begin{table}[htpb]
\begin{table}
\label{tab:purpose}
\caption{Purpose of replications}
%\small
\begin{tabular}{| l |c | c | c | c|}

%\begin{tabular}{| l |c | c | c | c| c | }
\hline

%\textbf{Related works} & \textbf{Confirm}  & \textbf{Extend} & \textbf{Overcome} &  \textbf{Not completed}   \\ \hline
\textbf{Related works} & \textbf{Confirm results}  & \textbf{Extend results} & \textbf{Overcome limitations}   \\ \hline
Mind \#2  & \ding{51} &   &    \\ \hline
Q-2009   & \ding{51} &   &    \\ \hline
%Q-2009   &  &   &  & \ding{51}   \\ \hline
Q-2009   &  &   &     \\ \hline
Q-2012   &  & \ding{51}   &    \\ \hline
E-2012A   &  & \ding{51}  &     \\ \hline
E-2012B   & \ding{51} &   &     \\ \hline
E-2013   &  &   & \ding{51}    \\ \hline
E-2014   &  \ding{51} &   &     \\ \hline
E-2015   & \ding{51} &   &      \\ \hline
VV-UPM1   &  &   & \ding{51}     \\ \hline
VV-UPV   &  & \ding{51}   &     \\ \hline
VV-Uds   &  & \ding{51}  &     \\ \hline
VV-ORT   &  & \ding{51}  &     \\ \hline


%Mind   & Guidelines & \ding{51}  &\ding{55} &  \ding{55} \\ \hline

%\noalign{\smallskip}\hline
%Confirm results|Extend results|Overcome any limitations}
\end{tabular}

\end{table}


%76 words

%The template is applied to the first family of experiments. 
\subsection{Procedure for template validation}
\label{sec:procedure}
The application of the template to the three families of experiments is an iterative process. The following phases can be distinguished: 

\begin{itemize}
 \item The first family of experiments is surveyed. 
    \begin{itemize}
        \item For each replication, the template is instantiated.
        \item If there is any problem in the definition of any field of the template, it is noted (see Table \ref{tab:plantillaProblem}).
        \begin{itemize}
            \item For each change, if there is any problem in defining the change, it is analyzed and typified (see Table \ref{tab:tipos}).
            \item The change is associated with one or more of the typified values.
        \end{itemize}
       
        \item The impact of the problems detected in the template and in the metamodel is analysed (see Section \ref{sec:impact}). 
    \end{itemize}

  
 \item Repeat the process with the next family updating Table \ref{tab:plantillaProblem} and Table \ref{tab:tipos} when necessary.
\end{itemize}

\subsection{Findings on the application of the template}
\label{sec:findings}
Table \ref{tab:plantillaProblem} shows, over the template, the difficulties faced when filling in each of its fields.
%
\begin{table}
\caption{Findings in each field of the template.}
\label{tab:plantillaProblem}

\begin{tabular}{| p{3.3cm} | p{9cm} |}

\hline

\textbf {\textless\textit{Acronym}\textbf {\textgreater} \#{\textless\textit{n}\textgreater}}  & \ding{51} (1)  \\  \hline

Site \& date  &  (2)   \\  \hline

Type of replication & \ding{51}  \\  \hline

%Purpose  &  \parbox[t]{9cm} {\{Confirm results \textbar }  \parbox[t]{9cm} {Extend results \textbar} \hspace{1 mm}  Overcome any limitations  \} \\  \hline 

Purpose  & \ding{51} (3) \\  \hline \hline

Change \#\textless\textit  {i..j}\textgreater  &   In 8 of the 59 changes, the reason for the change is not specified.  \\  \hline

[Modified Dimension] & In 6 of the 59 changes, the affected dimension or element is not identified.
   \\  \hline

Threat to validity   &  In 10 of the 59 changes, the threat to validity is not affected or decreases however the pattern does not allow this specification. \\  \hline
[Comments]  &  \ding{51} (4) \\  \hline


\end{tabular}

\end{table}


% ------------------------------------------------------
% File    : Fam-TplantillaProblemV2.tex
% Content : Problems and solutions
% Date    : 1/12/2018
% Version : 1.0
% Authors : M.Cruz
% ------------------------------------------------------

\begin{table}[h]
  %\renewcommand{\arraystretch}{1.45}
  \caption{Findings in each field of the template.}
  \label{tab:plantillaProblem}
  \centering
	\scriptsize
  %begin{tabularx}{\textwidth}{cXX}
\begin{tabularx}{0.9\textwidth}{
  >{\hsize=0.65\hsize}X
  >{\hsize=0.01\hsize}X
  >{\hsize=1.34\hsize}X}
  
	%\hline\noalign{\smallskip}
		
    %\textbf{Id\#} & 
    %\textbf{Problems detected in the use of template} & \textbf{Proposed solutions} \\

	\noalign{\smallskip}\hline\noalign{\smallskip}
	\textbf {\textless\textit{Acronym}\textbf {\textgreater} \#{\textless\textit{n}\textgreater}}  & &\ding{51} (1)  \\  
    Site \& date  & & (2)   \\ 
    Type of replication & & \ding{51}  \\ Purpose  & & \ding{51} (3) \\  \hline 
    Change \#\textless\textit  {i..j}\textgreater  &  & In 8 of the 59 changes, the reason for the change is not specified \\ 
    
    [Modified Dimension] & & In 6 of the 59 changes, the affected dimension or element is not identified \\

    Threat to validity   &  & In 10 of the 59 changes, the threat to validity is not affected or decreases however the pattern does not allow this specification \\
    
    [Comments]  & & \ding{51} (4) \\

	\noalign{\smallskip\smallskip}\hline
	\end{tabularx}  
\end{table}

%
\begin{table}
\caption{Findings in each field of the template.}
\label{tab:plantillaProblem2}

\begin{tabular}{| p{3.3cm} | p{9cm} |}

\hline

\textbf {\textless\textit{Acronym}\textbf {\textgreater} \#{\textless\textit{n}\textgreater}}  & Although replication is identified in the template by a code or acronym relating to the baseline experiment and followed by a sequential number, it is advisable to follow the author's notation if different.  \\  \hline

Site \& date  & We realise the need to add the \emph{date} and \emph{site} where the replication is carried out to the template. A new \emph{site} dimension is identified in \cite{Juristo2012}; if replication is carried out at a \emph{site} other than the original experiment, the \emph{site} dimension is changed. Nevertheless, we consider \emph{site} as a template field and not as a new dimension.   \\  \hline

Type of replication & \ding{51}  \\  \hline

%Purpose  &  \parbox[t]{9cm} {\{Confirm results \textbar }  \parbox[t]{9cm} {Extend results \textbar} \hspace{1 mm}  Overcome any limitations  \} \\  \hline 

Purpose  & Except for one of the replications in which its objective is not specified, the rest fits into one of the three possible values to be selected in the template. Purposes of replications of all the proposed types have been found. \\  \hline \hline

Change \#\textless\textit  {i..j}\textgreater  &   In 8 of the 59 changes, the reason for the change is not specified.  \\  \hline

[Modified Dimension] & In 6 of the 59 changes, the affected dimension or element is not identified.
   \\  \hline

Threat to validity   &  In 10 of the 59 changes, the threat to validity is not affected or decreases however the pattern does not allow this specification. \\  \hline
[Comments]  &   The \textit{quasi-experiments}, are included in the \textit{controlled experiments} category following \cite{wohlin:experimentation}, nevertheless, if the author defines replication as a \textit{quasi-experiment}, it should be reflected in the template (e.g. in the comments). \\  \hline


\end{tabular}

\end{table}


\begin{itemize}

\item (1) Although replication is identified in the template by a code or acronym relating to the baseline experiment and followed by a sequential number, it is advisable to follow the author's notation if different.
\item (2) We realize the need to add the \emph{date} and \emph{site} where the replication is carried out to the template. A new \emph{site} dimension is identified in \cite{Juristo2012}; if replication is carried out at a \emph{site} other than the original experiment, the \emph{site} dimension is changed. Nevertheless, we consider \emph{site} as a template field and not as a new dimension.  
 
%\item When the change affects more than one dimension it is advisable to split into several simple changes

\item (3) Concerning the \emph{Purpose of Replication}, except for one of the replications in which its objective is not specified, the rest fits into one of the three possible values to be selected in the template. Purposes of replications of all the proposed types have been found.

\item (4) The \textit{quasi-experiments}, are included in the \textit{controlled experiments} category following \cite{wohlin:experimentation}, nevertheless, if the author defines replication as a \textit{quasi-experiment}, it should be reflected in the template (e.g. in the comments).
%\item The empirical method has been removed from the template and the use of the template is recommended only for \textit{controlled experiments}.  In \emph{surveys}, \emph{observational studies} and other types of empirical methods, most of the concepts in the template, such as independent variables, do not exist.

%The use of the template avoids the lack of information on the purpose of the replication.
\end{itemize}

\subsection{Impact of the findings on the template and metamodel}
\label{sec:impact}

The problems arising in the definition of changes in replications have been typified in order to be able to analyse whether they affect: i) the metamodel and, therefore, also the template and its L-patterns; ii) only the template and L-patterns (see Table \ref{tab:tipos}).

% shows the typified problems faced through the iterative process for template validation in the replications of each family. The solution to these problems involves changes in the metamodel or changes in the template or its L-patterns. On the other hand, the lack of information when describing the changes will benefit from the use of the template.

\subsubsection{Modifications to the metamodel:}
\begin{itemize}
    \item  Due to problem P4 (see Table 5), it is necessary to modify the metamodel. Changes in \emph{context variables} are included within the \emph{operationalization dimension}. \emph{Context variables} are independent variables that were controlled at a fixed level during the experiment \cite{wohlin:experimentation}.
    
    This modification in the metamodel has repercussions on the definition of the template and L-patterns.
%    \item  Add the \emph{"sequence of steps"} as one of the possible elements to modify in the \emph{protocol dimension}.
\end{itemize}

\subsubsection{Modifications to the template and L-patterns:}
\begin{itemize}
    \item Due to problem P1 (see Table \ref{tab:tipos}), it is necessary to modify the pattern of \emph{threats to validity} to reflect that there are changes that do not affect the validity. 
    
    The new L-pattern to describe the  \emph{threats to validity} is:
    
     The change  \{(\{increases\textbar decreases \} \\
     \{the construct  \textbar external \textbar internal \textbar conclusion\} validity) \\
     \textbar  not affect the validity \}  \\
\end{itemize}

\subsection{Analysis of missing information}
\label{sec:missing}

%Due to the difficulty of their specification, some of the fields of the template are optional.  
In the three families analysed, among the fields not specified by the authors, \emph{"reason for the change"} and \emph{ "threat to validity"} stand out.   
The number of changes incompletely defined with the template due to \emph{"missing information"} is 8 out of 59. However, these fields are easily identifiable by the researchers who carry out the replication and fill in the template.

In order to clearly document the changes it is necessary to fill in all the fields in the template. This leads us to reinforce the template by making the fields that were optional mandatory so that this information is not missing.
It supports the idea that the template is necessary so that replications are not \textbf{under-specified}. 

To facilitate the identification of the \emph{threat to validity}, in \cite{gomez2014understanding} the threats are identified based on changes in the experimental baseline.

\begin{itemize}
%\item \emph{Changing the experimenters} does not control any threat. It only ensures that the experimenters do not influence the results \cite{Juristo2012}.
\item \emph{Changing the protocol:} Internal validity is addressed. Results are independent of experimental conditions.
\item \emph{Changing the operationalization:} Construct validity is addressed. %Determine limits for Operationalizations (range of variation of treatments and measures used ) \cite{Juristo2012}.
\item \emph{Changing the population:} External validity is addressed. %Determine limits in the population properties \cite{Juristo2012}.

\end{itemize}
%
\begin{table}
\caption{Types of problems identified and proposed solutions}
\label{tab:tipos}

\begin{tabular}{| l |p{5cm} |p{6cm} |}

\hline
\textbf{Id\#} & \textbf{Problems detected in the use of template} & \textbf{Proposed solutions
} \\
\hline
%P1-P
P1 & A \emph{dependent/independent variable} is suppressed or added. The \emph{operationalization dimension} is modified; no \emph{threat to validity} is identified. &  \textbf{Modify the template} so that the L-pattern allows you to specify that the change does not affect any \emph{threat to validity}. \\ \hline



P2 & The  \emph{reason for the change}  is not specified.  & The use of the template avoids this missing information. \\ \hline
P3 & The  \emph{name of the modified variable} is unknown. The change modifies the  \emph{operationalization dimension}.  & The use of the template avoids this missing information. \\ \hline
%T5 & It is not clear whether the change affects the operationalization or protocol dimension & The modified dimension is not identified \\ \hline
%T6 & It is not clear which element of the protocol dimension is affected by the change & The affected element is not identified \\ \hline
%T7 & The change modifies the protocol dimension & The affected element  is the change in the order of actions. This element is new \\ \hline
% P4-M Or
%P4 & The change consists of \emph{"alter the sequence of steps"}. It does not fit into any of the elements of the \emph{protocol dimension}. & \textbf{Modify the metamodel} to add the \emph{"sequence of steps"} as one of the possible elements to modify in the \emph{protocol dimension}.  \\ \hline
% P5-FI

% P6-M Co
P4 & The change consists of modifying a \emph{context variable}. The affected dimension is not identified.  & \textbf{Modify the metamodel} considering \emph{context variables} included in the \emph{operationalization dimension}. Therefore, it also implies \textbf{modifying the template}. \\ \hline

%T8 & The change modifies the operationalization dimension; the affected variable is not identified
 %\\ \hline

\end{tabular}
%\end{minipage}


%\caption{Types of problems identified and recommended solutions}
%\label{tab:tipos}
\end{table}





%% ------------------------------------------------------
% File    : FamTChangesSolV2.tex
% Content : Problems and solutions
% Date    : 1/12/2018
% Version : 1.0
% Authors : M.Cruz
% ------------------------------------------------------

\begin{table}[h]
  %\renewcommand{\arraystretch}{1.45}
  \caption{Types of problems identified and proposed solutions}
  \label{tab:tipos}
  \centering
	\scriptsize
  %begin{tabularx}{\textwidth}{cXX}
\begin{tabularx}{0.9\textwidth}{c
  >{\hsize=0.88\hsize}X
  >{\hsize=0.02\hsize}X
  >{\hsize=1.10\hsize}X}
  
	\hline\noalign{\smallskip}
		
    %\textbf{Id\#} & 
    %\textbf{Problems detected in the use of template} & \textbf{Proposed solutions} \\
    
    Id\# & 
    Problems detected in the use of template &  &
    Proposed solutions \\

	\noalign{\smallskip}\hline\noalign{\smallskip}
	P1 & A \emph{dependent/independent variable} is suppressed or added. The \emph{operationalization dimension} is modified; no \emph{threat to validity} is identified & & \textbf{Modify the template} so that the L-pattern allows you to specify that the change does not affect any \emph{threat to validity} \\ \\

    P2 & The  \emph{reason for the change}  is not specified  & &
    The use of the template avoids this missing information \\ \\
    
    P3 & The  \emph{name of the modified variable} is unknown. The change modifies the  \emph{operationalization dimension}  & &
    The use of the template avoids this missing information \\ \\
	
	P4 & The change consists of modifying a \emph{context variable}. The affected dimension is not identified  & &
	\textbf{Modify the metamodel} considering \emph{context variables} included in the \emph{operationalization dimension}. Therefore, it also implies \textbf{modifying the template} \\ 

	\noalign{\smallskip\smallskip}\hline
	\end{tabularx}  
\end{table}


\begin{table}
\caption{\textcolor[rgb]{1,0,0}{Probando} Analysis of the changes in the series of experiments on Code evaluation techniques}
\label{tab:changesVV}
\label{tab:1}       
\begin{minipage}{6cm}

\begin{tabular}{| l | l | l  | l |}

\hline
\textbf{Id} & \textbf{Original} & \textbf{Change}  &  \textbf{Type}\\
\hline

Fam\#1 & VV-UPM1 & Ch-1  & P1, P3 \\% Es P1 y P3 
 
~ & ~ & Ch-2  & P1 \\
~ & ~ & Ch-3 &   \ding{51} \\
~ & ~ & Ch-4 &   \ding{51} \\
~ & ~ & Ch-5 &   \ding{51} \\
~ & ~ & Ch-6 &   \ding{51} \\ 
~ & VV-UPV & Ch-1  & \ding{51} \\
~ & ~ & Ch-2  & \ding{51} \\
~ & ~ & Ch-3  & \ding{51} \\
~ & ~ & \textcolor[rgb]{1,0,0}{Ch-4}  & P4 \\
% T0, 
~ & ~ & Ch-5 & \ding{51} \\
~ & ~ & Ch-6 & \ding{51} \\ 

~ & VV-Uds & Ch-1 & \ding{51} \\
~ & ~ & Ch-2 & \ding{51} \\
% T0,  Protocolo en concreto el orden de ejecución
~ & ~ & Ch-3 & \ding{51} \\
~ & ~ & Ch-4 & \ding{51} \\
~ & ~ & \textcolor[rgb]{1,0,0}{Ch-5}  & P4 \\ 
% T0, No estaba claro 
~ & VV-ORT  & Ch-1  & \ding{51} \\
~ & ~ & Ch-2 & \ding{51} \\
%Instrumentalizacion en concreto el material experimental
~ & ~ & Ch-3  & P2 \\
~ & ~ & Ch-4  & \ding{51} \\
~ & ~ & Ch-5 & \ding{51} \\

%\noalign{\smallskip}\hline
\hline

\end{tabular}
\end{minipage}
%\caption{Comparison of related reviews}


\end{table}


%76 words




\subsection{Summary of specified changes}
\label{sec:summary}

%
\begin{table}
\caption{Analysis of the number of changes in the three families of experiments}
\label{tab:agrupa}
%\label{tab:1}       % Give a unique label
%\centering  
%\small
%\begin{tabular}{| l | c | c | p{2cm} | c | c |c |}
%\begin{tabular}{| l | c | p{2cm} | p{2cm} | c | p{1.5cm} | c |}
\begin{minipage}{6cm}

%\begin{tabular}{| l | l | l | l | l | l | l |}
\begin{tabular}{| p{6.5cm}| c | c | c |c |}
%\hline\noalign{\smallskip}
\hline
 & \textbf{Fam\#1} & \textbf{Fam\#2} & \textbf{Fam\#3} & Sum\\ \hline
Number of replications & 2 & 8 & 4 & 14\\ \hline
Number of changes  & 4 & 33 & 22 & 59\\ \hline
Average number of changes per replication & 2 & 4.1 & 5.5 & 4.2\\ \hline

Num. changes fully specified original template & 4 & 17 & 17 & 38\\ \hline
Num. changes with problem P1 & 0 & 6 & 1 & 7\\ \hline  %Template
Num. changes with problem P2 & 0 & 4 & 1 & 5\\ \hline %Incomp
Num. changes with problem P3 & 0 & 3 & 0 & 3\\ \hline %Incomp
Num. changes with problem P4 & 0 & 0 & 2 & 2\\\hline %Meta (Temp)
Num. changes with problems P1\&P2  & 0 & 2 & 0 & 2\\ \hline %Template + Incomp
Num. changes with problems P1\&P3 & 0 & 0 & 1 & 1\\ \hline %Template + Incomp
Num. changes with problems P2\&P4 & 0 & 1 & 0 & 1\\ \hline %Meta (Temp) + Incomp
 \hline 
Number of changes incompletely defined with the template due to missing information & 0 & 10 & 2 & 12\\ \hline % Con T3+T5  lack of information

Number of changes that lead to modify the original template & 0 & 8 & 2 & 10\\ \hline 

Number of changes that lead to modify the metamodel and therefore also the template & 0 & 1 & 2 & 3\\ \hline 

\end{tabular}
\end{minipage}
%\caption{Comparison of related reviews}


\end{table}


%76 words


% ------------------------------------------------------
% File    : Fam-AgrupaV2.tex
% Content : Analysis of changes
% Date    : 1/12/2018
% Version : 1.0
% Authors : M.Cruz
% ------------------------------------------------------

\begin{table}[h]
  %\renewcommand{\arraystretch}{1.45}
  \caption{Analysis of the number of changes in the three families of experiments}
  \label{tab:agrupa}
  \centering
	\scriptsize
  %begin{tabularx}{\textwidth}{cXX}
\begin{tabularx}{0.9\textwidth}{Xcccc}
  
	\hline\noalign{\smallskip}
		
    %\textbf{Id\#} & 
    %\textbf{Problems detected in the use of template} & \textbf{Proposed solutions} \\
    
    & \textbf{Fam\#1} & \textbf{Fam\#2} & \textbf{Fam\#3} & Sum \\

	\noalign{\smallskip}\hline\noalign{\smallskip}
	Number of replications & 2 & 8 & 4 & 14\\ 
	Number of changes  & 4 & 33 & 22 & 59\\ 
    Average number of changes per replication & 2 & 4.1 & 5.5 & 4.2\\ 
    Num. changes fully specified original template & 4 & 17 & 17 & 38\\ \hline
    Num. changes with problem P1 & 0 & 6 & 1 & 7\\ 
    Num. changes with problem P2 & 0 & 4 & 1 & 5\\ 
    Num. changes with problem P3 & 0 & 3 & 0 & 3 \\ 
    Num. changes with problem P4 & 0 & 0 & 2 & 2 \\
    Num. changes with problems P1\&P2  & 0 & 2 & 0 & 2 \\ 
    Num. changes with problems P1\&P3 & 0 & 0 & 1 & 1 \\ 
    Num. changes with problems P2\&P4 & 0 & 1 & 0 & 1 \\ \hline
    Sum. changes with problems (P1, P2, P3, P4) & 0 & 16 & 5 & 21 \\
    \hline
    \\ Number of changes incompletely defined with the template due to missing information & 0 & 10 & 2 & 12\\ 
    Number of changes that lead to modify the original template & 0 & 8 & 2 & 10\\ 
    Number of changes that lead to modify the metamodel and therefore also the template & 0 & 1 & 2 & 3\\ 
	\noalign{\smallskip\smallskip}\hline
	\end{tabularx}  
\end{table}

 
 Table \ref{tab:agrupa} shows the changes defined fitting the template and those that have presented difficulties.
By analyzing this table, two types of situations can be identified:
\begin{itemize}
\item \textbf{Due to limitations in the template}. Implies a change in the template (e.g. modify the pattern of \emph{threats to validity} or consider \emph{context variables}). Some of these situations also result in changes in the metamodel.
\item  \textbf{Due to missing information }. This situation does not imply changes in the template.  It supports the idea that the template is necessary so that applications are not \textbf{under-specified}.
\end{itemize}

